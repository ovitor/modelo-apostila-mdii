% ---
% Pacotes básicos 
% ---
\usepackage{lmodern}		% Usa a fonte Latin Modern			
\usepackage[T1]{fontenc}	% Selecao de codigos de fonte.
\usepackage[utf8]{inputenc}	% Codificacao do documento (conversão automática dos acentos)
\usepackage{lastpage}		% Usado pela Ficha catalográfica
\usepackage{indentfirst}	% Indenta o primeiro parágrafo de cada seção.
\usepackage{color}			% Controle das cores
\usepackage{xcolor}			% Controle das cores
\usepackage{graphicx}		% Inclusão de gráficos
\usepackage{microtype} 		% para melhorias de justificação
% ---
% Adicionado por Madson Dias
% ---

\graphicspath{ {figuras/graficos/} {figuras/imagens/}}  % Inclusão dos paths para imagens
\usepackage{xstring} 	% Criar comandos com IF
\usepackage{caption}
\usepackage{subcaption}
\usepackage{mdframed}
\usepackage{microtype}
\usepackage{pdfpages}
\usepackage{listings}
\usepackage{textcomp}
\usepackage{float}
\usepackage{chngcntr}		% Criar numeração de imagens por capítulo
\counterwithin{figure}{section}
\usepackage[os=win]{menukeys}
\usepackage{multicol}
\usepackage{framed}
% \usepackage{geometry}
% \usepackage{marginnote}

% ---
% Pacotes adicionais, usados apenas no âmbito do Modelo Canônico do abnteX2
% ---
\usepackage{lipsum}				% para geração de dummy text
% ---

% ---
% Pacotes de citações
% ---
\usepackage[brazilian,hyperpageref]{backref}	 % Paginas com as citações na bibl
\usepackage[alf]{abntex2cite}	% Citações padrão ABNT

% --- 
% CONFIGURAÇÕES DE PACOTES
% --- 

% ---
% Configurações do pacote backref
% Usado sem a opção hyperpageref de backref
\renewcommand{\backrefpagesname}{Citado na(s) página(s):~}
% Texto padrão antes do número das páginas
\renewcommand{\backref}{}
% Define os textos da citação
\renewcommand*{\backrefalt}[4]{
	\ifcase #1 %
		Nenhuma citação no texto.%
	\or
		Citado na página #2.%
	\else
		Citado #1 vezes nas páginas #2.%
	\fi}%
% ---

% ---
% Alterando o aspecto da cor azul
% ---
\definecolor{blue}{RGB}{41,5,195}

% --- 
% Espaçamentos entre linhas e parágrafos 
% --- 

% O tamanho do parágrafo é dado por:
\setlength{\parindent}{1.3cm}

% Controle do espaçamento entre um parágrafo e outro:
\setlength{\parskip}{0.2cm}  % tente também \onelineskip

\usepackage{utils/ejovem-modelo} 	% Pacote com informações da customização
\usepackage{utils/ejovem-programacao-web} 	% Pacote com informações da customização