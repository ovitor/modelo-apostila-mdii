%% abtex2-modelo-trabalho-academico.tex, v<VERSION> laurocesar
%% Copyright 2012-<COPYRIGHT_YEAR> by abnTeX2 group at http://www.abntex.net.br/ 
%%
%% This work may be distributed and/or modified under the
%% conditions of the LaTeX Project Public License, either version 1.3
%% of this license or (at your option) any later version.
%% The latest version of this license is in
%%   http://www.latex-project.org/lppl.txt
%% and version 1.3 or later is part of all distributions of LaTeX
%% version 2005/12/01 or later.
%%
%% This work has the LPPL maintenance status `maintained'.
%% 
%% The Current Maintainer of this work is the abnTeX2 team, led
%% by Lauro César Araujo. Further information are available on 
%% http://www.abntex.net.br/
%%
%% This work consists of the files abntex2-modelo-trabalho-academico.tex,
%% abntex2-modelo-include-comandos and abntex2-modelo-references.bib
%%

% ------------------------------------------------------------------------
% ------------------------------------------------------------------------
% abnTeX2: Modelo de Trabalho Academico (tese de doutorado, dissertacao de
% mestrado e trabalhos monograficos em geral) em conformidade com 
% ABNT NBR 14724:2011: Informacao e documentacao - Trabalhos academicos -
% Apresentacao
% ------------------------------------------------------------------------
% ------------------------------------------------------------------------

\documentclass[
	% -- opções da classe memoir --
	12pt,				% tamanho da fonte
	openright,			% capítulos começam em pág ímpar (insere página vazia caso preciso)
	twoside,			% para impressão em recto e verso. Oposto a oneside
	a4paper,			% tamanho do papel. 
	% -- opções da classe abntex2 --
	%chapter=TITLE,		% títulos de capítulos convertidos em letras maiúsculas
	%section=TITLE,		% títulos de seções convertidos em letras maiúsculas
	%subsection=TITLE,	% títulos de subseções convertidos em letras maiúsculas
	%subsubsection=TITLE,% títulos de subsubseções convertidos em letras maiúsculas
	% -- opções do pacote babel --
	english,			% idioma adicional para hifenização
	french,				% idioma adicional para hifenização
	spanish,			% idioma adicional para hifenização
	brazil,				% o último idioma é o principal do documento
	sumario=tradicional
	]{abntex2}

% ---
% Carregando pacotes necessários para gerar o PDF 
% ---
% ---
% Pacotes básicos 
% ---
\usepackage{lmodern}		% Usa a fonte Latin Modern			
\usepackage[T1]{fontenc}	% Selecao de codigos de fonte.
\usepackage[utf8]{inputenc}	% Codificacao do documento (conversão automática dos acentos)
\usepackage{lastpage}		% Usado pela Ficha catalográfica
\usepackage{indentfirst}	% Indenta o primeiro parágrafo de cada seção.
\usepackage{color}			% Controle das cores
\usepackage{xcolor}			% Controle das cores
\usepackage{graphicx}		% Inclusão de gráficos
\usepackage{microtype} 		% para melhorias de justificação
% ---
% Adicionado por Madson Dias
% ---

\graphicspath{ {figuras/graficos/} {figuras/imagens/}}  % Inclusão dos paths para imagens
\usepackage{xstring} 	% Criar comandos com IF
\usepackage{caption}
\usepackage{subcaption}
\usepackage{mdframed}
\usepackage{microtype}
\usepackage{pdfpages}
\usepackage{listings}
\usepackage{textcomp}
\usepackage{float}
\usepackage{chngcntr}		% Criar numeração de imagens por capítulo
\counterwithin{figure}{section}
\usepackage[os=win]{menukeys}
\usepackage{multicol}
\usepackage{framed}
% \usepackage{helvet}
% \usepackage{geometry}
% \usepackage{marginnote}

% ---
% Pacotes adicionais, usados apenas no âmbito do Modelo Canônico do abnteX2
% ---
\usepackage{lipsum}				% para geração de dummy text
% ---

% ---
% Pacotes de citações
% ---
\usepackage[brazilian,hyperpageref]{backref}	 % Paginas com as citações na bibl
\usepackage[alf]{abntex2cite}	% Citações padrão ABNT

% --- 
% CONFIGURAÇÕES DE PACOTES
% --- 

% ---
% Configurações do pacote backref
% Usado sem a opção hyperpageref de backref
\renewcommand{\backrefpagesname}{Citado na(s) página(s):~}
% Texto padrão antes do número das páginas
\renewcommand{\backref}{}
% Define os textos da citação
\renewcommand*{\backrefalt}[4]{
	\ifcase #1 %
		Nenhuma citação no texto.%
	\or
		Citado na página #2.%
	\else
		Citado #1 vezes nas páginas #2.%
	\fi}%
% ---

% ---
% Alterando espaçamento dos itens
% ---
\setlist[itemize]{noitemsep, topsep=0pt, leftmargin=1.75cm}
\setlist[enumerate]{noitemsep, topsep=0pt, leftmargin=1.75cm}
\setlist[description]{noitemsep, topsep=0pt, leftmargin=1.75cm}

% ---
% Alterando o aspecto da cor azul
% ---
\definecolor{blue}{RGB}{41,5,195}

% --- 
% Espaçamentos entre linhas e parágrafos 
% --- 

% O tamanho do parágrafo é dado por:
\setlength{\parindent}{1.3cm}

% Controle do espaçamento entre um parágrafo e outro:
\setlength{\parskip}{0.2cm}  % tente também \onelineskip

\usepackage{utils/ejovem-modelo} 	% Pacote com informações da customização
\usepackage{utils/ejovem-programacao-web} 	% Pacote com informações da customização

% ---
% Informações de dados para FOLHA DE ROSTO
% ---
% Informações do Autor e do trabalho
% ------------------------------------
\autor{<Nome do Autor>}
\titulo{<Título da Dissertação>}
\area{<Área de Concentração>}

\orientador{<Nome do Orientador>}
\coorientador{<Nome do Coorientador>} % Se você tem um coorientador, descomente esta linha

% Professores convidados para a banca
% ------------------------------------
% 
%  - Caso tenha um terceiro professor convidado, remova os comentários

\nomeprofessorA{<Nome do Professor A>}
\instituicaoprofessorA{<Instituição do Professor A> (<Sigla A>)}

\nomeprofessorB{<Nome do Professor B>}
\instituicaoprofessorB{<Instituição do Professor B> (<Sigla B>)}

%\nomeprofessorC{<Nome do Professor C>}
%\instituicaoprofessorC{<Instituição do Professor C> (<Sigla C>)}

% ---
% Informações do PDF
% ---
\makeatletter
\hypersetup{
     	%pagebackref=true,
		pdftitle={\@title}, 
		pdfauthor={\@author},
    	pdfsubject={\imprimirpreambulo},
	    pdfcreator={LaTeX with abnTeX2},
		pdfkeywords={abnt}{latex}{abntex}{abntex2}{apostila}, 
		colorlinks=true,       		% false: boxed links; true: colored links
    	linkcolor=blue,          	% color of internal links
    	citecolor=blue,        		% color of links to bibliography
    	filecolor=magenta,      		% color of file links
		urlcolor=blue,
		bookmarksdepth=4
}
\makeatother
% --- 

% ---
% compila o indice
% ---
\makeindex
% ---

% ----
% Início do documento
% ----
\begin{document}

% Seleciona o idioma do documento (conforme pacotes do babel)
%\selectlanguage{english}
\selectlanguage{brazil}

% Retira espaço extra obsoleto entre as frases.
\frenchspacing 

% ----------------------------------------------------------
% ELEMENTOS PRÉ-TEXTUAIS
% ----------------------------------------------------------
% \pretextual
\imprimircapa % Capa *

% inserir lista de ilustrações
% \pdfbookmark[0]{\listfigurename}{lof}
% \listoffigures*
% \cleardoublepage
% inserir lista de tabelas
% \pdfbookmark[0]{\listtablename}{lot}
% \listoftables* \cleardoublepage
% ---
\setlength{\absparsep}{18pt} % ajusta o espaçamento dos parágrafos do resumo
\begin{resumo}
 Segundo a \citeonline[3.1-3.2]{NBR6028:2003}, o resumo deve ressaltar o
 objetivo, o método, os resultados e as conclusões do documento. A ordem e a extensão
 destes itens dependem do tipo de resumo (informativo ou indicativo) e do
 tratamento que cada item recebe no documento original. O resumo deve ser
 precedido da referência do documento, com exceção do resumo inserido no
 próprio documento. (\ldots) As palavras-chave devem figurar logo abaixo do
 resumo, antecedidas da expressão Palavras-chave:, separadas entre si por
 ponto e finalizadas também por ponto.

 \textbf{Palavras-chave}: latex. abntex. editoração de texto.
\end{resumo}  	% Apresentação do programa e-jovem 
\begin{siglas}
  \item[ABNT] Associação Brasileira de Normas Técnicas
  \item[abnTeX] ABsurdas Normas para TeX
  \item[PHP] Pré-Processador de Hipertexto
\end{siglas}  			% Lista de abreviaturas e siglas
% \begin{simbolos}
\item[$ \Gamma $] Letra grega Gama
\item[$ \Lambda $] Lambda
\item[$ \zeta $] Letra grega minúscula zeta
\item[$ \in $] Pertence
\end{simbolos} 				% Lista de símbolos

% inserir o sumario
\pdfbookmark[0]{\contentsname}{toc}
\tableofcontents*
\cleardoublepage
% ---



% ----------------------------------------------------------
% ELEMENTOS TEXTUAIS
% ----------------------------------------------------------
\textual

% ---
% PHP 
% ---
\chapter{\php}
% ---

Ao final deste capítulo, o aluno terá as seguintes competências:
\begin{enumerate}
	\item Entender a arquitetura cliente-servidor;
	\item Instalar o servidor web (\apache) e a linguagem \php; e
	\item Testar o ambiente de desenvolvimento. 
\end{enumerate}

\section{\phpcompleto}

O \phpcompleto, foi criado por \phpcriador~em 1995 e originalmente chamado de 
“\textit{Personal Home Page Tools}” (Ferramentas para Página Pessoal). Com a 
aceitação do projeto, muitos programadores passaram a utilizar e propor mudanças,
surgindo assim, o \php~que iremos conhecer hoje. O \php~está atualmente na
versão 7.0, chamado de \php7 ou, simplesmente de \php. A nível de estudo, 
utilizaremos o \php~\phpversao, pois é uma versão mais estável e muito 
utilizada no mercado.

O \php~é uma linguagem de programação que funciona no lado do servidor, 
ele permite criarmos \sites~dinâmicos, ou seja, o \site~se comporta de acordo 
com a entrada de dados do usuário. Outros exemplos de linguagem semelhantes são 
ASP, JSP (Java) e Python.

A linguagem \php~trabalha lado a lado com o \htmlcompleto, por conta disso vamos
precisar saber o básico de \html, principalmente as \tags~de formulário. Devemos
lembrar que o \php~tem pouca relação com o \layout~ou eventos que compõem a 
aparência de uma página \web. Portanto, podemos dizer que a maior parte do que o
\php~realiza é invisível para o usuário final. O internauta, ao visualizar a 
página desenvolvida em \php~não será capaz de identificar que a página foi 
escrita utilizando a tecnologia disponibilizada pelo \php. 

Você arriscaria dizer que o Facebook foi desenvolvido com a linguagem \php?

\section{Arquitetura cliente-servidor}
\label{arquitetura-cliente-servidor}

Como visto na seção anterior, o \php~funciona do lado do servidor. Para entendermos
melhor isso, é necessário entender a estrutura cliente/servidor. Muito utilizada
na \internet. A figura abaixo exemplifica de maneira simples a comunicação entre
cliente e servidor.

\figurasimples{arquitetura-cliente-servidor}{Ilustração de uma requisição web.}

Dessa figura, podemos tirar algumas palavras chaves. Que são:
\begin{description}[noitemsep]
  \item [Recurso] Item disponível na Internet (uma figura, uma página, um arquivo .css);
  \item [Cliente] Aquele que \textbf{requisita} algum recurso (navegador Firefox); e
  \item [Servidor] Aquele que \textbf{provê} algum recurso (servidor \apache).
\end{description}

Portanto, quando o \textbf{cliente}, ou seja, o internauta, faz uma \textbf{requisição} 
- digitando na barra de endereços do navegador o \site~
\url{http://projetoejovem.seduc.ce.gov.br} e pressionando \keys{ENTER} - o navegador 
se encarrega de fazer um pedido ao \textbf{servidor} que guarda o \site~do projeto e-Jovem.

\begin{description}
  \item [Passo 1] Usuário requisita página de internet acessando o navegador e digitando
  o \site;
  \item [Passo 2] O servidor \web~se comunica com a linguagem \php;
  \item [Passos 3 e 4] O \php~e o \html~se juntam em um só arquivo;
  \item [Passo 5] O servidor tem agora a página pronta para ser enviada ao usuário; e
  \item [Passo 5] O usuário recebe a página completa em seu navegador.
\end{description}

\section{Instalação do \php}
\label{instalacao-do-php}

Para que possamos utilizar o \php, devemos instalar a linguagem no nosso computador
de trabalho. Vamos instalar esses pacotes através do \terminal. Podemos abrir o
\terminal de várias maneiras. Veja duas delas listadas abaixo:

\begin{enumerate}
	\item clique com o botão direito na área de trabalho, escolha a opção \\
	\opcao{Abra o Emulador de Terminal aqui}; e
	\item acione a combinação de teclas \altfdois e digite \xfceterminal.
\end{enumerate}

Em seguida escreva o comando abaixo no \terminal~que acabamos de abrir. Por segurança
a senha de usuário será requisitada, e \textbf{ela não aparece ao ser digitada}.
Não se preocupe, digite a senha e ao final aperte enter. 

\begin{lstlisting}[language=bash,style=codigos]
  $ sudo apt-get install php5 libapache2-mod-php5 php5-gd curl 
  	php5-curl php5-xmlrpc php5-cli
\end{lstlisting}

Se você estiver usando o Linux do Projeto e-Jovem, então esses pacotes já devem
ter sido instalados e você visualizou a seguinte tela.

\figurasimples{php-instalacao-ok}{Instalação do \php~bem sucedida.}

\section{Instalação do \apache}
\label{instalacao-do-apache}

O servidor \apache~é um dos principais aplicativos que fazem a \web~funcionar.
Ele é responsável por interpretar os arquivos \phpextensao~e retornar para o
cliente, apenas o que ele requisitou. A versão que vamos trabalhar é a \apacheversao.
O processo de instalação é parecido com o que foi utilizado no \php. Abra o
\terminal~utilizando um dos passos da seção \ref{instalacao-do-php} e digite a 
seguinte instrução.

\lstinputlisting[language=bash,style=codigos]{codigos/instalacao-apache.sh}

Se o sistema utilizado for o Linux do Projeto e-Jovem, então já temos o \apache
\apacheversao~instalado (figura \ref{fig:apache-instalacao-ok}). Digite no navegador 
Firefox o endereço de internet \url{http://localhost} (sem os sinais de maior e menor que). 
A tela será parecida com a mostrada na figura \ref{fig:apache-verificacao-ok}.

\figuradupla{apache-instalacao-ok}{Instalação do \apache \apacheversao~bem sucedida}
			{apache-verificacao-ok}{Verificação do \apache~em execução. Digite \url{http://localhost} no navegador Firefox}

A figura \ref{fig:apache-verificacao-ok} indica que o \apache~está funcionando corretamente.
O arquivo apresentado acima pode ser encontrado no diretório \dirpadrao. Será 
essa a localização dos arquivos que vamos desenvolver. Ou seja, sempre que criarmos
um arquivo \phpextensao~ele deverá ser salvo no \dirpadrao. 

Para que seja possível o usuário do sistema (no caso você) salve no \dirpadrao,
precisamos mudar a permissão de escrita do diretório. Vamos abrir o \terminal~
de acordo com o que foi mostrado na seção \ref{instalacao-do-php}. Com o \terminal~
aberto, digite o seguinte comando.

\begin{lstlisting}[language=bash,style=codigos]
  $ sudo chmod -R 777 /var/www 
\end{lstlisting}

O comando acima permite que o usuário comum do sistema grave arquivos no \dirpadrao.

O aplicativo \apache~pode ser configurado para funcionar de diversas maneiras. 
Essa disciplina necessita apenas da configuração básica. Caso queira modificá-la, 
o aluno poderá ler mais sobre o \apache~através do site: \url{http://httpd.apache.org/docs/2.2}.

Caso você use o sistema operacional Windows na sua casa, veja no apêndice 
\ref{ap:instalacao-env-windows}, lá é explicado como instalar o \php~e o \apache~no Windows.


\section{Testando o ambiente}
\label{testando-ambiente}

\subsection{Função phpinfo}
\label{subsection:funcao-phpinfo}

Após a instalação, devemos testar o nosso ambiente de desenvolvimento (composto
inicialmente por \php~e \apache). Abra novamente o \terminal~, navegue até o
diretório \dirpadrao.

Neste diretório, iremos criar \textbf{uma pasta para cada aula} do curso, portanto, hoje
criaremos o diretório \diraula{01}~no caminho \dirpadrao. Abra novamente o \terminal~
e digite os seguintes comandos:

\begin{lstlisting}[language=bash,style=codigos]
  $ cd /var/www 
  $ mkdir aula01
\end{lstlisting}

Lembre-se! É importante que o aluno crie em cada aula um diretório 
específico para aquela aula.

Testar o ambiente significa verificar se está tudo funcionando como deve ser. Vamos criar um 
arquivo \phpextensao. Usaremos o programa editor de textos \gedit, acione as teclas \altfdois 
e digite na janela o nome do programa: \gedit. 

A figura \ref{fig:gedit-ok} representa um código simples escrito no editor de textos \gedit. 
Perceba que o arquivo \phpextensao~começa com os caracteres \phpinicio~e \phpfim. 
Todo arquivo \phpextensao~tem essa estrutura no início e no fim. O conteúdo desse arquivo é 
a função \funcaophpinfo. Ela apresenta para nós, todas as opções que estão configuradas no 
nosso \php. A figura \ref{fig:gedit-ok} mostra o resultado ao acessarmos a URL \url{http://localhost} 
no nosso navegador Firefox.

Salve o arquivo no diretório \diretorio{/var/www/aula01}. Esse arquivo deve ter o nome \arquivo{index.php}.

\figurasimples{gedit-ok}{Código PHP escrito no programa \gedit.}

\subsection{Instrução echo}
\label{subsection:funcao-echo}
Outro teste que vamos fazer é a utilização da instrução \funcaoecho. Edite o arquivo 
\arquivo{index.php} com o programa \gedit~e adicione na primeira linha o código
\texttt{<?php echo "Aula 01"; ?>}

\lstinputlisting[language=php,style=codigos]{codigos/echo.php}

O resultado pode ser visto na figura abaixo.

\subsection{Comentários no \php}
\label{subsection:comentarios-no-php}

O último tópico deste capítulo trata dos comentários que podemos escrever nos nossos
arquivos \phpextensao. Comentários são partes importantes do código desenvolvido.
Eles servem para ajudar a enteder melhor determinadas partes do código ou ainda
para descrever o que o código desenvolvido realiza. Outra funcionalidade
importante dos comentários são os de orientar outros programadores, permitindo assim
que os desenvolvedores trabalhem em conjunto de maneira mais fácil.

No \php, os comentários uma linha são indicados pelos caracteres \verb$//$ ou \verb$#$
e os comentários de múltiplas linhas são representados pelos caracteres 
\verb$/*$ (início) e \verb$*/$ (fim).

Veja o nosso arquivo \arquivo{index.php} após a adição dos comentários explicativos.

\lstinputlisting[language=php,style=codigos]{codigos/echo-com-comentarios.php}

\section{Exercícios}
\label{cap1-exercicios}

Para realizar as questões abaixo, crie um documento no diretório \directory{/var/www/aula01} 
de nome \texttt{ejovem.php}.

\begin{description}[labelindent=19pt]
  \item [Q. 01] Como o arquivo \phpextensao~deve ser inicializado? Ou seja, o que deve vir no
  começo e no final do arquivo?
  \item [Q. 02] Exiba a mensagem ``projeto e-jovem'' utilizando o comando \funcaoecho'.
  \item [Q. 03] Adicione um comentário de uma linha com os dizeres: ``Primeira aula de PHP''.
  \item [Q. 04] Adicione um comentário de múltiplas linhas com cada palavra da frase 
  ``comentário de múltiplas linhas'' em uma linha.
  \item [Q. 05] Exiba a mensagem ``Olá Aluno!'' utilizando a \tag~\taghum~e o comando \funcaoecho.
\end{description}

\section{Desafio!}
\label{desafio}
O desafio deste capítulo é você encontrar na tela do navegador a versão do \php~
que vamos trabalhar, a versão do \apache~além de nos mostrar qual o diretório
padrão em que os arquivos \phpextensao~devem ser salvos. 				% Capítulo 1 - PHP
% ---
% Ambiente de desenvolvimento com Sublime Text
% ---
\chapter{Ambiente de desenvolvimento com Sublime Text}
\label{cap2}
% ---

Ao final deste capítulo, o aluno terá as seguintes competências:
\begin{enumerate}
	\item Instalar o \sublime~ \sublimeversao~ no sistema operacional;
	\item Instalar \plugins~ do \sublime; e
	\item Criar uma estrutura de aplicação \web~ com \php. 
\end{enumerate}

\section{O \sublime}
\label{o-sublime}

O \sublime~ é um editor de textos melhorado. Com esse tipo de \software~
é possível desenvolver os mais diversos programas, incluíndo os \sites. 
O \sublime~ hoje dispõe de duas versões que são amplamente utilizadas. 
A versão 2 - estável porém mais antiga e a versão \sublimeversao~ - caracterizada como 
versão \textit{beta} porém mais nova. Apesar de no curso sempre trabalharmos com 
as versões estáveis dos programas, com esse editor usaremos a versão \sublimeversao.

O editor de textos não vem por padrão nas distribuições Linux. Por isso, é necessário instalá-lo. 
Vamos abrir o navegador Firefox e digitar a URL \url{https://www.sublimetext.com/3}.
Em seguida, clique no \textit{link} ``Download''. Se você estiver usando o Linux do Projeto e-Jovem
escolha a opção ``Ubuntu 32 bit''. Essa ação vai baixar o arquivo \sublimefilename.
o \texttt{3XXX} indica a versão (\sublimeversao~ no caso) e pequenas mudanças na 
versão principal respectivamente.

Lembre-se! Saíba onde você salvou o arquivo baixado! Será importante saber a 
localização dele no momento da instalação!

\section{Instalando o \sublime}
\label{instalando-o-sublime}

Agora que baixamos o arquivo \texttt{.deb}, vamos instalá-lo. Abra o \terminal~ de acordo
com o apresentado na seção \ref{instalacao-do-php}. Utilize os comandos \comandocdcompleto, 
\comandolscompleto~ e \comandopwdcompleto~ para ir ao diretório que o arquivo 
\sublimefilename~ está salvo. No caso da apostila, o arquivo foi salvo dentro do 
diretório \directory{/home/ejovem/Downloads}.

Caso você não saíba onde o arquivo se encontra, peça ajuda ao seu instrutor.

\begin{multicols}{2}

  Comandos executados no terminal 
  \begin{lstlisting}[language=bash,style=codigos]
    $ cd ~/Downloads
    $ ls
  \end{lstlisting}

  \columnbreak

  Resposta do comando \comandolscompleto
  \begin{lstlisting}[language=bash, style=codigos]
    ...  
    sublime-text_build-3114_i368.deb
    ...  
  \end{lstlisting}

\end{multicols}

Agora que a gente sabe onde o arquivo se encontra, podemos instalar com o comando
\dpkg. Execute o seguinte código no diretório em que o \sublimefilename~ se encontra

  \begin{lstlisting}[language=bash,style=codigos]
    $ sudo dpkg -i sublime-text_build-3114_i368.deb
  \end{lstlisting}

Se a instalação do \sublime~ ocorreu com sucesso. Obtemos a seguinte tela na 
imagem \ref{fig:sublime-instalacao-ok}. Para abrir o programa, digite a combinação 
de teclas \altfdois e digite o comando \sublimebin. A tela inicial do programa é 
a exibida na figura \ref{fig:sublime-ok}.

\figuradupla{sublime-instalacao-ok}{Instalação do aplicativo \sublime~ realizada com sucesso}
{sublime-ok}{Tela inicial do \sublime}

\section{Primeiros passos}
\label{primeiros-passos}

Vamos começar nossos testes com o \sublime. Nosso primeiro teste é abrir o arquivo 
editado anteriormente pelo programa \gedit. Na tela inicial do \sublime, execute a seguinte
sequência de menus: \menu[,]{File, Open File...}. O arquivo deverá estar no diretório
\directory{/var/www/aula01}. E tem o nome de \arquivo{index.php}.

Vamos editar o arquivo. Troque a primeira linha do arquivo para que fique parecido com o
apresentado abaixo.

\lstinputlisting[language=php,style=codigos]{codigos/echo-sublime.php}

Lembre-se! Você pode colocar \tags~ \html~ dentro das instruções \php!

O resultado pode ser conferido abaixo.

\figurasimples{sublime-echo-ok}{Utilizando \funcaoecho.}

\section{Comandos do \sublime}
\label{comandos-do-sublime}

O nosso editor de textos tem um painel de comandos. Nele é possível aplicar comandos
relacionados ao \sublime. Os comandos que iremos utilizar são em inglês, portanto,
a tabela TAL trás a tradução de alguns termos utilizados.

Para acessar o painel digite a combinação de teclas \ctrlshiftp enquanto estiver
no \sublime. O painel deve aparecer na parte superior da janela. Digite o termo
\sidebar. Esse termo ativará o comando de nome \textit{View: Toggle Side Bar}.
Esse comando ativa ou desativa a barra lateral no \sublime~ (pode ser utilizado
a combinação de teclas \sublimesidebar). 

Outro comando que podemos utilizar é o de ativar ou desativar o \textit{minimap}. 
Pressione então as teclas \ctrlshiftp e digite o termo \textit{minimap}. 
O comando completo é \textit{View: Toggle Minimap}.

\section{Adicionando diretórios}
\label{adicionando-diretorios}

Uma das grandes facilidades do editor de textos \sublime~ e outros editores mais
avançados é a apresentação do projeto através da barra lateral (\textit{sidebar}).
No nosso caso, um projeto pode ser descrito simplesmente como uma pasta. Então
vamos adicionar a pasta \directory{aula01} ao \sublime. 

Para isso, abra o programa gerenciador de arquivos do sistema. No caso da 
distribuição do e-Jovem, o programa a ser aberto é o \thunar. Pressione a 
combinação de teclas \altfdois em seguida digite \thunar. Navegue até o 
diretório \directory{/var/www/} clique em cima da pasta \directory{aula01} 
segure e arraste até o editor de textos \sublime.

Nesse momento a tela principal do \sublime encontra-se com a \sideb (barra lateral)
ativada e apresentando o diretório que acabamos de adicionar.

Dica! Outro método que podemos utilizar para realizar a mesma tarefa é utilizar
os menus do \sublime. O caminho a ser percorrido é: \menu[,]{File, Open Folder...}
e escolher o diretório \directory{/var/www/aula01}.


\section{Configurações do Usuário}
\label{configuracoes-do-usuario}

O \sublime~ permite diversas modificações no seu funcionamento a partir das configurações
que o usuário (no caso você) pode realizar. Você deve editar o arquivo 
que se encontra no menu \menu[,]{Preferences, Settings - User}.

Edite o arquivo para que fique parecido com o código abaixo. Você pode procurar
na internet por mais comandos que modifiquem o comportamento do \sublime.

\lstinputlisting[style=codigos]{codigos/settings-user.json}

\section{Configurações e Plugins}
\label{configuracoes-e-plugins}

O \sublime~ se tornou famoso por conta da possibilidade de adicionarmos à ele,
diversas outras funcionalidades por meio de \plugins. Existem \plugins~ que auxiliam
no desenvolvimento de \html, \css~ e em diversas outras linguagens.

Para tirar o maior proveito dos \plugins, vamos adicionar um gerenciador de
\plugins ao nosso \sublime.

Vamos acessar o site \url{http://packagecontrol.io/installation} através do navegador 
Firefox. Copie toda o código representado na figura \ref{fig:sublime-packagecontrol-installation}.

Agora devemos ir até o \sublime e seguir o caminho \menu[,]{View, Show console...}.
Uma pequena seção irá aparecer na parte debaixo da tela principal do programa.
Usando o botão direito do \mouse, clique no campo como mostrado na figura 
\ref{fig:sublime-packagecontrol-paste-code} escolha a opção \textit{paste} (colar em inglês)
e pressione \avancar. Após a finalização da instalação, percorra o menu \menu[,]{View, Hide console...} 
para esconder o \terminal.

\figuradupla{sublime-packagecontrol-installation}{Código que deve ser copiado, 
em seguida, colado no \sublime.}{sublime-packagecontrol-paste-code}{Opção a ser
escolhida para colar código copiado anteriormente}

\subsection{Plugin \texttt{php-snippets}}
\label{plugin-php-snippets}

Como teste, instalaremos o pacote \texttt{php-snippets}. Esse
pacote nada mais é do que um conjunto de códigos pré-prontos. Inicialmente não
nos terá muita serventia, mas com o passar das aulas será bem útil.

Para instalar um \plugin~ precisamos executar alguns comandos no painel de comandos.
Com o \sublime~ aberto, digite a combinação de teclas \ctrlshiftp e digite
\textit{package install}. Esse termo ativará o comando \textit{Package Control: Install Package}.
Ao pressionar \avancar~ o painel de comandos listará todos os pacotes disponíveis
para instalação. Digite o termo \textit{php-snippets}, ativando assim, o pacote
de nome \textit{php-snippets}. Veja a figura

\figurasimples{sublime-plugin-installation}{Instalação do \plugin~ \texttt{php-snippets}.} 

\section{Desafio!}
\label{cap2-desafio}
O desafio deste capítulo é você configurar o \sublime~ de acordo com o seu gosto.
Modifique os seguintes itens:
\begin{enumerate}
  \item Tamanho e tipo da fonte.
  \item Cor do editor (\menu[,]{Preferences, Color Scheme, <opção>})
  \item Instalar o \plugin~ \texttt{phpdoc}~ (\url{https://packagecontrol.io/packages/PhpDoc}).
\end{enumerate}
 				% Capítulo 2 - Sublime Text
% ---
% Variáveis em PHP 
% ---
\chapter{Variáveis em PHP}
% ---

Ao final deste capítulo, o aluno terá as seguintes competências:
\begin{enumerate}
    \item Entender o que são variáveis e como o \php trabalha com elas; e
    \item Trabalhar com os diversos tipos de variáveis.
\end{enumerate}

\section{Variáveis}
\label{variaveis}

Variáveis são identificadores criados para guardar valores por determinado tempo. 
No \php elas são declaradas, inicializadas e armazenadas na memória RAM do servidor \web. 
Esse é um dos motivos pelo qual os servidores precisam de grandes quantidades de memória.

Quando desenvolvemos um \site e o disponibilizamos na \internet. É nosso desejo que ele seja
acessado pela maior quantidade de pessoas possível. Portanto, imagine um servidor com mais 
de 20 mil acessos simultâneos (mais de 20 mil pessoas visualizando o \site no mesmo momento).
Nesse processo são criadas variáveis diferentes para cada usuário, logo, isso faz com que o 
processamento que o servidor faz se intensifique. Por conta disso, o servidor deve ser um 
computador com bastante memória RAM.

Uma variável é inicializada no momento em que é feita a primeira atribuição. O tipo da
variável será definido de acordo com o valor atribuído. Esse é um fator importante \php, 
pois uma mesma variável pode ser de um tipo e pode assumir no decorrer do código outro 
valor de tipo diferente.

Para criar uma variável em PHP, precisamos atribuir-lhe um nome de identificação, 
sempre precedido pelo caractere cifrão (\textit{\$}). Observe um exemplo:

\lstinputlisting[language=php,style=codigos]{codigos/variaveis-1.php}

Modifique o código acima para imprimir na tela o valor da variável de 
nome \texttt{\$sobrenome}.

Observação! Podem acontecer erros na exibição das mensagens por conta 
das codificações de acentuação. Caso isso aconteça, mude a codificação 
do seu navegador ou utilize as metas de codificação. Para mudar a codificação
do Firefox aperte a tecla \keys{\Alt} para exibir a barra de menus e em seguida
percorra o caminho \menu[,]{Exibir, Codificação, Unicode}.

Nomes de variáveis devem ser significativas e transmitir a ideia de seu conteúdo 
dentro do contexto no qual está inserido.
Utilize preferencialmente palavras em minúsculo (separadas pelo caracter \texttt{\_}) ou 
somente as primeiras letras em maiúsculo quando você tiver duas ou mais palavras. 
Veja o exemplo abaixo. 

\lstinputlisting[language=php,style=codigos]{codigos/variaveis-2.php}

Dicas! \newline
\begin{itemize}
    \item Nunca inicie a nomenclatura de variáveis com números. Ex: \texttt{\$1nota;}
    \item Nunca utilize espaço em branco no meio do identificados da variável. Ex: \texttt{\$nome um;}
    \item Nunca utilize caracteres especiais (\texttt{! @ \# \& * | [ ] \{ \} $\backslash$ \^ } 
    entre outros) na nomenclatura das variáveis.
    \item Evite criar variáveis com nomes grandes demais em virtude da clareza do código-fonte.
    \item Com exceção de nomes de classes e funções, o \php~ é \textit{case sensitive}, ou seja, 
    é sensível a letras maiúsculas e minúsculas. Tome cuidado ao declarar variáveis. Por exemplo: 
    a variável \texttt{\$codigo} é diferente da variável \texttt{\$Codigo}.
\end{itemize}


\section{Tipos de Variáveis}
\label{tipos-de-variaveis}


\section{Desafio!}
\label{cap3-desafio}
O desafio deste capítulo é você configurar o \sublime~ de acordo com o seu gosto.
Modifique os seguintes itens:
\begin{enumerate}
  \item Tamanho e tipo da fonte.
  \item Cor do editor (\menu[,]{Preferences, Color Scheme, <opção>})
  \item Instalar o \plugin~ \texttt{phpdoc}~ (\url{https://packagecontrol.io/packages/PhpDoc}).
\end{enumerate}
 				% Capitulo 3 - Variáveis no PHP
% ---
% Operadores em PHP 
% ---
\chapter{Operadores em PHP}
% ---

Ao final deste capítulo, o aluno terá as seguintes competências:
\begin{enumerate}
    \item Trabalhar com operadores aritméticos; 
    \item Trabalhar com operadores relacionais; e
    \item Trabalhar com operadores de \tipostring.
\end{enumerate}

\section{Operadores}
\label{operadores}

Os operadores têm um papel importante dentro de qualquer linguagem de programação.
É através deles que podemos realizar diversas operações em um programa. Existem
operadores para atribuição, operadores aritméticos, operadores relacionais ou lógicos,
e por fim, operadores de \tipostring.

No \php, os operadores são utilizados constatemente e nesse capítulo iremos
aprender a trabalhar com a maioria deles.

\section{Operador de atribuição}
\label{operador-atribuicao}

O operador básico de atribuição é o caractere ``='' (igual). Com ele podemos atribuir 
valores as variáveis como foi visto em exemplos anteriores. Isto quer dizer que o 
operando da esquerda recebe o valor da expressão da direita, ou seja, a variável da 
esquerda contém o valor da direita do símbolo igual ``=''. Observe o exemplo abaixo:

\lstinputlisting[language=php,style=codigos]{codigos/operadores-atribuicao.php}

\section{Operadores de \textit{strings}}
\label{operadores-de-strings}
Os operadores de \tipostrings são utilizados para manipular o conteúdo de uma \tipostring.
O \php~ disponibiliza, basicamente, dois operadores de \tipostrings. O primeiro é o operador
de concatenação (\texttt{.}) - ele retorna a concatenação das variáveis envolvidas.
O segundo operador, é o operador de atribuição e concatenação (\texttt{.=}). Ele acrescenta
à variável do lado direito na variável do lado esquerdo do operador. Verifique os exemplos
abaixo.

\lstinputlisting[language=php,style=codigos]{codigos/operadores-string-1.php}

No exemplo acima, pode-se observar, na atribuição da variável \texttt{\$informacao} que, 
temos uma inicialização e atribuição de concatenação em uma mesma linha. Isso é possível
no \php, por mais que seja mais otimizado (mais rápido de ser processado), o código se
torna menos legível (mais difícil de ser entendido).

{\Large Pequeno desafio!}

Utilize o conhecimento adquirido nas aulas passadas para exibir, no navegador, as informações
separadas em células de uma tabela. A imagem a seguir deve ser o resultado apresentado. 

\figurasimples{cap4-pequeno-desafio-1}{Resultado do primeiro pequeno desafio.}

\section{Operadores Aritméticos}
\label{operadores-aritmeticos}

Os operadores aritméticos são utilizados para realizar cálculos matemáticos básicos, tais
como: soma, subtração, divisão e multiplicação. Os símbulos mais utilizados são descritos
na tabela abaixo.

\begin{table}[h]
\scalefont{1}
\caption{Símbolos matemáticos.}\label{tab:cap4-operadores-aritmeticos}
\begin{center}
\begin{tabular}{|c|c|c|c|}
\hline
  \multicolumn{1}{|c|}{ \textbf{Operação}}
&  \multicolumn{1}{|c|}{ \textbf{Operador}} 
&  \multicolumn{1}{|c|}{ \textbf{Exemplo}} 
&  \multicolumn{1}{|c|}{ \textbf{Resposta}} \\
\hline
\hline
Adição                    & + & \$a = 3 + 5 & 8   \\ \hline
Subtração                 & - & \$a = 6 - 2 & 4   \\ \hline
Multiplicação             & * & \$a = 2 * 5 & 10  \\ \hline
Divisão                   & / & \$a = 15 / 3 & 5  \\ \hline
Módulo (resto da divisão) & \% & \$a = 9 \% 2 & 1  \\ \hline
Negação                   & - & \$a = -3 & -3  \\ \hline
\end{tabular}
\end{center}
\end{table}


Na tabela acima fizemos operações básicas sem utilizar parênteses. O uso de parênteses
segue o mesmo princípio da matemática. Ele serve para dar prioridade a determinado
cálculo.

\section{Operadores Combinados}
\label{operadores-combinados}

No \php~ é possível combinar os dois operadores visto acima (de atribuição e artiméticos).
A partir deles é possível programar de forma mais ágil. Observe o código a seguir.

\begin{multicols}{2}

  \lstinputlisting[language=php,style=codigos]{codigos/operadores-atribuicao-combinada-1.php}
  \columnbreak

  \lstinputlisting[language=php,style=codigos]{codigos/operadores-atribuicao-combinada-2.php}

\end{multicols}

No código acima do lado esquerdo, declaramos a variável \texttt{\$peso} com um valor inicial,
em seguida utilizamos o operador combinado \texttt{+=} para incrementar o valor da
varíavel \texttt{\$peso}. Podemos obter o mesmo resultado com o código acima do lado direito.
Escrevemos menos no código do lado esquerdo, por isso, os programadores o utilizam bastante.

Já no código abaixo, utilizamos o operador combinado \texttt{.=}. Ele concatena as \tipostrings.

\lstinputlisting[language=php,style=codigos]{codigos/operadores-atribuicao-combinada-3.php}

Nesse exemplo utilizamos o operador de atribuição básico.

\lstinputlisting[language=php,style=codigos]{codigos/operadores-atribuicao-combinada-4.php}

A tabela abaixo lista os principais operadores de atribuição. 

\begin{table}[h]
\scalefont{1}
\caption{Operadores de atribuição combinados.}\label{tab:cap4-operadores-combinados}
\begin{center}
\begin{tabular}{|c|c|c|c|}
\hline
  \multicolumn{1}{|c|}{ \textbf{Operadores}}
&  \multicolumn{1}{|c|}{ \textbf{Descrição}} \\
\hline
\hline
=             & Atribuição simples   \\ \hline
+=             & Soma, em seguida, atribui   \\ \hline
-=             & Subtrai, em seguida, atribui   \\ \hline
*=             & Multiplica, em seguida, atribui   \\ \hline
/=  & Divide, em seguida, atribui   \\ \hline
\%=             & Tira o módulo, em seguida, atribui   \\ \hline
.=             & Concatena, em seguida, atribui   \\ \hline
\end{tabular}
\end{center}
\end{table}


\section{Operadores de decremento e incremento}
\label{operadores-de-decremento-e-incremento}

Os operadores exemplicificados nessa seção são usados para \textbf{somar} ou \textbf{subtrair}
o valor 1 (um) a variável. Esse cálculo pode ser feito antes ou depois da execução de 
determinada variável. A tabela abaixo mostra tais operadores.

\begin{table}[h]
\scalefont{1}
\caption{Operadores de incremento e decremento.}\label{tab:cap4-operadores-incremento-e-decremento}
\begin{center}
\begin{tabular}{|c|c|c|c|}
\hline
  \multicolumn{1}{|c|}{ \textbf{Operadores}}
&  \multicolumn{1}{|c|}{ \textbf{Descrição}} \\
\hline
\hline
\$b = ++\$a             & Incrementa o valor de \$a, e atribui à \$b (pré-incremento)  \\ \hline
\$b = \$a++             & Atribui à \$b, em seguida, incrementa o valor de \$a (pós-incremento)  \\ \hline
\$b = --\$a             & Decrementa o valor de \$a, e atribui à \$b (pré-decremento)  \\ \hline
\$b = \$a--             & Atribui à \$b, em seguida, incrementa o valor de \$a (pós-decremento)  \\ \hline
\end{tabular}
\end{center}
\end{table}


O exemplo abaixo mostra a comparação na forma da escrita. Os resultados alcançados serão
os mesmos.

\lstinputlisting[language=php,style=codigos]{codigos/operadores-incremento-1.php}

\section{Operadores relacionais}
\label{operadores-relacionais}

Os operadores relacionais são utilizados para realizar comparações entre valores (variáveis)
ou expressões. Essa comparação sempre resulta em um valor do tipo \booleano, ou seja,
verdadeiro (\true) ou falso (\false). Na tabela a seguir, listamos os operadores
que iremos trabalhar.

\begin{table}[h]
\scalefont{1}
\caption{Operadores relacionais.}\label{tab:cap4-operadores-relacionais}
\begin{center}
\begin{tabular}{|c|c|c|}
\hline
  \multicolumn{1}{|c|}{ \textbf{Operadores}}
&  \multicolumn{1}{|c|}{ \textbf{Nome}}
&  \multicolumn{1}{|c|}{ \textbf{Descrição}} \\
\hline
\hline
==            & Igual               & \specialcell{Resulta em \true~ se as \\ expressões forem iguais} \\ \hline
===           & Idêntico            & \specialcell{Resulta em \true~ se as \\ expressões forem iguais e do mesmo tipo} \\ \hline
!= ou <>      & Diferente           & \specialcell{Resulta em \true~ se as \\ expressões forem diferentes} \\ \hline
<             & Menor que           & \specialcell{Resulta em \true~ se a primeira \\ expressão for menor que a segunda expressão} \\ \hline
>             & Maior que           & \specialcell{Resulta em \true~ se a primeira \\ expressão for maior que a segunda expressão} \\ \hline
<=            & Menor ou igual que  & \specialcell{Resulta em \true~ se a primeira \\ expressão for menor ou igual a segunda expressão} \\ \hline
>=            & Maior ou igual que  & \specialcell{Resulta em \true~ se a primeira \\ expressão for maior ou igual a segunda expressão} \\ \hline
\end{tabular}
\end{center}
\end{table}


O exemplo de código abaixo descreve o uso de alguns operadores relacionais.

\lstinputlisting[language=php,style=codigos]{codigos/operadores-relacionais-1.php}

\subsection{Operador condicional ternário}
\label{operador-condicional-ternario}

É importante explicar o operador condicional ternário (símbolo ``?'') para termos um exemplo 
prático dos operadores relacionais. A sintaxe do operador condicional é exemplificada abaixo:

\lstinputlisting[language=php,style=codigos]{codigos/operadores-condicional-ternario.php}

A expressão 1 é sempre um teste a ser realizado com os operadores relacionais. A expressão
2 e expressão 3 são valores que serão atribuídos a variável \texttt{\$var}. Vamos ver um exemplo:

\lstinputlisting[language=php,style=codigos]{codigos/operadores-condicional-ternario-exemplo.php}

Qual será o resultado se mudarmos a expressão 1 (teste) para \texttt{\$num1 === \$num2}?

{\Large Desafio rápido!}

Faça um pequeno programa que utilize todos os operadores relacionais descritos acima.
É necessário criar exemplos que os resultados sejam \true~ e \false~ para cada
um dos casos. Utilize também o operador condicional ternário. A saída no navegador
deve ser parecida com a imagem abaixo.

\figurasimples{cap4-desafio-rapido-1}{Resultado do primeiro desafio rápido.}

\section{Operadores Lógicos}
\label{operadores-logicos}

Operadores lógicos, também conhecidos como operadores \booleanos, são utilizados para
avaliar expressões lógicas, ou seja, expressões que resultem em valores \booleanos~
(verdadeiro ou falso). A tabela a seguir lista cada um desses operadores e sua respectiva
função.

\begin{table}[h]
\scalefont{1}
\caption{Operadores lógicos.}\label{tab:cap4-operadores-logicos}
\begin{center}
\begin{tabular}{|c|c|c|}
\hline
  \multicolumn{1}{|c|}{ \textbf{Operadores}}
&  \multicolumn{1}{|c|}{ \textbf{Função}} \\
\hline
\hline
NÃO (NOT)       & Negação \\ \hline
E (AND)         & Conjunção \\ \hline
OU (OR)         & Disjunção \\ \hline
\end{tabular}
\end{center}
\end{table}


Com esses operadores, podemos construir a tabela verdade de determinada variável \booleana.
As próximas seções explicam os operadores lógicos \texttt{NOT}, \texttt{AND} e \texttt{OR}.

\subsection{Operador \texttt{NOT}}
\label{operador-not}

Para isso, vamos utilizar os dois valores que a variável pode assumir (Verdadeiro ou Falso)
e o operador lógico \texttt{NOT} listado na tabela \ref{tab:cap4-operadores-logicos}.
Veja a tabela abaixo, a variável \texttt{\$esta\_chovendo} pode assumir dois valores,
verdadeiro ou falso.

\begin{table}[h]
\scalefont{1}
\caption{Operação de negação.}\label{tab:cap4-operacao-de-negacao}
\begin{center}
\begin{tabular}{|c|c|c|}
\hline
  \multicolumn{1}{|c|}{ \textbf{\texttt{\$esta\_chovendo}}}
&  \multicolumn{1}{|c|}{ \textbf{NOT \texttt{\$esta\_chovendo}}} \\
\hline
\hline
Falso       & Verdadeiro \\ \hline
Verdadeiro  & Falso \\ \hline
\end{tabular}
\end{center}
\end{table}


Ou seja, considerando que a variável \texttt{\$esta\_chovendo} pode ser traduzida para
a frase: ``Hoje está chovendo!'', sua negação seria: ``Hoje \textbf{não} está chovendo''.

\subsection{Operador \texttt{AND}}
\label{operador-and}

O operador \texttt{AND} funciona com base em uma conjunção. Algo só será verdadeiro
se os termos envolvidos na expressão sejam verdadeiros. Nosso próximo exemplo leva em 
consideração a situação de um casamento. Então criamos duas variáveis: \texttt{\$noiva\_presente} 
e \texttt{\$noivo\_presente}. Se apenas o noivo estiver presente, ou seja,
\texttt{\$noivo\_presente = True} e \texttt{\$noiva\_presente = False} o casamento não acontece. 
Observe na tabela verdade abaixo.

\begin{table}[h]
\scalefont{1}
\caption{Operação de conjunção.}\label{tab:cap4-operacao-de-conjuncao}
\begin{center}
\begin{tabular}{|c|c|c|}
\hline
  \multicolumn{1}{|c|}{ \textbf{\texttt{\$noivo\_presente}}}
&  \multicolumn{1}{|c|}{ \textbf{\texttt{\$noiva\_presente}}}
&  \multicolumn{1}{|c|}{ \textbf{\texttt{\$noivo\_presente} AND \texttt{\$noiva\_presente}}} \\
\hline
\hline
Falso  & Falso & Falso\\ \hline
Falso  & Verdadeiro & Falso\\ \hline
Verdadeiro  & Falso & Falso\\ \hline
Verdadeiro  & Verdadeiro & Verdadeiro\\ \hline
\end{tabular}
\end{center}
\end{table}


\subsection{Operador \texttt{OR}}
\label{operador-or}

O operador \texttt{OR} funciona com base em uma disjunção. Algo será verdadeiro
assim que algum termo envolvido na expressão seja verdadeiro. Por exemplo, Joaquim,
aluno do programa e-Jovem, vai fazer uma prova de concurso. Para realizá-la, é 
obrigatório apresentar documento de identidade (variável \texttt{\$tem\_identidade})
\textbf{ou} o título de eleitor (variável \texttt{\$tem\_tituloeleitor}).
Joaquim só vai conseguir fazer a prova caso apresente um documento \textbf{ou} outro.
Dessa maneira, observe a tabela verdade do operador \texttt{OR}.

\begin{table}[H]
\scalefont{1}
\caption{Operação de disjunção.}\label{tab:cap4-operacao-de-disjuncao}
\begin{center}
\begin{tabular}{|c|c|c|}
\hline
  \multicolumn{1}{|c|}{ \textbf{\texttt{\$tem\_identidade}}}
&  \multicolumn{1}{|c|}{ \textbf{\texttt{\$tem\_tituloeleitor}}}
&  \multicolumn{1}{|c|}{ \textbf{\texttt{\$tem\_identidade} OR \texttt{\$tem\_tituloeleitor}}} \\
\hline
\hline
Falso  & Falso & Falso\\ \hline
Falso  & Verdadeiro & Verdadeiro\\ \hline
Verdadeiro  & Falso & Verdadeiro\\ \hline
Verdadeiro  & Verdadeiro & Verdadeiro\\ \hline
\end{tabular}
\end{center}
\end{table}


Verifique no código abaixo o uso dos operadores lógicos no \php.

\lstinputlisting[language=php,style=codigos]{codigos/operadores-logicos-1.php}

\section{Exercícios}
\label{cap4-exercicios}


\section{Desafio!}
\label{cap4-desafio}
O desafio deste capítulo é criar um arquivo \phpextensao~ declarando variáveis de todos os tipos
estudados (inteiro, ponto flutuante e \tipostring). Imprimí-los no navegador e por fim, converter
os valores das variáveis do tipo inteiro e ponto flutuante para \tipostring.
 				% Capitulo 4 - Operadores no PHP
% ---
% Estruturas de controle e repetição 
% ---
\chapter{Estruturas de controle e repetição}
% ---

Ao final deste capítulo, o aluno terá as seguintes competências:
\begin{enumerate}
    \item Trabalhar com estruturas de controle; e 
    \item Trabalhar com estruturas de repetição.
\end{enumerate}

As estruturas que veremos a seguir são comuns para várias linguagens de programação.
Entretanto é necessário que a apostila descreva a sintaxe dessas estruturas, resumindo
seu funcionamento.

É necessário ainda, entender o conceito de bloco. Um bloco consiste de vários
comandos agrupados com o objetivo de relacioná-los com determinado comando
ou função. Podemos (e devemos) usar blocos de comandos nas instruções que serão
vistas nesse capítulo (\comandoif, \comandofor, \comandowhile, \comandoswitch).
Os blocos de comandos devem ser utilizados para permitir que um conjunto de instruções
façam parte do mesmo contexto desejado.

Blocos em \php~ são delimitados pelos caracteres \texttt{\{} e \texttt{\}}.

\section{Comando \comandoif}
\label{comando-if}

Essa estrutura condicional está entre as mais usadas na programação. Sua finalidade é induzir um 
desvio condicional, ou seja, um desvio na execução natural do programa. Caso a condição (veja sintaxe abaixo) 
seja verdadeira, então serão executadas a instruções do bloco de comando. 
Caso a condição não seja satisfeita, o bloco de comando será simplesmente ignorado. 
Veja a sintaxe e em seguida um exemplo real.

\lstinputlisting[language=php,style=codigos]{codigos/estruturas-if-1.php}

Se o teste realizado no \comandoif~ for falso, ou seja, a expressão \texttt{\$media >= 7} der um
resultado \booleano~ falso, o bloco de comandos não será executado.

\subsection{Comando \comandoifelse}
\label{comando-if-else}

Um complemento do comando \comandoif~ é a adição da palavra chave \texttt{else}. Com esse termo
é possível tratar também as opções em que o teste é falso. Veja como fica a sintaxe da instrução
completa.

\lstinputlisting[language=php,style=codigos]{codigos/estruturas-if-2.php}

Como pode ser visto, o teste realizado \texttt{\$media >= 7} gerou um valor \booleano~
falso, resultando na execução do código relacionado ao bloco de comandos \texttt{else},
fazendo com que a mensagem exibida na tela do navegador seja ``Aluno em recuperação''.

O programador tem a possibilidade de adicionar quantos comandos \comandoifelse~
forem nescessários. Esses códigos são chamados de ``\texttt{ifs} encadeados''. 
Para facilitar, o \php~ criou a palavra chave \texttt{elseif} (tudo junto mesmo).
Veja a sintaxe de uso.

\lstinputlisting[language=php,style=codigos]{codigos/estruturas-if-3.php}

{\Large Desafio rápido!}

Faça um pequeno programa que utilize a idade de um nadador, a partir dela
o programa deve indicar em qual categoria o nadador irá concorrer no próximo
campeonato. Crianças menores de 5 anos não podem competir. 
Veja como deve ser a saída do seu programa. 

\figurasimples{cap4-desafio-rapido-1}{Resultado do primeiro desafio rápido.}

\section{Atribuição condicional (ternário)}
\label{atribuicao-condicional}

Esse tópico foi visto brevemente anteriormente. Vamos relembrar a sintaxe.

\lstinputlisting[language=php,style=codigos]{codigos/operadores-condicional-ternario-1.php}

Essa instrução se aplica quando queremos uma estrutura resumida, onde podemos ter um 
resultado mais direto, como por exemplo, atribuir um valor a uma variável dependendo de uma
expressão. Observe o exemplo abaixo: a variável \texttt{\$texto} receberá o valor 
``menor de 18'' ou ``maior de 18'' de acordo com o teste \texttt{\$idade > 18}.

\lstinputlisting[language=php,style=codigos]{codigos/operadores-condicional-ternario-2.php}
 
É uma estrutura semelhante ao comando \comandoifelse. Cabe ao programador escolher onde
cada uma das estruturas é melhor aplicada.

{\Large Desafio rápido!}

Faça um pequeno programa que verifique se um número é par ou ímpar. A variável
\texttt{\$num} deve ser inicializada pelo programador. O programador deve ainda
utilizar a estrutura de condição ternária e em um outro exemplo, utilizar o comando
\comandoifelse. Veja como deve ser a saída do seu programa. 

\figurasimples{cap4-desafio-rapido-1}{Resultado do primeiro desafio rápido.}

\section{Exercícios}
\label{cap5-exercicios}

\section{Desafio!}
\label{cap5-desafio}
O seguinte código é uma página inicial de um sistema de notas simplificado.
O programador deve inserir no próprio código as 3 notas do aluno. Se a média aritmética 
das notas for maior que 6, o aluno está aprovado, caso contrário o aluno se encontra de 
recuperação.

Esse código tem vários problemas e não está executando corretamente, identifique os erros, 
corrija e apresente ao professor.

\lstinputlisting[language=php,style=codigos]{codigos/cap4-desafio.php} 				% Capitulo 5 - Estruturas de controle e repetição 
% ---
% Manipulação de arrays 
% ---
\chapter{Manipulação de array's}
\label{manipulacao-de-arrays}
% ---

Ao final deste capítulo, o aluno terá as seguintes competências:
\begin{enumerate}
    \item Entender a estrutura do tipo \tipoarray;  
    \item Trabalhar com variáveis do tipo \tipoarray; e
    \item Trabalhar com \tipoarrays~de duas dimensões.
\end{enumerate}

O \tipoarray~é um tipo de variável que pode guardar diversos valores (chamados de: elementos 
de um \tipoarray), no \php, podemos definir como um conjunto de valores (elementos) ordenados. 
É possível relacionar cada valor com uma chave, no caso mais simples, a chave indica a 
posição (índice) em que um elemento do \tipoarray~se encontra.

\section{Criando um \comandoarray}
\label{criando-um-array}

Para criar um \tipoarray~pode-se utilizar a função \comandoarray. Veja a sintaxe e um exemplo:

\lstinputlisting[language=php,style=codigos]{codigos/array-1.php}

No exemplo acima, foi apresentado a forma mais simples de inicializar uma variável do 
tipo \tipoarray. A variável \texttt{\$alunos} pode ser percorrida utilizando o comando
\comandofor~por exemplo. Veremos mais a frente.

\section{\textit{Arrays} associativos}
\label{arrays-associativos}

Nos \tipoarrays~associativos determina-se um valor ou nome aos valores do \tipoarray.
Observe a sintaxe:

\lstinputlisting[language=php,style=codigos]{codigos/array-2.php}

No exemplo acima, utilizamos números para representar os índices do \tipoarray. 
Na seção \ref{acessando-um-array} aprenderemos como acessar elementos do \tipoarray.

A seguir, veremos outro exemplo. A figura abaixo representa um \tipoarray~que tem 
como elementos representação de cores. O \tipoarray~possui dez posições, 
cada posição representa uma cor, seu índice (chave) vai de 0 até 9. Veja:

\figurasimples{array-01}{\textit{Array} com 10 posições.}

Em código temos:

\lstinputlisting[language=php,style=codigos]{codigos/array-3.php}

No exemplo abaixo, o \tipoarray~associativo usa strings como índice, onde cada 
\tipostring~representa uma chave. Se quisermos representar as informações de uma
pessoa, podemos usar a variável do tipo \tipoarray.

\lstinputlisting[language=php,style=codigos]{codigos/array-4.php}

Observe que quando usamos \tipoarrays~associativos, a compreensão é mais fácil, 
dando mais legibilidade ao código, ou seja, é mais fácil entender o que cada valor
representa. Veja na representação gráfica abaixo:

\figurasimples{array-02}{\textit{Array} associativo com 4 posições.}

\subsection{Inicializando \tipoarrays}
\label{inicializando-arrays}

Outra forma de inicializar \tipoarrays~é adicionar valores de acordo com o seu 
uso. Veja no exemplo abaixo:

\lstinputlisting[language=php,style=codigos]{codigos/array-5.php}

\subsection{Acessando um \tipoarray}
\label{acessando-um-array}

Para acessarmos o valor do \tipoarray~que tem chaves, usamos \texttt{\$variavel['chave']}. 
Veja abaixo um exemplo de acesso ao valores armazenados em um \tipoarray~dessa natureza.

\lstinputlisting[language=php,style=codigos]{codigos/array-6.php}

Dessa forma podemos acessar um \tipoarray. Basta determinar o nome da variável
(\texttt{\$informacoes}) e a chave (pro nosso exemplo, os possíveis valores são:
\texttt{nome}, \texttt{endereco}, \texttt{bairro} e \texttt{cidade}), onde cada chave tem 
um valor já determinado. Caso a chave não exista, nada será impresso.

Para acessarmos o valor do \tipoarray~que (aparentemente) não tem chaves, usamos
os índices que se iniciam do 0 e vai até a quantidade de elementos do \tipoarray~menos 1.
Veja um exemplo abaixo:

\lstinputlisting[language=php,style=codigos]{codigos/array-10.php}

Por mais que não tenha os valores expressamente indicados na variável \texttt{\$alunos},
o \php~se encarrega de colocar os índices, permitindo ao programador, acessar os valores
de maneira fácil.

\section{Percorrendo um \tipoarray}
\label{percorrendo-um-array}

O processo de percorrer um \tipoarray~pode ser realizado por uma estrutura de 
repetição (\comandofor, \comandowhile, \comandodowhile~ou \comandoforeach). O exemplo a 
seguir apresenta uma abordagem utilizando o laço \comandoforeach.

\lstinputlisting[language=php,style=codigos]{codigos/array-7.php}

\begin{framed}
\textbf{{\Large Importante!}}
Sempre que se depararem com \tipoarrays, onde haja a necessidade de percorrer 
seus valores independentemente da chave, procure utilizar mecanismos de 
programação mais simplificados como o comando \comandoforeach.
\end{framed}

\section{Acessando um \tipoarray}
\label{acessando-um-array}

Quando criamos um \tipoarray~temos que ter em mente que estamos criando uma variável 
que possui vários valores e que os mesmo podem ser acessados a qualquer momento. 
Cada valor está guardado em uma posição que pode ser acessada através de uma chave.
A sintaxe para acesso simplificado de um \tipoarray~é a seguinte:
\lstinputlisting[language=php,style=codigos]{codigos/array-8.php}

Temos que ter cuidado ao passar uma chave para o \tipoarray, pois ela deve conter o 
mesmo nome de qualquer umas das chaves existentes no \tipoarray. Caso a chave não exista, 
o valor não poderá ser resgatado. A sintaxe acima retorna um valor contido na variável 
\texttt{\$variavel\_tipo\_array}. Verifique o exemplo abaixo: 

\lstinputlisting[language=php,style=codigos]{codigos/array-8.php}

No exemplo acima, apenas o valor ``Joaquim Silva'' será apresentado na tela, uma vez que
a chave \texttt{bairro} não está inserida no \tipoarray~\texttt{\$informacoes}.

\section{Alterando um \tipoarray}
\label{alterando-um-array}

Podemos alterar qualquer valor de um \tipoarray. É muito semelhante ao método de acesso. 
A diferença está na chamada do \tipoarray. É nesse momento que atribuímos um novo valor.
Veja o exemplo abaixo.

\lstinputlisting[language=php,style=codigos]{codigos/array-9.php}

Primeiro os valores iniciais da variável \texttt{\$informacoes} são apresentados na tela.
Em seguida, esses valores são modificados, por fim, os novos valores são impressos.
Veja outro exemplo:

\lstinputlisting[language=php,style=codigos]{codigos/array-11.php}

No exemplo acima, modificamos um \tipoarray~simples. Para acessá-los é necessário utilizar
os índices (que cada elemento tem em \tipoarrays~simples). Nesse caso, estamos alterando
o preço do produto na instrução \texttt{\$produto[1] += 3}. Poderíamos ter utilizado a forma
\texttt{\$produto[1] = \$produto[1] + 3} para obtermos o mesmo resultado.

\section{\textit{Arrays} com duas dimensões}
\label{arrays-com-duas-dimensoes}

Os \tipoarrays~com duas dimensões são estruturas de dados que armazenam os valores em 
mais de uma dimensão. Os \tipoarrays~que vimos até agora armazenam valores em uma dimensão, 
por isso para acessar às posições utilizamos somente um índice ou chave. Os \tipoarrays~de 
2 dimensões salvam seus valores na forma de filas e colunas e por isso, é 
necessário dois índices para acessar cada uma de seus valores.

Outra ideia que temos é que \tipoarrays~de duas dimensões são matrizes nos quais 
algumas de suas posições podem conter outros \tipoarrays. Na figura abaixo temos a 
representação de um \tipoarray~com duas dimensões.

\figurasimples{array-03}{\textit{Array} com duas dimensões.}

Um \tipoarray~de duas dimensões pode ser criado pela função \comandoarray~. Veja um exemplo
não funcional.

\lstinputlisting[language=php,style=codigos]{codigos/array-12.php}

Observe agora um exemplo real:

\lstinputlisting[language=php,style=codigos]{codigos/array-13.php}

\subsection{Acessando um \tipoarray~de duas dimensões}
\label{acessando-um-array-de-duas-dimensoes}

Para acessarmos o valor de um \tipoarray~de duas dimensões, basta colocar as 
duas chaves da posição que queremos acessar. É semelhante ao \tipoarray~de uma única dimensão.
Continuando com o exemplo acima, podemos acrescentar ao final do código as seguintes linhas.

\lstinputlisting[language=php,style=codigos]{codigos/array-14.php}

O código acima apresenta no navegador a primeira linha da matriz. Já o código abaixo
apresenta no navegador a última coluna da matriz.

\lstinputlisting[language=php,style=codigos]{codigos/array-15.php}

Perceba que quando queremos imprimir uma linha toda, o primeiro índice não muda, ou seja,
no nosso exemplo ele fica zero. Já quando queremos imprimir a coluna, deixamos o segundo
índice imutável (sem ser modificado). Desenhar a matriz que você quer representar no papel 
ajuda no entendimento da lógica de acesso aos elementos da matriz.

\subsection{Percorrendo um \tipoarray~de duas dimensões}
\label{percorrendo-um-array-de-duas-dimensoes}

Da mesma maneira que percorremos um \tipoarray~de uma dimensão, podemos percorrer um 
\tipoarray~de duas dimensões. Será necessário utilizar dois laços de repetição, seja
um comando \comandofor, \comandowhile~ou \comandoforeach. O \textbf{primeiro laço de repetição
é usado para percorrer as linhas}, o segundo laço utilizado para percorrer os elementos
das colunas. Acompanhe no código abaixo.

\lstinputlisting[language=php,style=codigos]{codigos/array-16.php}

\section{Visualizando \tipoarrays}
\label{visualizando-arrays}

Existe um conjunto de funcionalidades internas do \php~prontas para serem utilizadas nas variáveis
do tipo \tipoarray. Trata-se de funções pré-definidas, você pode encontrá-las facilmente no 
site \url{http://php.net}.

Abordaremos agora funções utilizadas exclusivamente para manipulação de \tipoarray, 
funções de acesso e visualização. Obviamente que não serão abordadas todas as funções, 
pois existem várias, mas mostraremos as mais utilizadas, e outras que são definidas como principais.

\subsection{Função \funcaovardump}
\label{funcao-vardump}

Essa função é muito usada por programadores que pretendem realizar \textit{debug}
(análise mais detalhada para encontrar supostos erros). Observe um exemplo prático:

\lstinputlisting[language=php,style=codigos]{codigos/array-17.php}

\figurasimples{array-vardump}{Saída da função \funcaovardump.}

A função \funcaovardump~é importante pois mostra no resultado o \tipoarray~completo
incluindo os tipos de variáveis que compõem o \tipoarray. No nosso exemplo, o \tipoarray~
tem 3 elementos. Mostra o tipo de cada um desses elementos e seus respectivos valores. 
Os elementos \texttt{id} e \texttt{nome} são do tipo \tipostring. Já o elemento 
\texttt{preco} é do tipo \tipofloat~(número real).

\subsection{Função \funcaoprintr}
\label{funcao-printr}

Imprime o conteúdo de uma variável assim como a função \funcaovardump, porém apresenta 
menos detalhes, alguns programadores a preferem por ser mais legível. Exemplo:

\lstinputlisting[language=php,style=codigos]{codigos/array-18.php}

\figurasimples{array-printr}{Saída da função \funcaoprintr.}

\subsection{Funções \funcaosort~e \funcaorsort}
\label{funcoes-sort-e-rsort}

Muitas vezes é necessário organizar o \tipoarray~de acordo com uma ordem. Por exemplo:
se temos um conjunto de alunos, podemos deixar seus nomes por ordem alfabética de A a Z.
Ou ainda a ordem pode ser de Z a A. Para isso, podemos usar as funções \funcaosort~e
\funcaorsort~respectivamente. Veja o exemplo abaixo.

\lstinputlisting[language=php,style=codigos]{codigos/array-19.php}

\section{Exercícios}
\label{cap6-exercicios}

\begin{description}[labelindent=30pt]
  \item [Q. 01] O que é um \tipoarray, e qual a sua principal finalidade?
  \item [Q. 02] O que é um \tipoarray~associativo?
  \item [Q. 03] O que é um \tipoarray~multidimensional?
  \item Para resolver as questões 4, 5, 6 e 7, utilize como base um \tipoarray~
  de 8 posições. Os valores do \tipoarray~devem ser nomes de pessoas que você conhece. 
  O nome da variável \tipoarray~deve ser \texttt{\$nomes}.
  \item [Q. 04] Apresente na tela o valor do terceiro elemento do \tipoarray.
  \item [Q. 05] O que aparece na tela se usarmos o código: \texttt{echo \$nomes[10]}
  \item [Q. 06] Apresente na tela os nomes da quarta posição até a penúltima posição.
  \item [Q. 07] Utilize o comando \comandoforeach~para apresentar na tela do navegador
  todos os nomes cadastrados.
  \item Para resolver as questões, 08, 09 e 10, utilize como base um \tipoarray~ associativo
  com as chaves \texttt{nome}, \texttt{sobrenome}, \texttt{endereco}, \texttt{bairro},
  \texttt{cidade} e \texttt{criado\_em}. O nome da variável \tipoarray~deve ser \texttt{\$informacoes}.
  \item [Q. 08] Apresente na tela do navegador o valor dos índices \texttt{nome} e \texttt{bairro}.
  \item [Q. 09] Troque o valor inicial do elemento de índice \texttt{criado\_em} para ``19/08/2016''.
  \item [Q. 10] Apresente na tela do navegador todos os valores do \tipoarray~\texttt{\$informacoes}.
  Utilize o comando \comandoforeach.
  \item Crie um vetor de duas dimensões de acordo com o gráfico abaixo e responda as questões 11 e 12.
  \lstinputlisting[language=bash,style=codigos]{codigos/cap6/exemplo-q11}
  \item [Q. 11] Apresente na tela a frase ``oi tudo bem? voce esta em casa?'' 
  \item [Q. 12] Imagine um diálogo possível e reproduza no navegador utilizando os índices dos
  elementos do \tipoarray.
\end{description}

\section{Desafio!}
\label{cap6-desafio} 				% Capitulo 6 - Manipulação de arrays 
% ---
% Interações PHP com HTML 
% ---
\chapter{Interações PHP com HTML}
\label{interacoes-php-com-html}
% ---

Ao final deste capítulo, o aluno terá as seguintes competências:
\begin{enumerate}
    \item Entender a estrutura dos métodos \metodoGET~ e \metodoPOST; e
    \item Enviar dados para o servidor através do navegador.
\end{enumerate}

Neste capítulo abordaremos algumas formas de interações utilizando a linguagem de 
programação \php~ e a linguagem de marcação \html. Além disso, mostraremos alguns dos
componentes mais utilizados para a construção de um sistema ou uma página web, e de 
que forma o \php~ pode receber, processar, e enviar essa informação.

Utilizamos linguagens para exibir os dados da aplicação, seja ela para simples 
conferência, em relatório ou ainda possibilitando a adição e exclusão de registros. 
Criaremos a princípio formulários e listagem.

\section{Formulários}
\label{formularios}

Formulário pode ser definido como um conjunto de campos disponíveis de forma agrupada 
para serem preenchidos com informações. Na internet, um formulário é composto por vários 
componentes, além de possuir botões de ação, no qual se define o programa que processará 
os dados.

\subsection{Criando um formulário}
\label{criando-um-formulario}

Para criarmos um formulário, utilizamos a \tag~ \tagform. Dentro dessa \tag~ podemos colocar 
diversos elementos, onde, cada um deles representa uma propriedade em particular. 
A seguir explicaremos os principais componentes de um formulário.

Todo formulário deve conter no mínimo as seguintes características: ter um nome, um método
e uma ação. O nome deve ser único e só pode haver um formulário com esse nome na página
\php. Veremos posteriormente nesse capítulo que podemos usar os métodos \metodoGET~ ou
\metodoPOST. Já a ação é o caminho do arquivo que vai tratar os dados do formulário.
Veja a sintaxe.

\lstinputlisting[language=html,style=codigos]{codigos/cap7/sintaxe.html}

Lembrando que em cada aula iremos criar o diretório da respectiva aula no caminho
\directory{/var/www/}. Salve o arquivo abaixo com o nome \texttt{index.html} no
diretório da aula.

\lstinputlisting[language=html,style=codigos]{codigos/cap7/index.html}

Ao abrir o código acima no navegador, encontramos apenas uma página em branco com o 
título ``Formulário''. Vamos colocar um campo de texto, permitindo que o usuário digite 
um texto simples. Para isso, é necessário utilizar a \tag~ \taginput~ do \html. Conforme 
exemplo abaixo. Para simplificar, vamos omitir algumas linhas do código.

\lstinputlisting[language=html,style=codigos]{codigos/cap7/index-2.html}

A \tag~ \taginput~ pode ser composta por vários atributos. A partir deles, vamos definir o 
comportamento da \tag. No nosso exemplo, definimos os atributos nome e tipo. O atributo
\texttt{name} é utilizado para identificação do elemento na página (ele deve ser único). 
Já o atributo \texttt{type} (tipo em inglês) indica qual campo de formulário a \tag~ \taginput 
irá representar no navegador. Escreva o código acima e visualize no seu navegador.

\section{Métodos \metodoGET~ e \metodoPOST}
Agora que relembramos a estrutura de um formulário \html, vamos aprender sobre os métodos
\metodoGET~ e \metodoPOST. A função básica desses métodos é a de enviar os dados para 
o servidor (lembre-se da arquitetura cliente-servidor) para que ele tenha acesso aos 
dados digitados no formulário. O protocolo HTTP disponibiliza vários métodos para 
trabalhar com requisições. Os dois métodos básicos de envio de dados ao servidor serão 
vistos nessa apostila.

Portanto, imagine a seguinte situação: Joaquim, aluno do e-jovem quer pesquisar no Google 
sobre \php. Ao digitar o termo de pesquisa na caixa de busca do Google, o navegador realiza 
uma requisição HTTP de método \metodoGET. O Google então apresenta uma lista de possíveis 
sites que o Joaquim pode entrar, ao escolher clicar no primeiro link, uma nova requisição 
HTTP com o método \metodoGET~ foi realizada.

Uma outra situação possível é o Joaquim acessar sua página do Facebook, ele precisa digitar 
o nome de usuário e a senha, ao acionar o botão “Entrar”, o navegador realiza uma requisição 
HTTP dessa vez com o método o \metodoPOST.

A figura abaixo ilustra a comunicação que ocorre entre o formulário (cliente) e o arquivo 
\php~ (servidor), seja utilizando o método \metodoGET~ ou método \metodoPOST. A diferença básica 
entre esses métodos será visto nas próximas seções. 

ILUSTRAÇÃO DA MILENA

\subsection{Método \metodoGET}

O método \metodoGET~ utiliza a própria URI (normalmente chamada de URL) para enviar dados ao 
servidor. Quando enviamos um formulário pelo método \metodoGET, o navegador pega as informações 
do formulário e coloca junto com a URI de onde o formulário vai ser enviado e envia, 
separando o endereço da URI dos dados do formulário por um \texttt{``?''} (ponto de interrogação) 
e \texttt{``\&''}.

Esse processo gera uma variável do tipo \tipoarray~ chamada de \variavelget. O conteúdo dela é
formado pelos dados do formulário enviado.

Quando o usuário faz uma busca no Google, ele faz uma requisição utilizando o método \metodoGET, 
você pode verificar na barra de endereço do seu navegador que o endereço ficou com um ponto de 
interrogação no meio, e depois do ponto de interrogação você pode ler, dentre outros caracteres, 
o que você pesquisou no Google.

Abaixo temos um exemplo de uma URL do Joaquim pesquisando sobre \php~ no Google. Observe:

\figurasimples{metodo-get-ok}{Exemplo de URL utilizando o método GET.}

Podemos notar a passagem de um parâmetro, chamado de \texttt{?gws\_rd=ssl\#q=}. Os últimos termos
são os digitados pelo usuário. No nosso exemplo, \texttt{php} e \texttt{documentacao}.

\subsection{Nosso primeiro formulário com método \metodoGET}
\label{nosso-primeiro-formulario-com-metodo-get}

Para nosso primeiro formulário, vamos simular o envio de um \textit{email}. Imagine que você
está cadastrando seu \textit{email} para participar de uma \textit{newsletter} (um boletim informativo).
Para isso, o formulário deve ter o campo \textit{email} e um botão de enviar.
Utilize seus conhecimentos de \html~ para representar a tela abaixo.

\figurasimples{metodo-get-exemplo-1}{Exemplo de um formulário simples.}

Ao clicarmos em avançar, será possível, visualizar no navegador as informações inseridas no campo
de texto. No nosso exemplo, joaquim@ejovem.com.br. Parte do código que foi utilizado para
construir essa página é apresentado abaixo.

\lstinputlisting[language=html,style=codigos]{codigos/cap7/index-4.html}

Perceba os atributos \texttt{name}, \texttt{method} e \texttt{action}. A partir desses
atributos será possível trabalhar com os dados inseridos no formulário.

\subsection{Recebendo dados via método \metodoGET}
\label{recebendo-dados-via-metodo-get}

Quando o botão ``Avançar'' for pressionado, uma requisição será feita ao site 
\url{http://localhost} (nosso servidor) enviando todos os dados preenchidos no
formulário. O arquivo \phpextensao~ responsável por tratar esses dados é o arquivo
mencionado no atributo \texttt{action}. No nosso primeiro exemplo, o arquivo de nome
\texttt{cadastro-newsletter.php}.

Caso você ainda não o tenha criado. Crie um novo arquivo dentro de \directory{/var/www/aula07/}
chamado \texttt{cadastro-newsletter.php}. Inicialmente, coloque o seguinte conteúdo dentro do arquivo.

\lstinputlisting[language=php,style=codigos]{codigos/cap7/cadastro-newsletter.php}

O código acima exibe no navegador a variável \variavelget~ que é um \tipoarray. Perceba que a
única chave do \tipoarray~ é igual ao nome do campo de texto no formulário na página \html.
Ou seja, \textbf{o \tipoarray~ \variavelget~ é um \tipoarray~ associativo}, onde as chaves são
os atributos \texttt{name} dos campos do formulário e os elementos são os valores inseridos
no formulário.

\figurasimples{metodo-get-exemplo-2}{Conteúdo enviado através do método \metodoGET.}

Caso seja necessário pegar apenas a informação de email, devemos lembrar como acessamos 
elementos em um \tipoarray associativo. Para isso, precisaremos da chave do elemento. Veja
no código abaixo.

\lstinputlisting[language=php,style=codigos]{codigos/cap7/cadastro-newsletter-2.php}

\begin{framed}
{\Large Desafio rápido!}

Altere o programa acima para receber, além do email do usuário, um nome e
data de nascimento. Apresente no arquivo \texttt{cadastro-newsletter.php} 
o \tipoarray~ \variavelget~ através da função \funcaoprintr~ e as variáveis
separadamente.
\end{framed}

\section{Exercícios}
\label{cap7-exercicios}

\section{Desafio!}
\label{cap7-desafio} 				% Capitulo 7 - Interações PHP com HTML

% ----------------------------------------------------------
% Finaliza a parte no bookmark do PDF
% para que se inicie o bookmark na raiz
% e adiciona espaço de parte no Sumário
% ----------------------------------------------------------
\phantompart

% ---
% Conclusão
% ---
\chapter{Conclusão}
% ---

\lipsum[31-33] 				% Capítulo de introdução

% ----------------------------------------------------------
% ELEMENTOS PÓS-TEXTUAIS
% ----------------------------------------------------------
\postextual
% ----------------------------------------------------------

% ----------------------------------------------------------
% Referências bibliográficas
% ----------------------------------------------------------
\bibliography{referencias}

% ----------------------------------------------------------
% Glossário
% ----------------------------------------------------------
%
% Consulte o manual da classe abntex2 para orientações sobre o glossário.
%
%\glossary

% ----------------------------------------------------------
% Apêndices
% ----------------------------------------------------------

% ---
% Inicia os apêndices
% ---
\begin{apendicesenv}

% Imprime uma página indicando o início dos apêndices
\partapendices
% Insere os apêndices
% ----------------------------------------------------------
\chapter{Instalação de ambiente de desenvolvimento no Windows}
\label{ap:instalacao-env-windows}
% ----------------------------------------------------------
\lipsum[55-57]
\chapter{Quisque libero justo}
% ----------------------------------------------------------

\lipsum[50]
%\include{apendices/apendice-c}
%\include{apendices/apendice-d}




\end{apendicesenv}
% ---


% ----------------------------------------------------------
% Anexos
% ----------------------------------------------------------

% ---
% Inicia os anexos
% ---
\begin{anexosenv}

% Imprime uma página indicando o início dos anexos
\partanexos
% Insere os anexos
\chapter{Morbi ultrices rutrum lorem.}
% ---
\lipsum[30]

\chapter{Cras non urna sed feugiat cum sociis natoque penatibus et magnis dis
parturient montes nascetur ridiculus mus}
% ---

\lipsum[31]
%\include{anexos/anexo-c}
%\include{anexos/anexo-d}

\end{anexosenv}

%---------------------------------------------------------------------
% INDICE REMISSIVO
%---------------------------------------------------------------------
\phantompart
\printindex
%---------------------------------------------------------------------

\end{document}
