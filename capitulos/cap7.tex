% ---
% Interações PHP com HTML 
% ---
\chapter{Interações PHP com HTML}
\label{interacoes-php-com-html}
% ---

Ao final deste capítulo, o aluno terá as seguintes competências:
\begin{enumerate}
    \item Entender a estrutura dos métodos \metodoGET~ e \metodoPOST; e
    \item Enviar dados para o servidor através do navegador.
\end{enumerate}

Neste capítulo abordaremos algumas formas de interações utilizando a linguagem de 
programação \php~ e a linguagem de marcação \html. Além disso, mostraremos alguns dos
componentes mais utilizados para a construção de um sistema ou uma página web, e de 
que forma o \php~ pode receber, processar, e enviar essa informação.

Utilizamos linguagens para exibir os dados da aplicação, seja ela para simples 
conferência, em relatório ou ainda possibilitando a adição e exclusão de registros. 
Criaremos a princípio formulários e listagem.

\section{Formulários}
\label{formularios}

Formulário pode ser definido como um conjunto de campos disponíveis de forma agrupada 
para serem preenchidos com informações. Na internet, um formulário é composto por vários 
componentes, além de possuir botões de ação, no qual se define o programa que processará 
os dados.

\subsection{Criando um formulário}
\label{criando-um-formulario}

Para criarmos um formulário, utilizamos a \tag~ \tagform. Dentro dessa \tag~ podemos colocar 
diversos elementos, onde, cada um deles representa uma propriedade em particular. 
A seguir explicaremos os principais componentes de um formulário.

Todo formulário deve conter no mínimo as seguintes características: ter um nome, um método
e uma ação. O nome deve ser único e só pode haver um formulário com esse nome na página
\php. Veremos posteriormente nesse capítulo que podemos usar os métodos \metodoGET~ ou
\metodoPOST. Já a ação é o caminho do arquivo que vai tratar os dados do formulário.
Veja a sintaxe.

\lstinputlisting[language=html,style=codigos]{codigos/cap7/sintaxe.html}

Lembrando que em cada aula iremos criar o diretório da respectiva aula no caminho
\directory{/var/www/}. Salve o arquivo abaixo com o nome \texttt{index.html} no
diretório da aula.

\lstinputlisting[language=html,style=codigos]{codigos/cap7/index.html}

Ao abrir o código acima no navegador, encontramos apenas uma página em branco com o 
título ``Formulário''. Vamos colocar um campo de texto, permitindo que o usuário digite 
um texto simples. Para isso, é necessário utilizar a \tag~ \taginput~ do \html. Conforme 
exemplo abaixo. Para simplificar, vamos omitir algumas linhas do código.

\lstinputlisting[language=html,style=codigos]{codigos/cap7/index-2.html}

A \tag~ \taginput~ pode ser composta por vários atributos. A partir deles, vamos definir o 
comportamento da \tag. No nosso exemplo, definimos os atributos nome e tipo. O atributo
\texttt{name} é utilizado para identificação do elemento na página (ele deve ser único). 
Já o atributo \texttt{type} (tipo em inglês) indica qual campo de formulário a \tag \taginput 
irá representar no navegador. Veja na imagem abaixo.

FIGURA

\section{Métodos \metodoGET~ e \metodoPOST}
Agora que relembramos a estrutura de um formulário \html, vamos aprender sobre os métodos
\metodoGET~ e \metodoPOST. A função básica desses métodos é a de enviar os dados para 
o servidor (lembre-se da arquitetura cliente-servidor) para que ele tenha acesso aos 
dados digitados no formulário. O protocolo HTTP disponibiliza vários métodos para 
trabalhar com requisições. Os dois métodos básicos de envio de dados ao servidor serão 
vistos nessa apostila.

Portanto, imagine a seguinte situação: Joaquim, aluno do e-jovem quer pesquisar no Google 
sobre \php. Ao digitar o termo de pesquisa na caixa de busca do Google, o navegador realiza 
uma requisição HTTP de método \metodoGET. O Google então apresenta uma lista de possíveis 
sites que o Joaquim pode entrar, ao escolher clicar no primeiro link, uma nova requisição 
HTTP com o método \metodoGET~ foi realizada.

Uma outra situação possível é o Joaquim acessar sua página do Facebook, ele precisa digitar 
o nome de usuário e a senha, ao acionar o botão “Entrar”, o navegador realiza uma requisição 
HTTP dessa vez com o método o \metodoPOST.

A figura abaixo ilustra a comunicação que ocorre entre o formulário (cliente) e o arquivo 
\php~ (servidor), seja utilizando o método \metodoGET~ ou método \metodoPOST. A diferença básica 
entre esses métodos será visto nas próximas seções. 

\subsection{Método \metodoGET}

O método \metodoGET~ utiliza a própria URI (normalmente chamada de URL) para enviar dados ao 
servidor. Quando enviamos um formulário pelo método \metodoGET, o navegador pega as informações 
do formulário e coloca junto com a URI de onde o formulário vai ser enviado e envia, 
separando o endereço da URI dos dados do formulário por um \texttt{``?''} (ponto de interrogação) 
e \texttt{``\&''}.

Quando o usuário faz uma busca algo no Google, ele faz uma requisição utilizando o método \metodoGET, 
você pode verificar na barra de endereço do seu navegador que o endereço ficou com um ponto de 
interrogação no meio, e depois do ponto de interrogação você pode ler, dentre outros caracteres, 
o que você pesquisou no Google.

Abaixo temos um exemplo de uma URL do Joaquim pesquisando sobre \php~ no Google. Observe:

PRINTSCREEN DO NAVEGADOR COM BUSCA

Podemos notar a passagem de um parâmetro, chamado de \texttt{?gws_rd=ssl#q=}. Os últimos termos
são os digitados pelo usuário. No nosso exemplo, \texttt{php} e \texttt{documentacao}.
?hl = pt-br, logo após &shva=1, ou seja, temos então a criação da “variável” hl que recebe o valor pt-br, e também a “variável” shva que recebe como valor 1. Veja exemplos de códigos com links enviando valores por métodos GET.


\subsection{Nosso primeiro formulário}
\label{nosso-primeiro-formulario}

Para nosso primeiro formulário, vamos simular o envio de um \textit{email}. Imagine que você
está cadastrando seu \textit{email} para participar de uma \textit{newsletter} (um boletim informativo).
Para isso, o formulário deve ter o campo \textit{email} e um botão de enviar. Como na figura abaixo.

FIGURA DO INDEX-3.HTML NO NAVEGADOR




\section{Exercícios}
\label{cap7-exercicios}

\section{Desafio!}
\label{cap7-desafio}