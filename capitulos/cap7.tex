% ---
% Interações PHP com HTML 
% ---
\chapter{Interações PHP com HTML}
\label{interacoes-php-com-html}
% ---

Ao final deste capítulo, o aluno terá as seguintes competências:
\begin{enumerate}
    \item Entender a estrutura dos métodos \metodoGET~e \metodoPOST; e
    \item Enviar dados para o servidor através do navegador.
\end{enumerate}

Neste capítulo abordaremos algumas formas de interações utilizando a linguagem de 
programação \php~e a linguagem de marcação \html. Além disso, mostraremos alguns dos
componentes mais utilizados para a construção de um sistema ou uma página web, e de 
que forma o \php~pode receber, processar, e enviar essa informação.

Utilizamos linguagens para exibir os dados da aplicação, seja ela para simples 
conferência, em relatório ou ainda possibilitando a adição e exclusão de registros. 
Criaremos a princípio formulários e listagem.

\section{Formulários}
\label{formularios}

Formulário pode ser definido como um conjunto de campos disponíveis de forma agrupada 
para serem preenchidos com informações. Na internet, um formulário é composto por vários 
componentes, além de possuir botões de ação, no qual se define o programa que processará 
os dados.

\subsection{Criando um formulário}
\label{criando-um-formulario}

Para criarmos um formulário, utilizamos a \tag~\tagform. Dentro dessa \tag~podemos colocar 
diversos elementos, onde, cada um deles representa uma propriedade em particular. 
A seguir explicaremos os principais componentes de um formulário.

Todo formulário deve conter no mínimo as seguintes características: ter um nome, um método
e uma ação. O nome deve ser único e só pode haver um formulário com esse nome na página
\php. Veremos posteriormente nesse capítulo que podemos usar os métodos \metodoGET~ou
\metodoPOST. Já a ação é o caminho do arquivo que vai tratar os dados do formulário.
Veja a sintaxe.

\lstinputlisting[language=html,style=codigos]{codigos/cap7/sintaxe.html}

Lembrando que em cada aula iremos criar o diretório da respectiva aula no caminho
\directory{/var/www/}. Salve o arquivo abaixo com o nome \texttt{index.html} no
diretório da aula.

\lstinputlisting[language=html,style=codigos]{codigos/cap7/index.html}

Ao abrir o código acima no navegador, encontramos apenas uma página em branco com o 
título ``Formulário''. Vamos colocar um campo de texto, permitindo que o usuário digite 
um texto simples. Para isso, é necessário utilizar a \tag~\taginput~do \html. Conforme 
exemplo abaixo. Para simplificar, vamos omitir algumas linhas do código.

\lstinputlisting[language=html,style=codigos]{codigos/cap7/index-2.html}

A \tag~\taginput~pode ser composta por vários atributos. A partir deles, vamos definir o 
comportamento da \tag. No nosso exemplo, definimos os atributos nome e tipo. O atributo
\texttt{name} é utilizado para identificação do elemento na página (ele deve ser único). 
Já o atributo \texttt{type} (tipo em inglês) indica qual campo de formulário a \tag~\taginput~
irá representar no navegador. Escreva o código acima e visualize no seu navegador.

\section{Métodos \metodoGET~e \metodoPOST}
Agora que relembramos a estrutura de um formulário \html, vamos aprender sobre os métodos
\metodoGET~e \metodoPOST. A função básica desses métodos é a de enviar os dados para 
o servidor (lembre-se da arquitetura cliente-servidor) para que ele tenha acesso aos 
dados digitados no formulário. O protocolo HTTP disponibiliza vários métodos para 
trabalhar com requisições. Os dois métodos básicos de envio de dados ao servidor serão 
vistos nessa apostila.

Portanto, imagine a seguinte situação: Joaquim, aluno do e-jovem quer pesquisar no Google 
sobre \php. Ao digitar o termo de pesquisa na caixa de busca do Google, o navegador realiza 
uma requisição HTTP de método \metodoGET. O Google então apresenta uma lista de possíveis 
sites que o Joaquim pode entrar, ao escolher clicar no primeiro link, uma nova requisição 
HTTP com o método \metodoGET~foi realizada.

Uma outra situação possível é o Joaquim acessar sua página do Facebook, ele precisa digitar 
o nome de usuário e a senha, ao acionar o botão “Entrar”, o navegador realiza uma requisição 
HTTP dessa vez com o método o \metodoPOST.

A figura abaixo ilustra a comunicação que ocorre entre o formulário (cliente) e o arquivo 
\php~(servidor), seja utilizando o método \metodoGET~ou método \metodoPOST. A diferença básica 
entre esses métodos será visto nas próximas seções. 

ILUSTRAÇÃO DA MILENA

\subsection{Método \metodoGET}
\label{metodo-get}

O método \metodoGET~utiliza a própria URI (normalmente chamada de URL) para enviar dados ao 
servidor. Quando enviamos um formulário pelo método \metodoGET, o navegador pega as informações 
do formulário e coloca junto com a URI de onde o formulário vai ser enviado e envia, 
separando o endereço da URI dos dados do formulário por um \texttt{``?''} (ponto de interrogação) 
e \texttt{``\&''}.

Esse processo gera uma variável do tipo \tipoarray~chamada de \variavelget. O conteúdo dela é
formado pelos dados do formulário enviado.

Quando o usuário faz uma busca no Google, ele faz uma requisição utilizando o método \metodoGET, 
você pode verificar na barra de endereço do seu navegador que o endereço ficou com um ponto de 
interrogação no meio, e depois do ponto de interrogação você pode ler, dentre outros caracteres, 
o que você pesquisou no Google.

Abaixo temos um exemplo de uma URL do Joaquim pesquisando sobre \php~no Google. Observe:

\figurasimples{metodo-get-ok}{Exemplo de URL utilizando o método GET.}

Podemos notar a passagem de um parâmetro, chamado de \texttt{?gws\_rd=ssl\#q=}. Os últimos termos
são os digitados pelo usuário. No nosso exemplo, \texttt{php} e \texttt{documentacao}.

\subsection{Nosso primeiro formulário com método \metodoGET}
\label{nosso-primeiro-formulario-com-metodo-get}

Para nosso primeiro formulário com método \metodoGET, vamos simular o envio de um 
\textit{email}. Imagine que você está cadastrando seu \textit{email} para participar 
de uma \textit{newsletter} (um boletim informativo). Para isso, o formulário deve 
ter o campo \textit{email} e um botão de enviar. Utilize seus conhecimentos de \html~
para representar a tela abaixo.

\figurasimples{metodo-get-exemplo-1}{Exemplo de um formulário simples.}

Ao clicarmos em avançar, será possível, visualizar no navegador as informações inseridas no campo
de texto. No nosso exemplo, joaquim@ejovem.com.br. Parte do código que foi utilizado para
construir essa página é apresentado abaixo.

\lstinputlisting[language=html,style=codigos]{codigos/cap7/index-4.html}

Perceba os atributos \texttt{name}, \texttt{method} e \texttt{action}. A partir desses
atributos será possível trabalhar com os dados inseridos no formulário.

\subsection{Recebendo dados via método \metodoGET}
\label{recebendo-dados-via-metodo-get}

Quando o botão ``Avançar'' for pressionado, uma requisição será feita ao site 
\url{http://localhost} (nosso servidor) enviando todos os dados preenchidos no
formulário. O arquivo \phpextensao~responsável por tratar esses dados é o arquivo
mencionado no atributo \texttt{action}. No nosso primeiro exemplo, o arquivo de nome
\texttt{cadastro-newsletter.php}.

Caso você ainda não o tenha criado. Crie um novo arquivo dentro de \directory{/var/www/aula06/}
chamado \texttt{cadastro-newsletter.php}. Inicialmente, coloque o seguinte conteúdo dentro do arquivo.

\lstinputlisting[language=php,style=codigos]{codigos/cap7/cadastro-newsletter.php}

O código acima exibe no navegador a variável \variavelget~que é um \tipoarray. Perceba que a
única chave do \tipoarray~é igual ao nome do campo de texto no formulário na página \html.
Ou seja, \textbf{o \tipoarray~\variavelget~é um \tipoarray~associativo}, onde as chaves são
os atributos \texttt{name} dos campos do formulário e os elementos são os valores inseridos
no formulário.

\figurasimples{metodo-get-exemplo-2}{Conteúdo enviado através do método \metodoGET.}

Caso seja necessário pegar apenas a informação de email, devemos lembrar como acessamos 
elementos em um \tipoarray~associativo. Para isso, precisaremos da chave do elemento. Veja
no código abaixo.

\lstinputlisting[language=php,style=codigos]{codigos/cap7/cadastro-newsletter-2.php}

\begin{framed}
{\Large Desafio rápido!}

Altere o arquivo \texttt{cadastro-newsletter.php} para receber, além do email do 
usuário, um nome e data de nascimento. Apresente no navegador  
o \tipoarray~\variavelget~através da função \funcaoprintr~e as variáveis
separadamente (lembre-se do procedimento de acessar uma variável do tipo \tipoarray~
associativo).
\end{framed}

\section{Método \metodoPOST}
\label{metodo-post}

O método \metodoPOST~é bastante semelhante ao método \metodoGET. Ele também é utilizado
para transferir os dados do formulário na página \html~para o servidor \php. A principal
diferença está em enviar os dados encapsulados no corpo da mensagem HTTP, ou seja,
ao invés das informações inseridas no formulário trafegarem junto à URL. Elas vão
``escondidas''. 

Sua utilização é mais frequente quando trabalhamos com informações sigilosas, como é o caso
de autenticação por meio de usuário e senha, por exemplo: Joaquim deve primeiro se autenticar,
inserindo seu \textit{email} e senha, para só depois, ter acesso à caixa de entrada do \textit{email}.

O código do formulário é modificado apenas no atributo \texttt{method}. O valor se torna
\metodoPOST. Veja no exemplo abaixo.

\lstinputlisting[language=html,style=codigos]{codigos/cap7/index-6.html}

\subsection{Nosso primeiro formulário com método \metodoPOST}
\label{nosso-primeiro-formulario-com-metodo-post}

Para nosso primeiro formulário com método \metodoPOST, vamos simular a autenticação em um 
sistema \web. Imagine que você está cadastrado nesse sistema e deseja se autenticar. 
Para isso, o formulário deve ser preenchido com um usuário e a senha do mesmo.
Utilize seus conhecimentos de \html~para representar a tela abaixo.

\figurasimples{metodo-post-exemplo-1}{Exemplo de um formulário de autenticação simples.}

O formulário acima deve ter as seguintes características na \tag~\tagform:

\lstinputlisting[language=html,style=codigos]{codigos/cap7/index-7.html}

\subsection{Recebendo dados via método \metodoPOST}
\label{recebendo-dados-via-metodo-post}

Ao clicarmos em avançar, os dados inseridos no formulário serão tratados no arquivo
\texttt{acesso.php}. Portanto, vamos criar o arquivo no diretório \directory{/var/www/aula06}
com o seguinte conteúdo:

\lstinputlisting[language=php,style=codigos]{codigos/cap7/acesso-2.php}

O código acima exibe a variável do tipo \tipoarray~com a função \funcaoprintr.
Perceba que não estamos mais trabalhando com a variável \variavelget~e sim com a
variável \variavelpost, pois o método utilizado no formulário mudou.

Outro detalhe importante a ser percebido é que as informações inseridas no
formulário não estão mais visíveis na URL. Para visualizarmos as informações
inseridas, devemos programar o arquivo \texttt{acesso.php}.

\begin{framed}
{\Large Desafio rápido!}

Altere o arquivo \texttt{acesso.php} para exibir no navegador o \textit{email} do usuário
e sua senha. Utilize o \tipoarray~\variavelpost. Lembre-se do procedimento de acessar uma 
variável do tipo \tipoarray~associativo.
\end{framed}

\section{Exercícios}
\label{cap7-exercicios}
Lembre-se que todo arquivo desenvolvido para resolver as questões abaixo devem ser
guardados no diretório \directory{/var/www/aula06}.

\begin{description}[labelindent=30pt]
  \item [Q. 01] O que é o método \metodoGET~e qual a sua principal finalidade?
  \item [Q. 02] O que é o método \metodoPOST~e qual sua principal finalidade?
  Para resolver as questões 03, 04 e 05, leve em consideração o seguinte código:
  \lstinputlisting[language=php,style=codigos]{codigos/cap7/exemplo-q03.php}
  \item [Q. 03] Qual método está sendo utilizado no formulário acima?
  \item [Q. 04] Que variável podemos usar no arquivo \texttt{mala\_direta.php}
  para obtermos o conteúdo do formulário?
  \item [Q. 05] Desenvolva o código do arquivo \texttt{mala\_direta.php} e
  apresente na tela o dado enviado pelo formulário.
  \item [Q. 06] Escreva um arquivo \htmlextensao~contendo um formulário com os
  campos \texttt{email} e \texttt{senha}. Os dados devem ser tratados no arquivo
  \texttt{acesso.php}. Você deve utilizar o método \metodoPOST.
\end{description}

\section{Desafio!}
\label{cap7-desafio}

Crie um formulário que receba do usuário um nome de um aluno e 2 notas. Em seguida, 
o programa deve calcular a média, que é obtida multiplicando a segunda nota por 2, 
somando com a primeira nota e depois dividindo tudo por 3. O programa deve apresentar
na tela o resultado se o aluno foi aprovado ou não, com a média calculada. 