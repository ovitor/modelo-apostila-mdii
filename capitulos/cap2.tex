% ---
% Ambiente de desenvolvimento com Sublime Text
% ---
\chapter{Ambiente de desenvolvimento com Sublime Text}
% ---

Ao final deste capítulo, o aluno terá as seguintes competências:
\begin{enumerate}
	\item Instalar o \sublime \sublimeversao~ no sistema operacional;
	\item Instalar \plugins~ do \sublime; e
	\item Criar uma estrutura de aplicação \web com \php. 
\end{enumerate}

\section{O \sublime}
\label{o-sublime}

O \sublime~ é um editor de textos melhorado. Com esse tipo de \software~
é possível desenvolver os mais diversos programas, incluíndo os \sites. 
O \sublime~ hoje dispõe de duas versões que são amplamente utilizadas. 
A versão 2 - estável porém mais antiga e a versão 3 - caracterizada com 
versão \textit{beta} porém mais nova. Apesar de no curso sempre trabalharmos com 
as versões estáveis dos programas, com esse editor usaremos a versão \sublimeversao.

O editor de textos não vem por padrão nas distribuições Linux. Por isso, é necessário instalá-lo. 
Vamos abrir o navegador Firefox e digitar a URL \url{https://www.sublimetext.com/3}.
Em seguida, clique no \textit{link} ``Download''. Se você estiver usando o Linux do Projeto e-Jovem
escolha a opção ``Ubuntu 32 bit''. Essa ação vai baixar o arquivo \sublimefilename.
o \texttt{3XXX} indica a versão (3 no caso) e pequenas mudanças na versão principal respectivamente.

Lembre-se! Saíba onde você salvou o arquivo baixado! Será importante saber a localização dele
no momento da instalação!

\section{Instalando o \sublime}
\label{instalando-o-sublime}

Agora que baixamos o arquivo \texttt{.deb}, vamos instalá-lo. Abra o \terminal~ de acordo
com o apresentado na seção \ref{instalacao-do-php}. Utilize os comandos \comandocdcompleto, 
\comandolscompleto~ e \comandopwdcompleto~ para ir ao diretório que o arquivo \sublimefilename~ está salvo.
No caso da apostila, o arquivo foi salvo dentro do diretório \directory{/home/ejovem/Downloads}.

Caso você não saíba onde o arquivo se encontra, peça ajuda ao seu instrutor.

\begin{multicols}{2}

  Comandos executados no terminal 
  \begin{lstlisting}[language=bash,style=codigos]
    $ cd ~/Downloads
    $ ls
  \end{lstlisting}

  \columnbreak

  Resposta do comando \comandolscompleto
  \begin{lstlisting}[language=bash, style=codigos]
    ...  
    sublime-text_build-3114_i368.deb
    ...  
  \end{lstlisting}

\end{multicols}

Agora que a gente sabe onde o arquivo se encontra, podemos instalar com o comando
\dpkg. Execute o seguinte código no diretório em que o \sublimefilename~ se encontra

  \begin{lstlisting}[language=bash,style=codigos]
    $ sudo dpkg -i sublime-text_build-3114_i368.deb
  \end{lstlisting}

Se a instalação do \sublime~ ocorreu com sucesso. Obtemos a seguinte tela na imagem \ref{fig:sublime-instalacao-ok}.
Para abrir o programa, digite a combinação de teclas \altfdois e digite o comando \sublimebin.
A tela inicial do programa é a exibida na figura \ref{fig:sublime-ok}.

\figuradupla{sublime-instalacao-ok}{Instalação do aplicativo \sublime~ realizada com sucesso}
{sublime-ok}{Tela inicial do \sublime}

\section{Primeiros passos}
\label{primeiros-passos}

Vamos começar nossos testes com o \sublime. Nosso primeiro teste é abrir o arquivo 
editado anteriormente pelo programa \gedit. Na tela inicial do \sublime, execute a seguinte
sequência de menus: \menu[,]{File, Open File...}. O arquivo deverá estar no diretório
\directory{/var/www/aula01}. E tem o nome de \arquivo{index.php}.

Vamos editar o arquivo. Troque a primeira linha do arquivo para que fique parecido com o
apresentado abaixo.

\lstinputlisting[language=php,style=codigos]{codigos/echo-sublime.php}

Lembre-se! Você pode colocar \tags~ \html~ dentro das instruções \php!

\section{Desafio!}
\label{desafio}
O desafio deste capítulo é você encontrar na tela do navegador a versão do \php~
que vamos trabalhar, a versão do \apache~ além de nos mostrar qual o diretório
padrão em que os arquivos \phpextensao~ devem ser salvos.