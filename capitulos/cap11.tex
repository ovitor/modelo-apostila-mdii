% ---
% Tratamentos de Erro
% ---
\chapter{Tratamentos de erros}
\label{tratamentos-de-erros}
% ---

Ao final deste capítulo, o aluno terá as seguintes competências:
\begin{enumerate}
    \item Entender os tipos de erro e suas possíveis causas;
    \item Tratar os erros;
\end{enumerate}

\section{Tipos de Erro}
\label{tipos-de-erro}

Durante o processo de programação é comum surgirem erros que interrompem, causam 
um mal funcionamento ao programa ou apresentam um resultado inesperado. Normalmente 
esses erros são provocados pelo próprio programador, e podem ser:

\begin{itemize}
  \item \textbf{Erro de sintaxe}: quando o programador escreve uma instrução da forma 
  errada, por exemplo, esquece de colocar um ``;'' (ponto-e-vírgula) no final da linha;
  \item \textbf{Erro de semântica}: quando o programador escreveu um código correto mas 
  não previu alguma outra situação ou dependência para que a instrução programada funcione. 
  Exemplo: ao programar a abertura de um arquivo, não verificou antes se o arquivo realmente existe;
  \item \textbf{Erros de lógica}: quando o programador não analisa corretamente o fluxo 
  lógico do que se espera que o programa execute.
\end{itemize}

Existem também os erros provocados por fatores externos, alheios ao programador, em relação a 
própria linguagem, ao computador, servidores, sistemas operacionais etc. As linguagens de 
programação possuem mecanismos para interromper o programa em caso de erro, e o \php~
oferece um conjunto de tipos de erro para alertar ao programador e, assim, auxiliá-lo 
na identificação da causa.

Geralmente o \php~apresenta os seguintes tipos de mensagens de erro (mais comuns):

\begin{itemize}
  \item \textbf{NOTICE}: quando alguma regra da linguagem não é seguida corretamente, 
  que pode ser um erro mas o programa segue sua execução normalmente, serve apenas 
  como alerta ao programador;
  \item \textbf{WARNING}: quando há realmente um erro, mas não-fatal, ou seja, 
  o programa continua em execução;
  \item \textbf{ERROR}: erro fatal, o programa é interrompido.
\end{itemize}

O programador \php~deve estar sempre atento a essas mensagens de erro, ler, interpretar e 
solucionar o problema indicado na mensagem. Porém, quando o programa estiver ``no ar'', 
ou seja, em produção, com usuários de verdade fazendo uso do programa, é importante que 
essas mensagens de erro sejam suprimidas, até porque essas mensagens de erro só interessam 
ao programador.

Para exibir a mensagem direta no navegador procure ativar a opção \texttt{display\_errors}, 
as funções serão ativadas para \textit{On} (ligada) ou \textit{Off} (desligada). O arquivo 
\texttt{php.ini} no sistema operacional Linux geralmente fica no diretório: 
\directory{/etc/php5/apache2/php.ini}. Por padrão a exibição das mensagens está desligada, porém, 
durante o desenvolvimento é interessante permitir que o \php~mostre as mensagens.

Veja o Exemplo:

\lstinputlisting[language=php,style=codigos]{codigos/cap11/error-exemplo.php}

A função \funcaostrtolower~recebe um texto e retorna o mesmo texto com letras minúsculas. 
Logo, se a função precisa receber um parâmetro e, no exemplo, esse parâmetro não foi fornecido, 
certamente vai gerar um erro:

\texttt{Warning: strtolower() expects exactly 1 parameter, 0 given in \\ /var/www/aula10/index.php on line 2.}

Essa mensagem será exibida se a opção \texttt{display\_errors} estiver setada para \textit{On} 
no arquivo \texttt{php.ini}.

Caso não queira ou não possa modificar o arquivo \texttt{php.ini}, é possível, fazer com que o erro seja 
exibido com a função \funcaoinitset, da seguinte maneira: \texttt{init\_set('display\_errors', 1)}, 
seguida da função \funcaoerrorreporting, da seguinte maneira: \texttt{error\_reporting(E\_ALL)}. 
Essas duas linhas indicam que tipo de erro o \php~deve exibir. O argumento \texttt{E\_ALL}, neste caso, 
indica que qualquer erro deve ser exibido. Por exemplo a função \texttt{error\_reporting(E\_WARNING)} 
vai exibir apenas os erros do tipo \textit{warning}.

\lstinputlisting[language=php,style=codigos]{codigos/cap11/error-exemplo-2.php}


\section{Exercícios}
\label{cap11-exercicios}

\begin{description}[labelindent=30pt]
  \item [Q. 01] Verifique o seu arquivo \texttt{php.ini} e veja se o parâmetro \texttt{display\_errors} está 
  setada para \textit{On}?  
  \item [Q. 02] Qual a importância do \php~emitir alerta de erros? 
  \item [Q. 03] Quais são os tipos mais comuns de erros e qual a importância de cada um?
  
\end{description}

\section{Desafio!}
\label{cap11-desafio}

Verifique o código a seguir:

\lstinputlisting[language=php,style=codigos]{codigos/cap11/error-exercicio.php}

Agora responda: o código possui erros? Se sim, quais e de que tipos? Esses erros prejudicariam a execução do programa? 