% ---
% Introdução a POO com PHP
% ---
\chapter{Introdução a POO com PHP}
\label{introducao-poo-php}
% ---

Ao final deste capítulo, o aluno terá as seguintes competências:
\begin{enumerate}
    \item Entender os conceitos da Programação Orientada a Objetos;
    \item Criar classes com atributos e métodos;
    \item Criar objetos;
    \item Herdar classes;
\end{enumerate}

\section{O que é programação orientada a objetos?}
\label{o-que-eh-poo}

A programação Orientada a Objetos (POO) é um estilo de programação, ou uma forma de programar, 
que ajuda a agrupar tarefas a respeito de um mesmo tema, organizar o código e entender como 
estruturar melhor a solução de um problema de modo que fique mais fácil também, posteriormente, 
fazer mudanças e dar manutenção ao programa. 

Usar esse estilo de programação orientado a objeto requer entender alguns conceitos. Como o 
próprio nome já diz, orientado a objetos indica que vamos usar ``objetos'', e esse é o primeiro 
conceito que vamos ver. \textbf{Objeto é a representação de algo que podemos atribuir características 
e ações}. Por exemplo, um carro é um ``objeto'', podemos dar a ele características (cor, modelo, 
fabricante etc.) e ações (buzinar, acelerar, estacionar etc.). Uma pessoa também pode ser 
representada num sistema como um objeto porque tem características (cor do cabelo, altura, 
peso, etc.) e executa ações (falar, andar, correr, dormir, etc.). 

Da mesma forma em um \site, podemos dizer que o usuário do \site~ é um objeto, um cliente, um 
produto ou serviço, uma foto, um vídeo etc. Em um comércio eletrônico, cada produto a venda 
é um ``objeto'', assim como o cliente que está comprando o produto e o carrinho de compras 
criado pelo cliente, são todos ``objetos''. Em uma rede social, cada foto, amigo, postagem, 
são objetos. 

Objeto, portanto, é um conceito e também um modo de pensar a programação, organizando e 
programando cada componente de uma forma separada (pensando em objetos), mas fazendo com 
que suas ações possam interagir com outros objetos, formando um sistema. Por exemplo, um 
objeto ``pessoa'', através de uma ação de ``dirigir'' pode interagir com um objeto ``carro''.

Um aspecto interessante da Programação Orientada a Objetos é o reuso. Reuso significa 
reutilizar, aproveitar o que já foi programado em um objeto para ser usado por outro objeto. 
Por exemplo, um carro e uma moto, apesar de serem objetos diferentes, compartilham de algumas 
características e ações. O carro e a moto tem peso, tem um fabricante, tem pneus, precisam 
ser abastecidos, são conduzidos por uma pessoa etc. Logo, se é programado a função de 
``abastecer'', essa única função serve para os dois (carro e moto), não é preciso repetir 
o mesmo código para os dois tipos de objetos.

\section{Classes e Objetos}
\label{classes-e-objetos}

Na Programação Orientada a Objeto o programador precisa desenvolver um modelo que vai ser 
usado como base para a criação de um objeto, com todas as suas características e comportamento, 
antes de usar esse objeto no sistema. Esse modelo do objeto chamamos de ``Classe'', e contêm 
a definição de um determinado objeto, ou como esse objeto vai funcionar. 

Uma ``classe'' é, portanto, um modelo de um objeto, com a programação de suas características 
(atributos) e ações (métodos).

Em \php, para criar uma classe usamos o comando \comandoclass. Exemplo:

\lstinputlisting[language=php,style=codigos]{codigos/cap12/poo-1.php}

No código anterior, usamos o comando \comandoclass~para criar uma nova classe chamada ``Carro''. Essa 
classe vai ser um modelo para todos os objetos do tipo ``carro''. 

Normalmente, como um meio de organizar melhor o sistema, colocamos cada classe em um arquivo \phpextensao~
diferente, mas não é necessário fazer isso por enquanto nesta aula, veremos no próximo capítulo 
como funciona essa organização dos arquivos.

Após programarmos uma classe, podemos criar um ou mais objetos com base nessa classe. O processo 
de criar um objeto com base nessa classe é chamada de ``instanciação''. Então, criar um objeto 
significa instanciar uma classe. Podemos fazer isso com o comando \comandonew, da seguinte 
forma:

\lstinputlisting[language=php,style=codigos]{codigos/cap12/poo-2.php}

Nesse código anterior criamos uma variável \texttt{\$meucarro}, e essa variável é um objeto 
do tipo ``Carro'', ou seja, um objeto criado com base na classe ``Carro''. Ainda com base na mesma 
classe podemos criar vários objetos, todos com as mesmas características e ações programadas 
na classe:

\lstinputlisting[language=php,style=codigos]{codigos/cap12/poo-3.php}

\section{Atributos}
\label{atributos}

No exemplo anterior criamos uma classe ``Carro'' mas não programamos mais nada. Vamos agora 
ver como podemos incluir e usar os atributos da classe. Atributo é o nome dado a uma 
característica ou propriedade da classe, que vai estar disponível a todos os objetos 
instanciados a partir dessa classe. Seguindo o mesmo exemplo, podemos ter uma classe Carro 
com os atributos de cor e marca.

\lstinputlisting[language=php,style=codigos]{codigos/cap12/poo-4.php}

Usamos o comando \comandopublic~seguido do nome de uma variável (\texttt{\$cor} e \texttt{\$marca}). 
Essas variáveis, dentro da classe, se tornam atributos da classe. Neste exemplo não foi 
inserido nenhum valor para essas variáveis, ou seja, são atributos sem valor padrão, os 
valores vão ser colocados depois do objeto criado, onde cada objeto pode ter uma cor e uma 
marca diferente.

Vamos agora instanciar essa classe, criando um objeto e atribuindo os valores:

\lstinputlisting[language=php,style=codigos]{codigos/cap12/poo-5.php}

Depois da classe Carro programada com dois atributos (cor e marca), podemos criar objetos 
com base nessa classe e usar em cada um deles os atributos disponíveis. No exemplo, 
\texttt{\$palio} e \texttt{\$fiesta} são dois objetos com as características de cor e marca, 
conforme descrito na classe, e cada um deles é um objeto distinto. Quando definimos 
``amarelo'' ao atributo ``cor'' do objeto \texttt{\$palio}, podemos por exemplo exibir com 
o comando \funcaoecho:

\lstinputlisting[language=php,style=codigos]{codigos/cap12/poo-6.php}

Observe que para os atributos de um objeto usamos o sinal de (\texttt{->}), e o nome do 
atributo não tem o sinal de \texttt{\$}, ficando apenas no nome do objeto. 

Veja outro exemplo, dessa vez os atributos da classe Carro com alguns valores pré definidos:

\lstinputlisting[language=php,style=codigos]{codigos/cap12/poo-7.php}

\section{Métodos}
\label{metodos}

Como visto anteriormente, uma classe tem atributos (suas características) e métodos 
(suas funções). Já vimos como trabalhar com os atributos, agora vamos conhecer os métodos, 
ou o que um objeto pode fazer.

Continuando com o exemplo da classe Carro, vamos imaginar que além dos atributos cor e 
marca, tem também uma função para dar a partida. Esse método da classe é construído da 
mesma forma que uma função do \php:

\lstinputlisting[language=php,style=codigos]{codigos/cap12/poo-8.php}

Agora sim, temos uma visão completa de uma classe, com seus atributos e métodos. Vamos 
criar um objeto Carro:

\lstinputlisting[language=php,style=codigos]{codigos/cap12/poo-9.php}

Observe que para executar um método do objeto usamos novamente o sinal de (->) seguido 
do nome do método e depois os parênteses, já que como foi visto anteriormente no curso,
se trata de uma função. Essa função \texttt{ligar()} do exemplo apenas exibe uma mensagem, 
mas podemos programa-la para fazer outras coisas, como mudar os atributos do próprio objeto. 

Outro exemplo, dessa vez criando um método que exibe os atributos do objeto:

\lstinputlisting[language=php,style=codigos]{codigos/cap12/poo-10.php}

Vamos ver um exemplo onde ao invés de exibir uma mensagem, o método \texttt{ligar()} 
muda um atributo.

\lstinputlisting[language=php,style=codigos]{codigos/cap12/poo-11.php}

Agora temos o termo \texttt{\$this}, que é usado somente dentro de uma classe e faz referência 
a algum método ou atributo da própria classe. No caso, o método ligar(), quando executado, 
vai mudar o valor do atributo ``ligado'' de ``não'' para ``sim''. Vamos ver nosso carro 
funcionando:

\lstinputlisting[language=php,style=codigos]{codigos/cap12/poo-12.php}

Claro que podemos melhorar essa classe, por exemplo, alterando o tipo de atributo 
``ligado" para booleano (\true~ou \false) e inserindo uma lógica para que o carro, depois 
de ligado, não possa ser ligado novamente, o mesmo para o método de desligar.

\lstinputlisting[language=php,style=codigos]{codigos/cap12/poo-13.php}

\section{Herança}
\label{heranca}

Um outro conceito dentro da programação orientada a objetos é o de Herança. Herança 
significa que é possível criar uma classe que herda atributos e métodos de outra classe. 
O sentido disso é organizar mais ainda, de forma que um código possa ser reutilizado, 
facilitando o desenvolvimento. 

Vamos imaginar, pegando o exemplo anterior do Carro, que existem vários tipos de carro 
diferentes, e cada categoria tem suas próprias características. Por exemplo, carros de 
corrida, ônibus, jipe, buggy etc. Então, seria complicado uma única classe representando 
todos os carros, porque existem categorias com características próprias, mas ao mesmo 
tempo eles compartilham de características semelhantes. Então, vamos deixar essas 
características semelhantes em um único lugar, e aproveitar essas informações para serem 
usadas em outras classes.

Vamos analisar essa seguinte classe:

\lstinputlisting[language=php,style=codigos]{codigos/cap12/poo-14.php}

Esta classe é uma classe genérica de um Carro, podemos dizer que todo carro tem uma cor, 
pode ser ligado e desligado. Podemos criar objetos a partir dessa classe para outros 
carros. Porém, um carro de Fórmula 1, por exemplo, pode ter outros atributos e métodos 
que um carro convencional não tem. Para aproveitarmos o que já é comum a todos os carros, 
podemos criar uma nova classe para a categoria de Fórmula 1 usando (herdando) a classe 
genérica de Carro e incluindo o que for específico da categoria. Vamos ver como isso 
funciona:

\lstinputlisting[language=php,style=codigos]{codigos/cap12/poo-15.php}

Observe a palavra \textbf{extends}, que é um comando para criar a herança. Este código 
cria uma nova classe Formula1 como uma extensão da classe Carro. A classe \texttt{Formula1} é como 
se fosse a classe ``filha'' da classe \texttt{Carro}. Vamos criar um objeto para \texttt{Formula1}:

\lstinputlisting[language=php,style=codigos]{codigos/cap12/poo-16.php}

Mesmo sem programar nada na classe Formula1, por ela ser ``filha" de Carro ela herda 
seus atributos e métodos, por isso podemos usar a função \texttt{ligar()} em um objeto do tipo 
\texttt{Formula1}. Agora vamos incrementar essa classe \texttt{Formula1}:

\lstinputlisting[language=php,style=codigos]{codigos/cap12/poo-17.php}

No exemplo anterior, incluímos 3 novos atributos a classe \texttt{Formula1}, piloto, equipe e velocidade, 
sendo a velocidade iniciada com o valor 0. Agora a classe \texttt{Formula1} tem seus próprios atributos, 
além daqueles que vem herdados da classe \texttt{Carro}, ou seja, a cor. Vamos incluir um método para 
aumentar a velocidade do \texttt{Formula1}:

\lstinputlisting[language=php,style=codigos]{codigos/cap12/poo-18.php}

Dessa forma, usando a Herança, se for necessário mudar alguma coisa no método de ligar o carro, 
essa mudança vai automaticamente refletir em todas as classes que herdam da classe \texttt{Carro} 
esse método. Esse processo torna a programação mais intuitiva e fácil de manter. 

\section{Exercícios}
\label{cap11-exercicios}

\section{Desafio!}
\label{cap11-desafio}