% ---
% Variáveis em PHP 
% ---
\chapter{Variáveis em PHP}
% ---

Ao final deste capítulo, o aluno terá as seguintes competências:
\begin{enumerate}
    \item Entender o que são variáveis e como o \php~ trabalha com elas; e
    \item Trabalhar com os diversos tipos de variáveis.
\end{enumerate}

\section{Variáveis}
\label{variaveis}

Variáveis são identificadores criados para guardar valores por determinado tempo. 
No \php elas são declaradas, inicializadas e armazenadas na memória RAM do servidor \web. 
Esse é um dos motivos pelo qual os servidores precisam de grandes quantidades de memória.

Quando desenvolvemos um \site e o disponibilizamos na \internet. É nosso desejo que ele seja
acessado pela maior quantidade de pessoas possível. Portanto, imagine um servidor com mais 
de 20 mil acessos simultâneos (mais de 20 mil pessoas visualizando o \site no mesmo momento).
Nesse processo são criadas variáveis diferentes para cada usuário, logo, isso faz com que o 
processamento que o servidor faz se intensifique. Por conta disso, o servidor deve ser um 
computador com bastante memória RAM.

Uma variável é inicializada no momento em que é feita a primeira atribuição. O tipo da
variável será definido de acordo com o valor atribuído. Esse é um fator importante \php, 
pois uma mesma variável pode ser de um tipo e pode assumir no decorrer do código outro 
valor de tipo diferente.

Para criar uma variável em PHP, precisamos atribuir-lhe um nome de identificação, 
sempre precedido pelo caractere cifrão (\textit{\$}). Observe um exemplo:

\lstinputlisting[language=php,style=codigos]{codigos/variaveis-1.php}

Modifique o código acima para imprimir na tela o valor da variável de 
nome \texttt{\$sobrenome}.

Observação! Podem acontecer erros na exibição das mensagens por conta 
das codificações de acentuação. Caso isso aconteça, mude a codificação 
do seu navegador ou utilize as metas de codificação. Para mudar a codificação
do Firefox aperte a tecla \keys{\Alt} para exibir a barra de menus e em seguida
percorra o caminho \menu[,]{Exibir, Codificação, Unicode}.

Nomes de variáveis devem ser significativas e transmitir a ideia de seu conteúdo 
dentro do contexto no qual está inserido.
Utilize preferencialmente palavras em minúsculo (separadas pelo caracter \texttt{\_}) ou 
somente as primeiras letras em maiúsculo quando você tiver duas ou mais palavras. 
Veja o exemplo abaixo. 

\lstinputlisting[language=php,style=codigos]{codigos/variaveis-2.php}

Dicas! \newline
\begin{itemize}
    \item Nunca inicie a nomenclatura de variáveis com números. Ex: \texttt{\$1nota;}
    \item Nunca utilize espaço em branco no meio do identificados da variável. Ex: \texttt{\$nome um;}
    \item Nunca utilize caracteres especiais (\texttt{! @ \# \& * | [ ] \{ \} $\backslash$ \^ } 
    entre outros) na nomenclatura das variáveis.
    \item Evite criar variáveis com nomes grandes demais em virtude da clareza do código-fonte.
    \item Com exceção de nomes de classes e funções, o \php~ é \textit{case sensitive}, ou seja, 
    é sensível a letras maiúsculas e minúsculas. Tome cuidado ao declarar variáveis. Por exemplo: 
    a variável \texttt{\$codigo} é diferente da variável \texttt{\$Codigo}.
\end{itemize}


\section{Tipos de Variáveis}
\label{tipos-de-variaveis}

O \php~ tem uma grande flexibilidade na hroa de operar com viaráveis. Quando definimos
uma variável juntamente com seu valor, o \php~ atribui um \textbf{tipo} à essa variável.
Isso permite que o programador não se preocupe muito na definição de tipos de 
variáveis (pois é feito de forma automática). Porém, devemos ter cuidado com atribuições
de valores, evitando erros nos cálculos.

\subsection{Tipo Inteiro}
\label{tipo-inteiro}
São os números que pertencem ao conjunto dos números inteiros, abragendo valores negativos
e positivos. No \php~ os valores máximos e mínimos do tipo inteiro depende da plataforma,
ou seja, do sistema operacional que o \php~ está sendo executado. No geral, esse número
pode ter um valor mínimo e máximo por volta de 2 bilhões (positivo e negativo).

Veja exemplos de códigos de atribuição de variáveis do tipo inteiro.

\lstinputlisting[language=php,style=codigos]{codigos/variaveis-inteiro.php}

\subsection{Tipo Ponto flutuante}
\label{tipo-ponto-flutuante}
Os números de ponto flutuante (\textit{float} e \textit{doubles}) são números com
casas decimais. No Brasil, usamos a vírgula (,) para escrever números decimais.
Nas linguagens de programação em geral, utilizamos o ponto (.). Veja exemplos de códigos
abaixo.

\lstinputlisting[language=php,style=codigos]{codigos/variaveis-ponto-flutuante.php}

\subsection{Tipo \textit{Booleano}}
\label{tipo-booleano}

Um \booleano~ expressa um valor lógico que pode ser \textbf{verdadeiro} ou 
\textbf{falso}. Para especificar um valor \booleano~ utilize a palavra chave
\texttt{\textbf{true}} para verdadeiro e \texttt{\textbf{false}} para falso. No exemplo 
a seguir, declaramos uma variável do tipo \booleano~ \texttt{\$exibir\_nome}, cujo conteúdo
é \textbf{\true}.

Esse tipo de variável é bastante útil quando estamos trabalhando com estruturas de decisão
ou estruturas de repetição com condição de parada. Esse assunto será abordado posteriormente.

\lstinputlisting[language=php,style=codigos]{codigos/variaveis-booleano-1.php}

Também podemos atribuir outros valores \textit{booleanos}~ para representação de valores 
\textbf{falso}, veja o exemplo abaixo.

\lstinputlisting[language=php,style=codigos]{codigos/variaveis-booleano-2.php}

\subsection{Tipo \textit{String}}
\label{tipo-string}

Uma \tipostring~ é uma cadeira de caracteres alfanuméricos, ou seja, podemos tanto usar números
quanto letras. Porém, é necessário declará-las utilizando aspas simples 
(\texttt{\textquotesingle \textquotesingle}) ou aspas duplas (\texttt{``''}). Veja o exemplo abaixo.

\lstinputlisting[language=php,style=codigos]{codigos/variaveis-string.php}

Caso seja necessário escrever na \tipostring~ os caracteres \texttt{\$}, 
\texttt{$\backslash$}, \texttt{\textquotesingle} ou \texttt{{``''}}, devemos utilizar a tabela abaixo. 

% \begin{table}[h]
\scalefont{1}
\caption{Caracteres especiais em \textit{strings}.}\label{tab:cap3-tipo-string}
\begin{center}
\begin{tabular}{|c|c|c|c|}
\hline
  \multicolumn{1}{|c|}{ \textbf{Sintaxe}}
&  \multicolumn{1}{|c|}{ \textbf{Significado}} \\
\hline
\hline
$\$$                      &  O símbolo \$ \\ \hline
$\backslash$ $\backslash$    &  O símbolo $\backslash$ \\ \hline
$\'$          &  Aspas simples \\ \hline
$\"$          &  Aspas duplas \\ \hline
\end{tabular}
\end{center}
\end{table}

\subsection{Tipo \textit{Array}}
\label{tipo-array}

\textit{Array} é uma lista de valores armazenados na memória. Os elementos de um \tipoarray~
podem ser de tipos diferentes (inteiro, decimal, \tipostring etc). Um \tipoarray~ pode crescer
dinamicamente com a adição de novos itens. O capítulo \ref{cap2} explica como
manipular esse tipo de estrutura.

\subsection{Tipo \textit{NULL}}
\label{tipo-null}

Ao atribuímos valor do tipo \texttt{NULL} (nulo) a uma variável, estamos determinando
que a variável não possui valor número ou alfanumérico e que seu valor é nulo. Veja o exemplo.

\lstinputlisting[language=php,style=codigos]{codigos/variaveis-null.php}

\section{Constantes}
\label{constantes}

\subsection{Constantes pré-definidas}
\label{constantes-pre-definidas}

O \php~ possui algumas constantes pré-definidas, como por exemplo, constantes que indicam a versão
do \php, o sistema operacional em que o \php~ está sendo executado, o arquivo em execução
e diversas outras informações. Para ter conhecimento de todas as constantes pré-definidas,
podemos utilizar a função \funcaophpinfo. Ela exibe uma tabela contendo todas as constantes
pré-definidas.

\subsection{Definindo constantes}
\label{definindo-constantes}

O programador também pode definir constantes para serem utilizadas no projeto. Para isso,
utilizamos a função \funcaodefine. Uma vez que o valor tenha sido definido, não poderá
ser alterado. A constante só pode conter valores inteiro, ponto flutuantes ou \tipostring.
Não podemos iniciar uma constante com uma variável do tipo \tipoarray~ por exemplo.

Um exemplo de utilização da função \funcaodefine~ pode ser visto a seguir.

\lstinputlisting[language=php,style=codigos]{codigos/constantes-define-1.php}
\lstinputlisting[language=php,style=codigos]{codigos/constantes-define-2.php}

O nome de uma constante tem as mesmas regras de nomenclatura de qualquer variável.
Ele é válido quando começa com uma letra ou \textit{underline} (\texttt{\_}) seguido
por qualquer número de ltras, números ou sublinhados.

A seguir listamos algumas diferenças entre constantes e variáveis.

\begin{enumerate}
  \item Constantes podem ser definidas e acessadas em qualquer lugar do código são
  indiferentes ao escopo;
  \item Constantes só podem conter valores numéricos ou alfanuméricos;
  \item Constantes não podem ter um sinal de cifrão (\texttt{\$}) no início;
  \item Constantes só podem ser definidas utilizando a função \funcaodefine~ e não por atribuição; e
  \item Constantes não podem ser redefinidas ou eliminadas após sua criação.
\end{enumerate}

\section{Conversão de variáveis}
\label{conversao-de-variaveis}

O \php~ permite que o programador converta um tipo de variável em outro tipo de varíavel.
As conversões mais comuns são as do tipo inteiro para decimal e do tipo decimal para inteiro.
Outra conversão bastante utilizada é de \tipostring para decimal ou inteiro.

A instrução utilizada para as conversões é denominada de \funcaotypeset. A ação é denominada
de \textit{typecasting}, ou seja, o ato de transformar uma variável de um tipo em outro tipo. A
tabela a seguir mostra os \textit{typeset} que iremos utilizar.

\begin{table}[h]
\scalefont{1}
\caption{Instruções de transformação de variáveis.}\label{tab:cap3-conversao}
\begin{center}
\begin{tabular}{|c|c|c|c|}
\hline
  \multicolumn{1}{|c|}{ \textbf{\textit{typecasting}}}
&  \multicolumn{1}{|c|}{ \textbf{Descrição}} \\
\hline
\hline
(int) ou (integer)    &  Converte em inteiro \\ \hline
(float) ou (double)   &  Converte em ponto flutuante \\ \hline
(string)              &  Converte em \tipostring \\ \hline
\end{tabular}
\end{center}
\end{table}

Veja alguns exemplos de códigos de conversões de tipos de variáveis.

\lstinputlisting[language=php,style=codigos]{codigos/conversao-1.php}

No primeiro e último exemplo de código não notamos diferença ao imprimirmos no navegador 
o valor das variáveis \texttt{\$temperatura} e \texttt{\$preco\_total} respectivamente. 
Mas se utilizarmos a função \funcaogettype~ teremos o tipo de variável correspodente. Veja
o código abaixo.

\lstinputlisting[language=php,style=codigos]{codigos/conversao-2.php}

\section{Exercícios}
\label{cap3-exercicios}


\section{Desafio!}
\label{cap3-desafio}
O desafio deste capítulo é criar um arquivo \phpextensao~ declarando variáveis de todos os tipos
estudados (inteiro, ponto flutuante e \tipostring). Imprimí-los no navegador e por fim, converter
os valores das variáveis do tipo inteiro e ponto flutuante para \tipostring.
