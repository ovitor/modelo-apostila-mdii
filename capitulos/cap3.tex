% ---
% Variáveis em PHP 
% ---
\chapter{Variáveis em PHP}
% ---

Ao final deste capítulo, o aluno terá as seguintes competências:
\begin{enumerate}
    \item Entender o que são variáveis e como o \php trabalha com elas; e
    \item Trabalhar com os diversos tipos de variáveis.
\end{enumerate}

\section{Variáveis}
\label{variaveis}

Variáveis são identificadores criados para guardar valores por determinado tempo. 
No \php elas são declaradas, inicializadas e armazenadas na memória RAM do servidor \web. 
Esse é um dos motivos pelo qual os servidores precisam de grandes quantidades de memória.

Quando desenvolvemos um \site e o disponibilizamos na \internet. É nosso desejo que ele seja
acessado pela maior quantidade de pessoas possível. Portanto, imagine um servidor com mais 
de 20 mil acessos simultâneos (mais de 20 mil pessoas visualizando o \site no mesmo momento).
Nesse processo são criadas variáveis diferentes para cada usuário, logo, isso faz com que o 
processamento que o servidor faz se intensifique. Por conta disso, o servidor deve ser um 
computador com bastante memória RAM.

Uma variável é inicializada no momento em que é feita a primeira atribuição. O tipo da
variável será definido de acordo com o valor atribuído. Esse é um fator importante \php, 
pois uma mesma variável pode ser de um tipo e pode assumir no decorrer do código outro 
valor de tipo diferente.

Para criar uma variável em PHP, precisamos atribuir-lhe um nome de identificação, 
sempre precedido pelo caractere cifrão (\textit{\$}). Observe um exemplo:

\lstinputlisting[language=php,style=codigos]{codigos/variaveis-1.php}

Modifique o código acima para imprimir na tela o valor da variável de 
nome \texttt{\$sobrenome}.

Observação! Podem acontecer erros na exibição das mensagens por conta 
das codificações de acentuação. Caso isso aconteça, mude a codificação 
do seu navegador ou utilize as metas de codificação. Para mudar a codificação
do Firefox aperte a tecla \keys{\Alt} para exibir a barra de menus e em seguida
percorra o caminho \menu[,]{Exibir, Codificação, Unicode}.

Nomes de variáveis devem ser significativas e transmitir a ideia de seu conteúdo 
dentro do contexto no qual está inserido.
Utilize preferencialmente palavras em minúsculo (separadas pelo caracter \texttt{\_}) ou 
somente as primeiras letras em maiúsculo quando você tiver duas ou mais palavras. 
Veja o exemplo abaixo. 

\lstinputlisting[language=php,style=codigos]{codigos/variaveis-2.php}

Dicas! \newline
\begin{itemize}
    \item Nunca inicie a nomenclatura de variáveis com números. Ex: \texttt{\$1nota;}
    \item Nunca utilize espaço em branco no meio do identificados da variável. Ex: \texttt{\$nome um;}
    \item Nunca utilize caracteres especiais (\texttt{! @ \# \& * | [ ] \{ \} $\backslash$ \^ } 
    entre outros) na nomenclatura das variáveis.
    \item Evite criar variáveis com nomes grandes demais em virtude da clareza do código-fonte.
    \item Com exceção de nomes de classes e funções, o \php~ é \textit{case sensitive}, ou seja, 
    é sensível a letras maiúsculas e minúsculas. Tome cuidado ao declarar variáveis. Por exemplo: 
    a variável \texttt{\$codigo} é diferente da variável \texttt{\$Codigo}.
\end{itemize}


\section{Tipos de Variáveis}
\label{tipos-de-variaveis}


\section{Desafio!}
\label{cap3-desafio}
O desafio deste capítulo é você configurar o \sublime~ de acordo com o seu gosto.
Modifique os seguintes itens:
\begin{enumerate}
  \item Tamanho e tipo da fonte.
  \item Cor do editor (\menu[,]{Preferences, Color Scheme, <opção>})
  \item Instalar o \plugin~ \texttt{phpdoc}~ (\url{https://packagecontrol.io/packages/PhpDoc}).
\end{enumerate}
