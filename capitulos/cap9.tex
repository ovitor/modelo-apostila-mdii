% ---
% Manipulação de arquivos 
% ---
\chapter{Manipulação de arquivos e diretórios}
\label{manipulacao-de-arquivos-e-diretorios}
% ---

Ao final deste capítulo, o aluno terá as seguintes competências:
\begin{enumerate}
    \item Abrir e fechar arquivos; e
    \item Editar conteúdo de um arquivo texto;
\end{enumerate}

Assim como em outras linguagens de programação é muito importante trabalharmos 
com manipulações de arquivos e diretórios em \php. Através dessa funcionalidade 
podemos manipular um arquivo ou diretório dentro do servidor \web, criando assim
arquivos responsáveis por guardar informações referentes à página ou referente
aos usuários do sistema. 

Trabalhar com arquivos, significa que devemos realizar no mínimo duas operações: 
abrir e fechar um arquivo.

\section{Criando e abrindo um arquivo}
\label{criando-e-abrindo-um-arquivo}

No \php~vamos utilizar a função \funcaofopen. Sua sintaxe é simples. Precisamos
passar o caminho (\textit{path}) e o nome do arquivo a ser criado além do modo de
abertura do arquivo. Veja a sintaxe:

\lstinputlisting[language=php,style=codigos]{codigos/cap9/sintaxe-fopen.php}

Os modos de abertura de um arquivo que o desenvolvedor pode utilizar são listados
abaixo. É importante que o programador entenda sobre cada um dos modos, para a 
correta utilização dos arquivos. Será apresentado apenas 3 modos (os mais utilizados).

\begin{itemize}
  \item \textbf{r} (\textit{read}): este modo abre o arquivo somente para leitura,
  ou seja, o programador não vai conseguir escrever nenhum dado no arquivo enquanto
  ele estiver aberto nesse modo;
  \item \textbf{w} (\textit{write}): abre o arquivo somente para escrita, caso o 
  arquivo não exista, tenta criá-lo (é necessário que o diretório tenha permissão
  de escrita); e
  \item \textbf{a+} (\textit{append}): abre o arquivo para leitura e escrita, 
  caso o arquivo não exista, o \php~tentará criá-lo.  
\end{itemize}

\begin{framed}
{\Large Importante!}

Para trabalharmos com arquivos eh sempre importante verificarmos se a pasta ou o arquivo 
tem permissões dentro do sistema operacional, caso isso não aconteça, o arquivo não 
será criado, lido ou até mesmo gravado. Portanto, vamos mudar as permissões da pasta
\texttt{aula08}. Execute os comandos a seguir a partir do terminal do Linux:

\lstinputlisting[language=bash,style=codigos]{codigos/cap9/permissao.sh}

\end{framed}

Vamos iniciar criando um arquivo de texto na pasta da nossa aula, no caso, 
\directory{/var/www/aula08}. Em seguida escreva no arquivo \texttt{index.php} o 
seguinte conteúdo:

\lstinputlisting[language=php,style=codigos]{codigos/cap9/index.php}

Esse código irá criar o arquivo \texttt{disciplinas.txt} caso ele não exista e
vai retornar o identificador para a variável \texttt{\$fid} (que significa 
\textit{file identifier} em inglês ou identificador de arquivo em português).
Ao acessar o endereço \url{http://localhost/aula08/} o seu navegador
vai apresentar apenas a mensagem ``Trabalhando com arquivos'', porém, se 
verificarmos o diretório \directory{/var/www/aula08}, vamos visualizar o arquivo
``disciplinas.txt''.

\section{Gravando em um arquivo}
\label{gravando-em-um-arquivo}

Após o uso do comando \funcaofopen, temos o identificador apontando para o 
arquivo. Com ele podemos fazer alterações ou manipulações. Nessa seção vamos 
gravar dados dentro do arquivo com o uso do comando \funcaofwrite. Veja a sintaxe
abaixo:

\lstinputlisting[language=php,style=codigos]{codigos/cap9/sintaxe-fwrite.php}

Os parâmetros da função \funcaofwrite~são descritos abaixo:

\begin{itemize}
  \item \textbf{identificador} (\textit{handle}): variável identificadora do
  arquivo. Essa variável é retornada pela função \funcaofopen; e
  \item \textbf{texto} (\tipostring): texto a ser gravado no arquivo.
\end{itemize}

Vamos modificar o arquivo \texttt{index.php} para conter o seguinte conteúdo:

\lstinputlisting[language=php,style=codigos]{codigos/cap9/index-2.php}

O segundo parâmetro - o texto a ser gravado - recebeu ao final da \tipostring~
os caracteres \texttt{\textbackslash r\textbackslash n}. Esses caracteres fazem com 
que o \php~``aperte um \enter'' ao salvar o arquivo.

\section{Fechando um arquivo}
\label{fechando-um-arquivo}

Todo arquivo aberto deve ser fechado após sua utilização. Caso contrário, ele não 
poderá ser reutilizado em algum outro local do código. Para fechar um arquivo
você deve passar o identificador para a função \funcaofclose.

\lstinputlisting[language=php,style=codigos]{codigos/cap9/sintaxe-fclose.php}


\section{Lendo de um arquivo}
\label{lendo-de-um-arquivo}

Após abrirmos um arquivo, outra operação que podemos efetuar é a leitura do conteúdo 
existente no arquivo. Essa operação é feita linha por linha. Podemos ler os 
valores existentes nos arquivos usando diversas funções. Nessa apostila 
veremos essa funcionalidade com a função \funcaofile. 

\subsection{Comando \funcaofile}
\label{comando-file}

O comando \funcaofile~ler um arquivo e retorna um \tipoarray~com todo o seu conteúdo.
Cada posição do \tipoarray~representa uma linha do arquivo começando pelo índice
0. Veja a sintaxe a seguir:

\lstinputlisting[language=php,style=codigos]{codigos/cap9/sintaxe-file.php}

Seguindo o exemplo do arquivo \texttt{index.php}, adicione, usando o \sublime,
outras disciplinas. O código responsável por realizar a leitura do arquivo é
apresentado em seguida.

\lstinputlisting[language=bash,style=codigos]{codigos/cap9/disciplinas.txt}
\lstinputlisting[language=php,style=codigos]{codigos/cap9/exemplo-file.php}

\section{Trabalhando com arquivos \csvextensao}
\label{trabalhando-com-arquivos-csv}

Os arquivos \csvextensao~são arquivos de formato regulamentado, ou seja, eles tem
uma estrutura certa a ser seguida. \texttt{CSV} significa \textit{Comma-Separated Values}
que traduzindo para o português é: valores separados por vírgula. Um exemplo de arquivo 
\csvextensao~pode ser visualizado abaixo:

\lstinputlisting[language=bash,style=codigos]{codigos/cap9/informacoes.csv}

A primeira linha do arquivo diz respeito ao cabeçalho, é nessa linha que informamos
quais campos serão gravados. No nosso exemplo, os campos que vamos guardar são:
\texttt{id}, \texttt{nome}, \texttt{endereco} e \texttt{cidade}. Nas linhas seguintes
nós vamos guardar os \textbf{registros}, ou seja, as informações. Veja o arquivo
\texttt{informacoes.csv} com a adição de informações dos usuários.

\lstinputlisting[language=bash,style=codigos]{codigos/cap9/informacoes-2.csv}

Crie esse arquivo com o aplicativo \sublime~e salve com o nome \texttt{informacoes.csv}
no diretório \directory{/var/www/aula08}. Iremos trabalhar com ele nos próximos
exemplos.

O código a seguir abre o arquivo \texttt{informacoes.csv} no modo \texttt{r}. Isso 
indica que o programa será capaz de trabalhar com o arquivo de uma maneira: 
apenas lendo o arquivo.

\subsection{Função \funcaogetcsv}
\label{funcao-getcsv}

O código abaixo realiza a leitura do arquivo \csvextensao~e guarda na variável
\texttt{\$dados} os valores de cada linha.

\lstinputlisting[language=php,style=codigos]{codigos/cap9/exemplo-csv.php}

\figurasimples{cap9-exemplo-csv}{Resultado do código acima.}

\begin{framed}
{\Large Desafio rápido!}

Altere o código exemplificado na seção \ref{funcao-getcsv} para exibir os dados
em uma tabela. O resultado final esperado é o apresentado na figura abaixo:

\figurasimples{cap9-desafio-rapido-1}{Resultado esperado do desafio rápido.}

\end{framed}

\section{Manipulando diretórios}
\label{manipulando-diretorios}

Existem alguns comandos básicos para manipulação de diretórios. Mostraremos
na apostila, apenas como obter o diretório atual, ou seja, o diretório em que o
arquivo \phpextensao~está sendo executado. Outras operações como criar e apagar
um diretório são explicadas na documentação oficial do \php.

\subsection{Comando \funcaogetcwd}
\label{comando-getcwd}

Essa função retorna o diretório atual do arquivo \php~que está sendo executado.
Por exemplo, se existe o arquivo \directory{/var/www/aula08/index.php} e nesse
arquivo \texttt{index.php} estiver escrito \texttt{\$diretorio = getcwd()},
então o valor da variável \texttt{\$diretorio} será \texttt{/var/www/aula08}.
Reescreva o código abaixo.

\lstinputlisting[language=php,style=codigos]{codigos/cap9/exemplo-getcwd.php}

\begin{framed}
{\Large Desafio rápido!}

Crie um diretório dentro de \texttt{/var/www/aula08} com seu nome. Por exemplo:
\directory{/var/www/aula08/joaquim}. Salve o código acima com o nome \texttt{index.php}
dentro desse novo diretório. Qual o resultado?
\end{framed}

\section{Exercícios}
\label{cap9-exercicios}
Lembre-se que todo arquivo desenvolvido para resolver as questões abaixo devem ser
guardados no diretório \directory{/var/www/aula08}.

\begin{description}[labelindent=30pt]
  \item [Q. 01] O que é manipulação de arquivos no \php? Cite 3 funções utilizadas.
  \item [Q. 02] Na função de abrir um arquivo (\funcaofopen) quais parâmetros são utilizados?
  Para que serve cada um deles?
  \item [Q. 03] Quais os modos de abertura de um arquivo? Diferencie esses métodos. 
  \item Para resolver as questões 4, 5, e 6, utilize como base o código apresentado
  abaixo. 
  \lstinputlisting[language=php,style=codigos]{codigos/cap9/exemplo-q04.php}
  \item [Q. 04] Que tipo de arquivo é gerado na linha 2 e onde o mesmo é criado?
  \item [Q. 05] Qual a finalidade do comando \funcaofwrite?
  \item [Q. 06] Qual o principal motivo de fecharmos o arquivo com o comando \funcaofclose?
  \item Para resolver as questões 7, 8, 9 e 10 crie o arquivo \texttt{arquivos.php} e desenvolva
  as seguintes funções que irão trabalhar com arquivos:
  \item [Q. 07] Uma função que recebe como parâmetro o nome ou caminho do arquivo e retorne o conteúdo 
  do arquivo. Exemplo:
  \lstinputlisting[language=php,style=codigos]{codigos/cap9/exemplo-q07.php}
  \item [Q. 08] Uma função que receba o nome ou caminho do arquivo e retorne a última linha do arquivo.
  \lstinputlisting[language=php,style=codigos]{codigos/cap9/exemplo-q08.php}
  \item [Q. 09] Uma função que receba o nome ou caminho do arquivo e uma linha a ser gravada no arquivo. 
  A função deve retornar 0 se tiver sido executada com sucesso ou 1 se o arquivo informado não existe.
  \lstinputlisting[language=php,style=codigos]{codigos/cap9/exemplo-q09.php}
  \item [Q. 10] Uma função que receba o nome ou caminho do arquivo e delete todo o conteúdo desse arquivo.
  \lstinputlisting[language=php,style=codigos]{codigos/cap9/exemplo-q10.php}
  \item Para resolver as questões 11 e 12 crie o arquivo de nome \texttt{informacoes.csv} de acordo com
  o código abaixo:
  \lstinputlisting[language=bash,style=codigos]{codigos/cap9/exemplo-q11}
  \item [Q. 11] Crie uma função que leia o arquivo acima e retorne o \tipoarray~com o conteúdo do
  arquivo.
  \item [Q. 12] Crie uma função que insira mais um registro no arquivo \texttt{informacoes.csv}. Exemplo
  de linha a ser inserida: \texttt{3;Luis;Cavalcante;"Rua D, n. 32";Caucaia}


\end{description}

\section{Desafio!}
\label{cap9-desafio}

Crie uma página com um formulário de cadastro para praias do Ceará, com as informações: nome da praia, 
distancia da capital e qualidade (de 0 a 10). Essas informações devem ser armazenadas em um arquivo. Na 
mesma página do formulário deve ter um link para que o conteúdo desse arquivo seja exibido na página.

