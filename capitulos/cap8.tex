% ---
% Manipulação de funções 
% ---
\chapter{Manipulação de funções}
\label{manipulacao-de-funcoes}
% ---

Ao final deste capítulo, o aluno terá as seguintes competências:
\begin{enumerate}
    \item Entender o propósito da criação de funções; e
    \item Criar e manipular funções no \php.
\end{enumerate}

A maioria das linguagens de programação permitem que o programador crie 
funções para serem utilizadas no código. As funções são um bloco de código 
definidos e que tem um objetivo específico. Elas podem ainda receber 
parâmetros (dados de entrada) e retornar um valor (dados de saída). 
A maior qualidade das funções é sua propriedade de ser reutilizada, ou seja, 
o programador declara uma vez e a utiliza em diversas partes do seu código. 

\section{Declarando uma função}
\label{declarando-uma-funcao}

Declarar uma função no \php~é simples. Essa seção irá lhe guiar pelo processo. 
O código abaixo apresenta a sintaxe de uma declaração de função. Na linha \ref{line:function} 
utilizamos a instrução \comandofunction~e em seguida o nome da função, os termos 
entre parenteses são os parâmetros daquela função. O bloco da função é definido 
pelos caracteres \texttt{\{} e \texttt{\}}, ou seja, toda vez que executarmos a 
função definida, o código entre as chaves é que será executado. Na linha \ref{line:return} 
a instrução \comandoreturn~é responsável por retornar 
o valor calculado. Nem sempre uma função recebe algum parâmetro, por isso, 
eles são opcionais, assim como a instrução \comandoreturn, que também é opcional.

\lstinputlisting[language=php,style=codigos]{codigos/cap8/funcao-sintaxe.php}

\section{Passagem de parâmetros}
\label{passagem-de-parametros}

Como vimos anteriormente na sintaxe da função, podemos passar ou não parâmetros pra uma função. 
porém existem alguns tipos de passagem de parâmetros: Por valor (\textit{by value}), por 
referência (\textit{by reference}) e por argumentos variáveis - esses últimos dois não 
serão abordados nessa apostila.

O tipo de passagem de parâmetro por valor tem a seguinte característica: se o conteúdo da 
variável for alterado, essa alteração não afeta a variável original. Exemplo:

\lstinputlisting[language=php,style=codigos]{codigos/cap8/funcao-exemplo-2.php}

No exemplo acima, a mensagem ``Seja bem vindo!'' não é alterada.

\section{Valor de retorno}
\label{valor-de-retorno}

Toda função pode opcionalmente retornar um valor, ou simplesmente executar os comandos 
e não retornar valor algum. No \php~não é possível que uma função retorne mais de um valor, 
mas é permitido fazer com que uma função retorne um valor composto, como o \tipoarray. 

Observe o exemplo abaixo. Nele criamos uma função para calcular o índice de massa 
corporal de uma pessoa (IMC). Essa função recebe como parâmetro dois argumentos (por valor). 
A altura representada pela variável \texttt{\$altura} e o peso representado pela 
variável \texttt{\$peso}. 

\lstinputlisting[language=php,style=codigos]{codigos/cap8/funcao-exemplo-1.php}

A função utiliza as duas variáveis para fazer o cálculo \texttt{\$peso / (\$altura * \$altura)}
No exemplo acima é apresentado a declaração da função e logo após a chamada
da função criada. Passamos como parâmetros para essa função o peso = 62 e a 
altura = 1.75. 

\begin{framed}
{\Large Desafio rápido!}

Crie um arquivo no diretório \directory{/var/www/aula08/} com o nome \texttt{funcoes.php}
e desenvolva 4 funções. Todas as funções devem receber como parâmetro, dois valores numéricos
(inteiros ou reais). 

A função de nome \texttt{somar} retorna a soma dos parâmetros. A função
de nome \texttt{subtrair} retorna a diferença do primeiro parâmetro pelo segundo.
Já a função \texttt{dividir} retorna o quociente da divisão do primeiro parâmetro pelo segundo.
Por fim, a última função de nome \texttt{multiplicar}, multiplica os dois valores e retorna
o resultado.
\end{framed}

\section{Escopo de variáveis em funções}
\label{escopo-de-variaveis-em-funcoes}

Um conceito importante em programação são os tipos de declarações de variáveis.
A visibilidade das variáveis dependerá de onde ela é declarada. Os acessos a essas 
variáveis podem ser definidos da seguinte forma:

\textbf{Variáveis locais} (código: \ref{lst:variaveis-locais}): São aquelas 
  declaradas dentro de uma função e não tem visibilidade fora dela;

\lstinputlisting[language=php,style=codigos,caption={Exemplo: Variáveis Locais},label={lst:variaveis-locais}]
{codigos/cap8/funcao-exemplo-3.php}

Nada foi impresso pois a variável \texttt{\$a} só existe dentro da função \texttt{teste()}.

\textbf{Variáveis globais} (código: \ref{lst:variaveis-globais}): São variáveis 
  declaradas fora do escopo de uma função, porém tem visibilidade (pode ser acessada) 
  de dentro de uma função sem ser passada como parâmetro. Para isso declaramos a 
  variável e fazemos a sua chamada logo após com o uso do termo \comandoglobal; e

\lstinputlisting[language=php,style=codigos,caption={Exemplo: Variáveis globais},label={lst:variaveis-globais}]
{codigos/cap8/funcao-exemplo-4.php}

\textbf{Variáveis estáticas} (código: \ref{lst:variaveis-estaticas}): Podemos 
  armazenar variáveis de forma estática dentro de uma função. Significa que ao 
  fazermos isso, temos o valor preservado independente da última execução. 
  Usamos o operador \comandostatic~para declaramos a variável. 

\lstinputlisting[language=php,style=codigos,caption={Exemplo: Variáveis estáticas},label={lst:variaveis-estaticas}]
{codigos/cap8/funcao-exemplo-5.php}

\section{Exercícios}
\label{cap8-exercicios}

\section{Desafio!}
\label{cap8-desafio}