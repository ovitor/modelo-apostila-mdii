% ---
% Manipulação de arrays 
% ---
\chapter{Manipulação de array's}
\label{manipulacao-de-arrays}
% ---

Ao final deste capítulo, o aluno terá as seguintes competências:
\begin{enumerate}
    \item Entender a estrutura do tipo \tipoarray;  
    \item Trabalhar com variáveis do tipo \tipoarray; e
    \item Trabalhar com \tipoarrays~de duas dimensões.
\end{enumerate}

O \tipoarray~é um tipo de variável que pode guardar diversos valores (chamados de: elementos 
de um \tipoarray), no \php, podemos definir como um conjunto de valores (elementos) ordenados. 
É possível relacionar cada valor com uma chave, no caso mais simples, a chave indica a 
posição (índice) em que um elemento do \tipoarray~se encontra.

\section{Criando um \comandoarray}
\label{criando-um-array}

Para criar um \tipoarray~pode-se utilizar a função \comandoarray. Veja a sintaxe e um exemplo:

\lstinputlisting[language=php,style=codigos]{codigos/array-1.php}

No exemplo acima, foi apresentado a forma mais simples de inicializar uma variável do 
tipo \tipoarray. A variável \texttt{\$alunos} pode ser percorrida utilizando o comando
\comandofor~por exemplo. Veremos mais a frente.

\section{\textit{Arrays} associativos}
\label{arrays-associativos}

Nos \tipoarrays~associativos determina-se um valor ou nome aos valores do \tipoarray.
Observe a sintaxe:

\lstinputlisting[language=php,style=codigos]{codigos/array-2.php}

No exemplo acima, utilizamos números para representar os índices do \tipoarray. 
Na seção \ref{acessando-um-array} aprenderemos como acessar elementos do \tipoarray.

A seguir, veremos outro exemplo. A figura abaixo representa um \tipoarray~que tem 
como elementos representação de cores. O \tipoarray~possui dez posições, 
cada posição representa uma cor, seu índice (chave) vai de 0 até 9. Veja:

\figurasimples{array-01}{\textit{Array} com 10 posições.}

Em código temos:

\lstinputlisting[language=php,style=codigos]{codigos/array-3.php}

No exemplo abaixo, o \tipoarray~associativo usa strings como índice, onde cada 
\tipostring~representa uma chave. Se quisermos representar as informações de uma
pessoa, podemos usar a variável do tipo \tipoarray.

\lstinputlisting[language=php,style=codigos]{codigos/array-4.php}

Observe que quando usamos \tipoarrays~associativos, a compreensão é mais fácil, 
dando mais legibilidade ao código, ou seja, é mais fácil entender o que cada valor
representa. Veja na representação gráfica abaixo:

\figurasimples{array-02}{\textit{Array} associativo com 4 posições.}

\subsection{Inicializando \tipoarrays}
\label{inicializando-arrays}

Outra forma de inicializar \tipoarrays~é adicionar valores de acordo com o seu 
uso. Veja no exemplo abaixo:

\lstinputlisting[language=php,style=codigos]{codigos/array-5.php}

\subsection{Acessando um \tipoarray}
\label{acessando-um-array}

Para acessarmos o valor do \tipoarray~que tem chaves, usamos \texttt{\$variavel['chave']}. 
Veja abaixo um exemplo de acesso ao valores armazenados em um \tipoarray~dessa natureza.

\lstinputlisting[language=php,style=codigos]{codigos/array-6.php}

Dessa forma podemos acessar um \tipoarray. Basta determinar o nome da variável
(\texttt{\$informacoes}) e a chave (pro nosso exemplo, os possíveis valores são:
\texttt{nome}, \texttt{endereco}, \texttt{bairro} e \texttt{cidade}), onde cada chave tem 
um valor já determinado. Caso a chave não exista, nada será impresso.

Para acessarmos o valor do \tipoarray~que (aparentemente) não tem chaves, usamos
os índices que se iniciam do 0 e vai até a quantidade de elementos do \tipoarray~menos 1.
Veja um exemplo abaixo:

\lstinputlisting[language=php,style=codigos]{codigos/array-10.php}

Por mais que não tenha os valores expressamente indicados na variável \texttt{\$alunos},
o \php~se encarrega de colocar os índices, permitindo ao programador, acessar os valores
de maneira fácil.

\section{Percorrendo um \tipoarray}
\label{percorrendo-um-array}

O processo de percorrer um \tipoarray~pode ser realizado por uma estrutura de 
repetição (\comandofor, \comandowhile, \comandodowhile~ou \comandoforeach). O exemplo a 
seguir apresenta uma abordagem utilizando o laço \comandoforeach.

\lstinputlisting[language=php,style=codigos]{codigos/array-7.php}

\begin{framed}
\textbf{{\Large Importante!}}
Sempre que se depararem com \tipoarrays, onde haja a necessidade de percorrer 
seus valores independentemente da chave, procure utilizar mecanismos de 
programação mais simplificados como o comando \comandoforeach.
\end{framed}

\section{Acessando um \tipoarray}
\label{acessando-um-array}

Quando criamos um \tipoarray~temos que ter em mente que estamos criando uma variável 
que possui vários valores e que os mesmo podem ser acessados a qualquer momento. 
Cada valor está guardado em uma posição que pode ser acessada através de uma chave.
A sintaxe para acesso simplificado de um \tipoarray~é a seguinte:
\lstinputlisting[language=php,style=codigos]{codigos/array-8.php}

Temos que ter cuidado ao passar uma chave para o \tipoarray, pois ela deve conter o 
mesmo nome de qualquer umas das chaves existentes no \tipoarray. Caso a chave não exista, 
o valor não poderá ser resgatado. A sintaxe acima retorna um valor contido na variável 
\texttt{\$variavel\_tipo\_array}. Verifique o exemplo abaixo: 

\lstinputlisting[language=php,style=codigos]{codigos/array-8.php}

No exemplo acima, apenas o valor ``Joaquim Silva'' será apresentado na tela, uma vez que
a chave \texttt{bairro} não está inserida no \tipoarray~\texttt{\$informacoes}.

\section{Alterando um \tipoarray}
\label{alterando-um-array}

Podemos alterar qualquer valor de um \tipoarray. É muito semelhante ao método de acesso. 
A diferença está na chamada do \tipoarray. É nesse momento que atribuímos um novo valor.
Veja o exemplo abaixo.

\lstinputlisting[language=php,style=codigos]{codigos/array-9.php}

Primeiro os valores iniciais da variável \texttt{\$informacoes} são apresentados na tela.
Em seguida, esses valores são modificados, por fim, os novos valores são impressos.
Veja outro exemplo:

\lstinputlisting[language=php,style=codigos]{codigos/array-11.php}

No exemplo acima, modificamos um \tipoarray~simples. Para acessá-los é necessário utilizar
os índices (que cada elemento tem em \tipoarrays~simples). Nesse caso, estamos alterando
o preço do produto na instrução \texttt{\$produto[1] += 3}. Poderíamos ter utilizado a forma
\texttt{\$produto[1] = \$produto[1] + 3} para obtermos o mesmo resultado.

\section{\textit{Arrays} com duas dimensões}
\label{arrays-com-duas-dimensoes}

Os \tipoarrays~com duas dimensões são estruturas de dados que armazenam os valores em 
mais de uma dimensão. Os \tipoarrays~que vimos até agora armazenam valores em uma dimensão, 
por isso para acessar às posições utilizamos somente um índice ou chave. Os \tipoarrays~de 
2 dimensões salvam seus valores na forma de filas e colunas e por isso, é 
necessário dois índices para acessar cada uma de seus valores.

Outra ideia que temos é que \tipoarrays~de duas dimensões são matrizes nos quais 
algumas de suas posições podem conter outros \tipoarrays. Na figura abaixo temos a 
representação de um \tipoarray~com duas dimensões.

\figurasimples{array-03}{\textit{Array} com duas dimensões.}

Um \tipoarray~de duas dimensões pode ser criado pela função \comandoarray~. Veja um exemplo
não funcional.

\lstinputlisting[language=php,style=codigos]{codigos/array-12.php}

Observe agora um exemplo real:

\lstinputlisting[language=php,style=codigos]{codigos/array-13.php}

\subsection{Acessando um \tipoarray~de duas dimensões}
\label{acessando-um-array-de-duas-dimensoes}

Para acessarmos o valor de um \tipoarray~de duas dimensões, basta colocar as 
duas chaves da posição que queremos acessar. É semelhante ao \tipoarray~de uma única dimensão.
Continuando com o exemplo acima, podemos acrescentar ao final do código as seguintes linhas.

\lstinputlisting[language=php,style=codigos]{codigos/array-14.php}

O código acima apresenta no navegador a primeira linha da matriz. Já o código abaixo
apresenta no navegador a última coluna da matriz.

\lstinputlisting[language=php,style=codigos]{codigos/array-15.php}

Perceba que quando queremos imprimir uma linha toda, o primeiro índice não muda, ou seja,
no nosso exemplo ele fica zero. Já quando queremos imprimir a coluna, deixamos o segundo
índice imutável (sem ser modificado). Desenhar a matriz que você quer representar no papel 
ajuda no entendimento da lógica de acesso aos elementos da matriz.

\subsection{Percorrendo um \tipoarray~de duas dimensões}
\label{percorrendo-um-array-de-duas-dimensoes}

Da mesma maneira que percorremos um \tipoarray~de uma dimensão, podemos percorrer um 
\tipoarray~de duas dimensões. Será necessário utilizar dois laços de repetição, seja
um comando \comandofor, \comandowhile~ou \comandoforeach. O \textbf{primeiro laço de repetição
é usado para percorrer as linhas}, o segundo laço utilizado para percorrer os elementos
das colunas. Acompanhe no código abaixo.

\lstinputlisting[language=php,style=codigos]{codigos/array-16.php}

\section{Visualizando \tipoarrays}
\label{visualizando-arrays}

Existe um conjunto de funcionalidades internas do \php~prontas para serem utilizadas nas variáveis
do tipo \tipoarray. Trata-se de funções pré-definidas, você pode encontrá-las facilmente no 
site \url{http://php.net}.

Abordaremos agora funções utilizadas exclusivamente para manipulação de \tipoarray, 
funções de acesso e visualização. Obviamente que não serão abordadas todas as funções, 
pois existem várias, mas mostraremos as mais utilizadas, e outras que são definidas como principais.

\subsection{Função \funcaovardump}
\label{funcao-vardump}

Essa função é muito usada por programadores que pretendem realizar \textit{debug}
(análise mais detalhada para encontrar supostos erros). Observe um exemplo prático:

\lstinputlisting[language=php,style=codigos]{codigos/array-17.php}

\figurasimples{array-vardump}{Saída da função \funcaovardump.}

A função \funcaovardump~é importante pois mostra no resultado o \tipoarray~completo
incluindo os tipos de variáveis que compõem o \tipoarray. No nosso exemplo, o \tipoarray~
tem 3 elementos. Mostra o tipo de cada um desses elementos e seus respectivos valores. 
Os elementos \texttt{id} e \texttt{nome} são do tipo \tipostring. Já o elemento 
\texttt{preco} é do tipo \tipofloat~(número real).

\subsection{Função \funcaoprintr}
\label{funcao-printr}

Imprime o conteúdo de uma variável assim como a função \funcaovardump, porém apresenta 
menos detalhes, alguns programadores a preferem por ser mais legível. Exemplo:

\lstinputlisting[language=php,style=codigos]{codigos/array-18.php}

\figurasimples{array-printr}{Saída da função \funcaoprintr.}

\subsection{Funções \funcaosort~e \funcaorsort}
\label{funcoes-sort-e-rsort}

Muitas vezes é necessário organizar o \tipoarray~de acordo com uma ordem. Por exemplo:
se temos um conjunto de alunos, podemos deixar seus nomes por ordem alfabética de A a Z.
Ou ainda a ordem pode ser de Z a A. Para isso, podemos usar as funções \funcaosort~e
\funcaorsort~respectivamente. Veja o exemplo abaixo.

\lstinputlisting[language=php,style=codigos]{codigos/array-19.php}

\section{Exercícios}
\label{cap6-exercicios}

\begin{description}[labelindent=30pt]
  \item [Q. 01] O que é um \tipoarray, e qual a sua principal finalidade?
  \item [Q. 02] O que é um \tipoarray~associativo?
  \item [Q. 03] O que é um \tipoarray~multidimensional?
  \item Para resolver as questões 4, 5, 6 e 7, utilize como base um \tipoarray~
  de 8 posições. Os valores do \tipoarray~devem ser nomes de pessoas que você conhece. 
  O nome da variável \tipoarray~deve ser \texttt{\$nomes}.
  \item [Q. 04] Apresente na tela o valor do terceiro elemento do \tipoarray.
  \item [Q. 05] O que aparece na tela se usarmos o código: \texttt{echo \$nomes[10]}
  \item [Q. 06] Apresente na tela os nomes da quarta posição até a penúltima posição.
  \item [Q. 07] Utilize o comando \comandoforeach~para apresentar na tela do navegador
  todos os nomes cadastrados.
  \item Para resolver as questões, 08, 09 e 10, utilize como base um \tipoarray~ associativo
  com as chaves \texttt{nome}, \texttt{sobrenome}, \texttt{endereco}, \texttt{bairro},
  \texttt{cidade} e \texttt{criado\_em}. O nome da variável \tipoarray~deve ser \texttt{\$informacoes}.
  \item [Q. 08] Apresente na tela do navegador o valor dos índices \texttt{nome} e \texttt{bairro}.
  \item [Q. 09] Troque o valor inicial do elemento de índice \texttt{criado\_em} para ``19/08/2016''.
  \item [Q. 10] Apresente na tela do navegador todos os valores do \tipoarray~\texttt{\$informacoes}.
  Utilize o comando \comandoforeach.
  \item Crie um vetor de duas dimensões de acordo com o gráfico abaixo e responda as questões 11 e 12.
  \lstinputlisting[language=bash,style=codigos]{codigos/cap6/exemplo-q11}
  \item [Q. 11] Apresente na tela a frase ``oi tudo bem? voce esta em casa?'' 
  \item [Q. 12] Imagine um diálogo possível e reproduza no navegador utilizando os índices dos
  elementos do \tipoarray.
\end{description}

\section{Desafio!}
\label{cap6-desafio}