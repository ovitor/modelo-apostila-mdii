% ---
% Estruturas de controle e repetição 
% ---
\chapter{Estruturas de controle e repetição}
% ---

Ao final deste capítulo, o aluno terá as seguintes competências:
\begin{enumerate}
    \item Trabalhar com estruturas de controle; e 
    \item Trabalhar com estruturas de repetição.
\end{enumerate}

As estruturas que veremos a seguir são comuns para várias linguagens de programação.
Entretanto é necessário que a apostila descreva a sintaxe dessas estruturas, resumindo
seu funcionamento.

É necessário ainda, entender o conceito de bloco. Um bloco consiste de vários
comandos agrupados com o objetivo de relacioná-los com determinado comando
ou função. Podemos (e devemos) usar blocos de comandos nas instruções que serão
vistas nesse capítulo (\comandoif, \comandofor, \comandowhile, \comandoswitch).
Os blocos de comandos devem ser utilizados para permitir que um conjunto de instruções
façam parte do mesmo contexto desejado.

Blocos em \php~ são delimitados pelos caracteres \texttt{\{} e \texttt{\}}.

\section{Comando \comandoif}
\label{comando-if}

Essa estrutura condicional está entre as mais usadas na programação. Sua finalidade é induzir um 
desvio condicional, ou seja, um desvio na execução natural do programa. Caso a condição (veja sintaxe abaixo) 
seja verdadeira, então serão executadas a instruções do bloco de comando. 
Caso a condição não seja satisfeita, o bloco de comando será simplesmente ignorado. 
Veja a sintaxe e em seguida um exemplo real.

\lstinputlisting[language=php,style=codigos]{codigos/estruturas-if-1.php}

Se o teste realizado no \comandoif~ for falso, ou seja, a expressão \texttt{\$media >= 7} der um
resultado \booleano~ falso, o bloco de comandos não será executado.

\subsection{Comando \comandoifelse}
\label{comando-if-else}

Um complemento do comando \comandoif~ é a adição da palavra chave \texttt{else}. Com esse termo
é possível tratar também as opções em que o teste é falso. Veja como fica a sintaxe da instrução
completa.

\lstinputlisting[language=php,style=codigos]{codigos/estruturas-if-2.php}

Como pode ser visto, o teste realizado \texttt{\$media >= 7} gerou um valor \booleano~
falso, resultando na execução do código relacionado ao bloco de comandos \texttt{else},
fazendo com que a mensagem exibida na tela do navegador seja ``Aluno em recuperação''.

O programador tem a possibilidade de adicionar quantos comandos \comandoifelse~
forem nescessários. Esses códigos são chamados de ``\texttt{ifs} encadeados''. 
Para facilitar, o \php~ criou a palavra chave \texttt{elseif} (tudo junto mesmo).
Veja a sintaxe de uso.

\lstinputlisting[language=php,style=codigos]{codigos/estruturas-if-3.php}

{\Large Desafio rápido!}

Faça um pequeno programa que utilize a idade de um nadador, a partir dela
o programa deve indicar em qual categoria o nadador irá concorrer no próximo
campeonato. Crianças menores de 5 anos não podem competir. 
Veja como deve ser a saída do seu programa. 

\figurasimples{cap4-desafio-rapido-1}{Resultado do primeiro desafio rápido.}

\section{Atribuição condicional (ternário)}
\label{atribuicao-condicional}

Esse tópico foi visto brevemente anteriormente. Vamos relembrar a sintaxe.

\lstinputlisting[language=php,style=codigos]{codigos/operadores-condicional-ternario-1.php}

Essa instrução se aplica quando queremos uma estrutura resumida, onde podemos ter um 
resultado mais direto, como por exemplo, atribuir um valor a uma variável dependendo de uma
expressão. Observe o exemplo abaixo: a variável \texttt{\$texto} receberá o valor 
``menor de 18'' ou ``maior de 18'' de acordo com o teste \texttt{\$idade > 18}.

\lstinputlisting[language=php,style=codigos]{codigos/operadores-condicional-ternario-2.php}
 
É uma estrutura semelhante ao comando \comandoifelse. Cabe ao programador escolher onde
cada uma das estruturas é melhor aplicada.

{\Large Desafio rápido!}

Faça um pequeno programa que verifique se um número é par ou ímpar. A variável
\texttt{\$num} deve ser inicializada pelo programador. O programador deve ainda
utilizar a estrutura de condição ternária e em um outro exemplo, utilizar o comando
\comandoifelse. Veja como deve ser a saída do seu programa. 

\figurasimples{cap4-desafio-rapido-1}{Resultado do primeiro desafio rápido.}

\section{Exercícios}
\label{cap5-exercicios}

\section{Desafio!}
\label{cap5-desafio}
O seguinte código é uma página inicial de um sistema de notas simplificado.
O programador deve inserir no próprio código as 3 notas do aluno. Se a média aritmética 
das notas for maior que 6, o aluno está aprovado, caso contrário o aluno se encontra de 
recuperação.

Esse código tem vários problemas e não está executando corretamente, identifique os erros, 
corrija e apresente ao professor.

\lstinputlisting[language=php,style=codigos]{codigos/cap4-desafio.php}