% ---
% Interações com o navegador
% ---
\chapter{Interações com o navegador}
\label{interacoes-com-o-navegador}
% ---

Ao final deste capítulo, o aluno terá as seguintes competências:
\begin{enumerate}
    \item Acessar informações do navegador;
    \item Criar e acessar dados em cookies;
    \item Criar e acessar dados em sessões;
\end{enumerate}

\section{Variável \variavelserver}
\label{variavel-server}

A linguagem de programação \php~também permite acessar informações do navegador 
(\browser) automaticamente. Isso pode ser muito útil quando queremos coletar 
informações sobre o cliente, como por exemplo, o tipo de navegador, qual o 
sistema operacional ou outras informações. O código a seguir mostra informações 
sobre o navegador do usuário:

\lstinputlisting[language=php,style=codigos]{codigos/cap10/metodo-server.php}

\variavelserver~é uma variável global do \php, do tipo \tipoarray, que armazena uma série 
de informações sobre o servidor e o cliente, como endereço do servidor, IP do servidor, 
nome do arquivo que está em execução etc. No caso, a chave \chavehttp~exibe as informações 
sobre o navegador. Abaixo um exemplo desse retorno:

\section{\titulocookies}
\label{cookies}

\cookies~são mecanismos para armazenar e consultar informações. Eles são armazenados na máquina 
do cliente, ou seja, o \cookie~está armazenado na máquina que acessou o site. Os \cookies possuem várias 
atribuições que são definidas pelo programador.

Por exemplo, Joaquim, aluno do e-jovem, vai ajudar sua mãe a comprar alguns eletrodomésticos, 
por isso acessou uma loja virtual, escolheu uma geladeira e um fogão, ou seja, adicionou eles ao 
carrinho de compras virtual da loja. Por algum motivo, Joaquim teve que sair e desligou o computador. 
Mais tarde ao voltar pra casa, volta a entrar na loja virtual e percebe que os itens adicionados 
ainda estão no carrinho de compras.

Esse é um exemplo básico de uso de \cookies~no desenvolvimento de sites. O site guardou a informação 
que os itens geladeira e fogão tinham sidos adicionados ao carrinho. Quando Joaquim acessou o site 
pela segunda vez, essas informações foram enviadas ao servidor do site, processadas e por fim, 
apresentadas ao usuário.

Os \cookies~são armazenados em uma pasta específica do navegador, por isso não é possível encontrá-lo 
na pasta ``Meus Documentos'' ou no diretório \directory{/home/aluno} por exemplo. Mesmo assim, 
pessoas mal intencionadas podem acessá-los. Por conta disso, o uso de \cookies~\textbf{não} é 
recomendado quando se trata de informações sigilosas. O usuário tem ainda a opção ``habilitar \cookies'' 
que pode ser desativada a qualquer momento pelo visitante, impedindo assim, a criação de \cookies~no 
sistema. Mas em cada navegador essa opção pode mudar de nome. 

A linguagem \php~atribui \cookies~utilizando a função \funcaosetcookie. Essa função deve ser incluída antes 
da \tag~\taghtml num arquivo \phpextensao.

A sintaxe do comando \funcaosetcookie possui muitos parâmetros. Abaixo está representada a função
com alguns desses parâmetros. 

\lstinputlisting[language=php,style=codigos]{codigos/cap10/sintaxe-setcookie.php}

A tabela abaixo relaciona cada um dos parâmetros com suas respectivas funções.

INCLUIR TABELA

Observe no exemplo abaixo a criação de um \cookie através da função \funcaosetcookie:

\lstinputlisting[language=php,style=codigos]{codigos/cap10/exemplo-setcookie.php}

Criamos uma variável na linha \ref{variable}, em seguida a função \funcaosetcookie recebe como 
parâmetro somente o seu nome e o valor a ser gravado.

Usaremos o navegador Mozilla Firefox para visualizar o \cookie criado. Acesse o site 
\url{http://localhost} na barra de endereços do navegador. Em seguida, você pode pressionar as 
teclas \keys{CTRL + I}, selecionar a aba ``Segurança'' em seguida selecionar o botão 
``Exibir cookies''. Lembre-se de criar o código acima primeiro e depois acessar a página pelo 
navegador de sua máquina. A imagem abaixo deverá ser exibida no seu monitor.

INCLUIR PRINTSCREEN!!

Veja que outras informações como caminho, enviar, e validade não foram especificados, 
porém aparecem na tela mesmo assim. No exemplo abaixo o programador atribui o tempo de vida do 
\cookie. para isso será necessário capturar o dia e a hora com a função \funcaotime e somar o tempo 
adicional de vida do \cookie. Esse tempo deve ser informado em segundos. Isso faz com que o \cookie~
exista na máquina do cliente de acordo com a quantidade de tempo determinado pelo programador.
Verifique o exemplo abaixo.

\lstinputlisting[language=php,style=codigos]{codigos/cap10/exemplo-setcookie-2.php}

Esse \cookie~tem a validade de 3600 segundos, ou seja 1 hora, sabendo disso e supondo que 
o usuário fez o acesso às 14:47:36, o \cookie~tem validade até às 15:47:36. Isso é muito 
importante para a programação dos \cookies. Se quisermos que ele exista por um determinado 
tempo, temos que calcular tudo em segundos. Veja um outro exemplo a seguir.

\lstinputlisting[language=php,style=codigos]{codigos/cap10/exemplo-setcookie-3.php}

Esse \cookie tem seu tempo de vida de 7 dias, pois 3600 segundos = 1 hora, 24 horas = 1 
dia e 7 * \texttt{horas\_de\_um\_dia} resulta em 7 dias.

\subsection{Acessando \titulocookies}
\label{acessando-cookies}

É possível acessar os valores dos \cookies~criados anteriormente. Para isso, basta utilizar 
a variável \variavelcookie\texttt{['nome\_do\_cookie']}, veja um exemplo:

\lstinputlisting[language=php,style=codigos]{codigos/cap10/acessando-cookies.php}

Observe agora um exemplo de um código utilizado para contar as visitas de um site 
usando \cookie:

\lstinputlisting[language=php,style=codigos]{codigos/cap10/contador-cookies.php}

O resultado na tela é de acordo com a quantidade de vezes que o cliente entrou no site 
ou atualizou o mesmo.

\section{Sessão}
\label{sessao}

Sessões são mecanismos muito parecidos com os tradicionais \cookies. A principal diferença é 
que as sessões são armazenadas no próprio servidor.

As sessões são métodos de manter (ou preservar) determinados dados no servidor, ao invés do computador 
do usuário. A sessão fica ativa (mantendo os dados armazenados) enquanto o navegador estiver aberto, 
ou enquanto a sessão não expirar (por inatividade, ou por conta da lógica do programa). Para criarmos 
uma sessão utilizaremos a função abaixo:

\lstinputlisting[language=php,style=codigos]{codigos/cap10/session-start.php}

Dessa forma estamos iniciando um conjunto de regras. Essa função deve sempre está no início do 
seu programa, com exceção de algumas regras. Agora trabalharemos com essa sessão, primeiro podemos 
determinar o tempo de vida da sessão com o seguinte comando:

\lstinputlisting[language=php,style=codigos]{codigos/cap10/session-cache.php}

Neste caso, a função \funcaosessioncache~vem antes da função \funcaosessionstart. O código faz
o seguinte: primeiro ele avisa que a sessão, quando iniciada, deve expirar em 5 minutos, e depois a inicia.
Com o uso da variável global \variavelsession~podemos gravar valores na sessão, veja um exemplo:

\lstinputlisting[language=php,style=codigos]{codigos/cap10/session-grava.php}

Criamos na sessão uma chave com o nome \texttt{minha\_sessao} e atribuímos a ela o valor gravado na 
variável do tipo \tipostring~chamada \texttt{\$var}. Essas informações ficam gravadas no servidor, 
em seguida é possível resgatar o valor da seguinte forma:

\lstinputlisting[language=php,style=codigos]{codigos/cap10/session-mostra.php}

Observe que \variavelsession~tem seu tratamento igual a uma variável do tipo \tipoarray. Para resgatar 
o valor gravado na sessão, basta fazer a chamada a essa variável \variavelsession~passando como chave, o nome 
especificado, no exemplo: \texttt{minha\_sessao}.

Abaixo temos um exemplo com o uso da função \funcaoisset, que verifica se uma variável 
existe ou não, retornando um valor \booleano~(\true~ou \false):

\lstinputlisting[language=php,style=codigos]{codigos/cap10/session-isset.php}

Para desativarmos uma sessão podemos utilizar dois tipos de mecanismos: um deles é o uso da função 
\funcaosessiondestroy~que tem como finalidade destruir a sessão por completo, e todas as informações 
nela contidas. Outra forma é apagarmos apenas uma informação (variável) gravada na sessão, 
com o uso da função \funcaounset. Veja abaixo exemplos de uso das funções \funcaosessiondestroy~e
\funcaounset.

\lstinputlisting[language=php,style=codigos]{codigos/cap10/session-destroy.php}

Usamos \funcaounset~quando queremos desativar determinada sessão, O exemplo abaixo destrói a 
sessão especificada:

\lstinputlisting[language=php,style=codigos]{codigos/cap10/session-unset.php}

Dessa forma destruímos a informação (variável) \texttt{minha\_sessao}, armazenada na sessão 
ativa, mantendo as demais informações intactas.


\section{Exercícios}
\label{cap10-exercicios}

\section{Desafio!}
\label{cap10-desafio}