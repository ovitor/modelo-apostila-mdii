% ---
% Operadores em PHP 
% ---
\chapter{Operadores em PHP}
% ---

Ao final deste capítulo, o aluno terá as seguintes competências:
\begin{enumerate}
    \item Trabalhar com operadores aritméticos; 
    \item Trabalhar com operadores relacionais; e
    \item Trabalhar com operadores de \tipostring.
\end{enumerate}

\section{Operadores}
\label{operadores}

Os operadores têm um papel importante dentro de qualquer linguagem de programação.
É através deles que podemos realizar diversas operações em um programa. Existem
operadores para atribuição, operadores aritméticos, operadores relacionais ou lógicos,
e por fim, operadores de \tipostring.

No \php, os operadores são utilizados constatemente e nesse capítulo iremos
aprender a trabalhar com a maioria deles.

\section{Operador de atribuição}
\label{operador-atribuicao}

O operador básico de atribuição é o caractere ``='' (igual). Com ele podemos atribuir 
valores as variáveis como foi visto em exemplos anteriores. Isto quer dizer que o 
operando da esquerda recebe o valor da expressão da direita, ou seja, a variável da 
esquerda contém o valor da direita do símbolo igual ``=''. Observe o exemplo abaixo:

\lstinputlisting[language=php,style=codigos]{codigos/operadores-atribuicao.php}

\section{Operadores de \textit{strings}}
\label{operadores-de-strings}
Os operadores de \tipostrings são utilizados para manipular o conteúdo de uma \tipostring.
O \php~ disponibiliza, basicamente, dois operadores de \tipostrings. O primeiro é o operador
de concatenação (\texttt{.}) - ele retorna a concatenação das variáveis envolvidas.
O segundo operador, é o operador de atribuição e concatenação (\texttt{.=}). Ele acrescenta
à variável do lado direito na variável do lado esquerdo do operador. Verifique os exemplos
abaixo.

\lstinputlisting[language=php,style=codigos]{codigos/operadores-string-1.php}

No exemplo acima, pode-se observar, na atribuição da variável \texttt{\$informacao} que, 
temos uma inicialização e atribuição de concatenação em uma mesma linha. Isso é possível
no \php, por mais que seja mais otimizado (mais rápido de ser processado), o código se
torna menos legível (mais difícil de ser entendido).

{\Large Pequeno desafio!}

Utilize o conhecimento adquirido nas aulas passadas para exibir, no navegador, as informações
separadas em células de uma tabela. A imagem a seguir deve ser o resultado apresentado. 

\figurasimples{cap4-pequeno-desafio-1}{Resultado do primeiro pequeno desafio.}

\section{Operadores Aritméticos}
\label{operadores-aritmeticos}

Os operadores aritméticos são utilizados para realizar cálculos matemáticos básicos, tais
como: soma, subtração, divisão e multiplicação. Os símbulos mais utilizados são descritos
na tabela abaixo.

\begin{table}[h]
\scalefont{1}
\caption{Símbolos matemáticos.}\label{tab:cap4-operadores-aritmeticos}
\begin{center}
\begin{tabular}{|c|c|c|c|}
\hline
  \multicolumn{1}{|c|}{ \textbf{Operação}}
&  \multicolumn{1}{|c|}{ \textbf{Operador}} 
&  \multicolumn{1}{|c|}{ \textbf{Exemplo}} 
&  \multicolumn{1}{|c|}{ \textbf{Resposta}} \\
\hline
\hline
Adição                    & + & \$a = 3 + 5 & 8   \\ \hline
Subtração                 & - & \$a = 6 - 2 & 4   \\ \hline
Multiplicação             & * & \$a = 2 * 5 & 10  \\ \hline
Divisão                   & / & \$a = 15 / 3 & 5  \\ \hline
Módulo (resto da divisão) & \% & \$a = 9 \% 2 & 1  \\ \hline
Negação                   & - & \$a = -3 & -3  \\ \hline
\end{tabular}
\end{center}
\end{table}


Na tabela acima fizemos operações básicas sem utilizar parênteses. O uso de parênteses
segue o mesmo princípio da matemática. Ele serve para dar prioridade a determinado
cálculo.

\section{Operadores Combinados}
\label{operadores-combinados}

No \php~ é possível combinar os dois operadores visto acima (de atribuição e artiméticos).
A partir deles é possível programar de forma mais ágil. Observe o código a seguir.

\begin{multicols}{2}

  \lstinputlisting[language=php,style=codigos]{codigos/operadores-atribuicao-combinada-1.php}
  \columnbreak

  \lstinputlisting[language=php,style=codigos]{codigos/operadores-atribuicao-combinada-2.php}

\end{multicols}

No código acima do lado esquerdo, declaramos a variável \texttt{\$peso} com um valor inicial,
em seguida utilizamos o operador combinado \texttt{+=} para incrementar o valor da
varíavel \texttt{\$peso}. Podemos obter o mesmo resultado com o código acima do lado direito.
Escrevemos menos no código do lado esquerdo, por isso, os programadores o utilizam bastante.

Já no código abaixo, utilizamos o operador combinado \texttt{.=}. Ele concatena as \tipostrings.

\lstinputlisting[language=php,style=codigos]{codigos/operadores-atribuicao-combinada-3.php}

Nesse exemplo utilizamos o operador de atribuição básico.

\lstinputlisting[language=php,style=codigos]{codigos/operadores-atribuicao-combinada-4.php}

A tabela abaixo lista os principais operadores de atribuição. 

\begin{table}[h]
\scalefont{1}
\caption{Operadores de atribuição combinados.}\label{tab:cap4-operadores-combinados}
\begin{center}
\begin{tabular}{|c|c|c|c|}
\hline
  \multicolumn{1}{|c|}{ \textbf{Operadores}}
&  \multicolumn{1}{|c|}{ \textbf{Descrição}} \\
\hline
\hline
=             & Atribuição simples   \\ \hline
+=             & Soma, em seguida, atribui   \\ \hline
-=             & Subtrai, em seguida, atribui   \\ \hline
*=             & Multiplica, em seguida, atribui   \\ \hline
/=  & Divide, em seguida, atribui   \\ \hline
\%=             & Tira o módulo, em seguida, atribui   \\ \hline
.=             & Concatena, em seguida, atribui   \\ \hline
\end{tabular}
\end{center}
\end{table}


\section{Operadores de decremento e incremento}
\label{operadores-de-decremento-e-incremento}

Os operadores exemplicificados nessa seção são usados para \textbf{somar} ou \textbf{subtrair}
o valor 1 (um) a variável. Esse cálculo pode ser feito antes ou depois da execução de 
determinada variável. A tabela abaixo mostra tais operadores.

\begin{table}[h]
\scalefont{1}
\caption{Operadores de incremento e decremento.}\label{tab:cap4-operadores-incremento-e-decremento}
\begin{center}
\begin{tabular}{|c|c|c|c|}
\hline
  \multicolumn{1}{|c|}{ \textbf{Operadores}}
&  \multicolumn{1}{|c|}{ \textbf{Descrição}} \\
\hline
\hline
\$b = ++\$a             & Incrementa o valor de \$a, e atribui à \$b (pré-incremento)  \\ \hline
\$b = \$a++             & Atribui à \$b, em seguida, incrementa o valor de \$a (pós-incremento)  \\ \hline
\$b = --\$a             & Decrementa o valor de \$a, e atribui à \$b (pré-decremento)  \\ \hline
\$b = \$a--             & Atribui à \$b, em seguida, incrementa o valor de \$a (pós-decremento)  \\ \hline
\end{tabular}
\end{center}
\end{table}


O exemplo abaixo mostra a comparação na forma da escrita. Os resultados alcançados serão
os mesmos.

\lstinputlisting[language=php,style=codigos]{codigos/operadores-incremento-1.php}

\section{Operadores relacionais}
\label{operadores-relacionais}

Os operadores relacionais são utilizados para realizar comparações entre valores (variáveis)
ou expressões. Essa comparação sempre resulta em um valor do tipo \booleano, ou seja,
verdadeiro (\true) ou falso (\false). Na tabela a seguir, listamos os operadores
que iremos trabalhar.

\begin{table}[h]
\scalefont{1}
\caption{Operadores relacionais.}\label{tab:cap4-operadores-relacionais}
\begin{center}
\begin{tabular}{|c|c|c|}
\hline
  \multicolumn{1}{|c|}{ \textbf{Operadores}}
&  \multicolumn{1}{|c|}{ \textbf{Nome}}
&  \multicolumn{1}{|c|}{ \textbf{Descrição}} \\
\hline
\hline
==            & Igual               & \specialcell{Resulta em \true~ se as \\ expressões forem iguais} \\ \hline
===           & Idêntico            & \specialcell{Resulta em \true~ se as \\ expressões forem iguais e do mesmo tipo} \\ \hline
!= ou <>      & Diferente           & \specialcell{Resulta em \true~ se as \\ expressões forem diferentes} \\ \hline
<             & Menor que           & \specialcell{Resulta em \true~ se a primeira \\ expressão for menor que a segunda expressão} \\ \hline
>             & Maior que           & \specialcell{Resulta em \true~ se a primeira \\ expressão for maior que a segunda expressão} \\ \hline
<=            & Menor ou igual que  & \specialcell{Resulta em \true~ se a primeira \\ expressão for menor ou igual a segunda expressão} \\ \hline
>=            & Maior ou igual que  & \specialcell{Resulta em \true~ se a primeira \\ expressão for maior ou igual a segunda expressão} \\ \hline
\end{tabular}
\end{center}
\end{table}


O exemplo de código abaixo descreve o uso de alguns operadores relacionais.

\lstinputlisting[language=php,style=codigos]{codigos/operadores-relacionais-1.php}

\subsection{Operador condicional ternário}
\label{operador-condicional-ternario}

É importante explicar o operador condicional ternário (símbolo ``?'') para termos um exemplo 
prático dos operadores relacionais. A sintaxe do operador condicional é exemplificada abaixo:

\lstinputlisting[language=php,style=codigos]{codigos/operadores-condicional-ternario.php}

A expressão 1 é sempre um teste a ser realizado com os operadores relacionais. A expressão
2 e expressão 3 são valores que serão atribuídos a variável \texttt{\$var}. Vamos ver um exemplo:

\lstinputlisting[language=php,style=codigos]{codigos/operadores-condicional-ternario-exemplo.php}

Qual será o resultado se mudarmos a expressão 1 (teste) para \texttt{\$num1 === \$num2}?

{\Large Tarefa rápida!}

Faça um pequeno programa que utilize todos os operadores relacionais descritos acima.
É necessário criar exemplos que os resultados sejam \true~ e \false~ para cada
um dos casos. Utilize também o operador condicional ternário.


\section{Exercícios}
\label{cap4-exercicios}


\section{Desafio!}
\label{cap4-desafio}
O desafio deste capítulo é criar um arquivo \phpextensao~ declarando variáveis de todos os tipos
estudados (inteiro, ponto flutuante e \tipostring). Imprimí-los no navegador e por fim, converter
os valores das variáveis do tipo inteiro e ponto flutuante para \tipostring.
