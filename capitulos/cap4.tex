% ---
% Operadores em PHP 
% ---
\chapter{Operadores em PHP}
% ---

Ao final deste capítulo, o aluno terá as seguintes competências:
\begin{enumerate}
    \item Trabalhar com operadores aritméticos; 
    \item Trabalhar com operadores relacionais; e
    \item Trabalhar com operadores de \tipostring.
\end{enumerate}

\section{Operadores}
\label{operadores}

Os operadores têm um papel importante dentro de qualquer linguagem de programação.
É através deles que podemos realizar diversas operações em um programa. Existem
operadores para atribuição, operadores aritméticos, operadores relacionais ou lógicos,
e por fim, operadores de \tipostring.

No \php, os operadores são utilizados constatemente e nesse capítulo iremos
aprender a trabalhar com a maioria deles.

\section{Operadores de \textit{strings}}
\label{operadores-de-strings}
Os operadores de \tipostrings são utilizados para manipular o conteúdo de uma \tipostring.
O \php~ disponibiliza, basicamente, dois operadores de \tipostrings. O primeiro é o operador
de concatenação (\texttt{.}) - ele retorna a concatenação das variáveis envolvidas.
O segundo operador, é o operador de atribuição e concatenação (\texttt{.=}). Ele acrescenta
à variável do lado direito na variável do lado esquerdo do operador. Verifique os exemplos
abaixo.

\lstinputlisting[language=php,style=codigos]{codigos/operadores-string-1.php}

No exemplo acima, pode-se observar, na atribuição da variável \texttt{\$informacao} que, 
temos uma inicialização e atribuição de concatenação em uma mesma linha. Isso é possível
no \php, por mais que seja mais otimizado (mais rápido de ser processado), o código se
torna menos legível (mais difícil de ser entendido).

{\Large Pequeno desafio!}

Utilize o conhecimento adquirido nas aulas passadas para exibir, no navegador, as informações
separadas em células de uma tabela. A imagem 

\figurasimples{cap4-pequeno-desafio-1}{Resultado do primeiro pequeno desafio.}

\section{Operadores Aritméticos}
\label{operadores-artimeticos}


\section{Exercícios}
\label{cap4-exercicios}


\section{Desafio!}
\label{cap4-desafio}
O desafio deste capítulo é criar um arquivo \phpextensao~ declarando variáveis de todos os tipos
estudados (inteiro, ponto flutuante e \tipostring). Imprimí-los no navegador e por fim, converter
os valores das variáveis do tipo inteiro e ponto flutuante para \tipostring.
