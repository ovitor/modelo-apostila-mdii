% ---
% PHP 
% ---
\chapter{\php}
% ---

Ao final deste capítulo, o aluno terá as seguintes competências:
\begin{enumerate}
	\item Entender a arquitetura cliente-servidor;
	\item Instalar o servidor web (nginx) e a linguagem \php; e
	\item Testar o ambiente de desenvolvimento. 
\end{enumerate}

\section{\phpcompleto}

O \phpcompleto, foi criado por \phpcriador~ em 1995 e originalmente chamado de 
“\textit{Personal Home Page Tools}” (Ferramentas para Página Pessoal). Com a 
aceitação do projeto, muitos programadores passaram a utilizar e propor mudanças,
surgindo assim, o \php~ que iremos conhecer hoje. O \php~ está atualmente na
versão 7.0, chamado de \php7 ou, simplesmente de \php. A nível de estudo, 
utilizaremos o \php~ \phpversao, pois é uma versão mais estável e muito 
utilizada no mercado.

O \php~ é uma linguagem de programação que funciona no lado do servidor, 
ele permite criarmos \sites dinâmicos, ou seja, o \site se comporta de acordo 
com a entrada de dados do usuário. Outros exemplos de linguagem semelhantes são 
ASP, JSP (Java) e Python.

A linguagem \php~ trabalha lado a lado com o \htmlcompleto, por conta disso vamos
precisar saber o básico de \html, principalmente as \tags~ de formulário. Devemos
lembrar que o \php~ tem pouca relação com o \layout~ ou eventos que compõem a 
aparência de uma página \web. Portanto, podemos dizer que a maior parte do que o
\php realiza é invisível para o usuário final. O internauta, ao visualizar a 
página desenvolvida em \php não será capaz de identificar que a página foi 
escrita utilizando a tecnologia disponibilizada pelo \php. 

Você arriscaria dizer que o Facebook foi desenvolvido com a linguagem \php?

\section{Arquitetura cliente-servidor}
\label{arquitetura-cliente-servidor}

Como visto na seção anterior, o \php~ funciona do lado do servidor. Para entendermos
melhor isso, é necessário entender a estrutura cliente/servidor. Muito utilizada
na \internet. A figura abaixo exemplifica de maneira simples a comunicação entre
cliente e servidor.

FIGURA

....

\section{Instalação do \php}
\label{instalacao-do-php}

Para que possamos utilizar o \php, devemos instalar a linguagem no nosso computador
de trabalho. Vamos instalar esses pacotes através do \terminal. Podemos abrir o
\terminal de várias maneiras. Veja duas delas listadas abaixo:

\begin{enumerate}
	\item clique com o botão direito na área de trabalho e escolha a a opção
	"Abra o Emulador de Terminal aqui"; e
	\item acione a combinação de teclas ALT + F2 e digite \xfceterminal.
\end{enumerate}

Em seguida escreva o comando abaixo no \terminal~ que acabamos de abrir. Por segurança
a senha de usuário será requisitada, e \textbf{ela não aparece ao ser digitada}.
Não se preocupe, digite a senha e ao final aperte enter. 

\begin{lstlisting}[language=bash, style=Comandos]
  $ sudo apt-get install php5 libapache2-mod-php5 php5-gd curl 
  	php5-curl php5-xmlrpc php5-cli
\end{lstlisting}

Se você estiver usando o Linux do Projeto e-Jovem, então esses pacotes já devem
ter sido instalados e você visualizou a seguinte tela.

\figurasimples{php-instalacao-ok}{Instalação do \php~ bem sucedida.}

\section{Instalação do \apache}
\label{instalacao-do-apache}

O servidor \apache~ é um dos principais aplicativos que fazem a \web~ funcionar.
Ele é responsável por interpretar os arquivos \phpextensao~ e retornar para o
cliente, apenas o que ele requisitou. A versão que vamos trabalhar é a \apacheversao.
O processo de instalação é parecido com o que foi utilizado no \php. Abra o
\terminal~ utilizando um dos passos da seção \ref{instalacao-do-php}.

\begin{lstlisting}[language=bash, style=Comandos]
  $ sudo apt-get install apache2
\end{lstlisting}

Se o sistema utilizado for o Linux do Projeto e-Jovem, então já temos o \apache
\apacheversao~ instalado (figura \ref{apache-instalacao-ok}). Digite no navegador 
Firefox o endereço de internet \url{http://localhost}. A tela será parecida com 
a mostrada na figura \ref{apache-verificacao-ok}.

\figuradupla{apache-instalacao-ok}{Instalação do \apache \apacheversao~ bem sucedida}
			{apache-verificacao-ok}{Verificação do \apache~ em execução. Digite \url{http://localhost} no navegador Firefox}

A figura \ref{apache-verificacao-ok} indica que o \apache~ está funcionando corretamente.
O arquivo apresentado acima pode ser encontrado no diretório \dirpadrao. Será 
essa a localização dos arquivos que vamos desenvolver. Ou seja, sempre que criarmos
um arquivo \phpextensao~ ele deverá ser salvo no \dirpadrao. 

Para que seja possível o usuário do sistema (no caso você) salve no \dirpadrao,
precisamos mudar a permissão de escrita do diretório. Vamos abrir o \terminal~
de acordo com o que foi mostrado na seção \ref{instalacao-do-php}. Com o \terminal~
aberto, digite o seguinte comando.

\begin{lstlisting}[language=bash]
  $ sudo chmod -R 777 /var/www 
\end{lstlisting}

O comando acima permite que o usuário comum do sistema grave arquivos no \dirpadrao.

O aplicativo \apache~ pode ser configurado para funcionar de diversas maneiras. 
Essa disciplina necessita apenas da configuração básica. Caso queira modificá-la, 
o aluno poderá ler mais sobre o \apache~ através do site: \url{http://httpd.apache.org/docs/2.2}.

Caso você use o sistema operacional Windows na sua casa, veja no apêndice 
TAL, lá é explicado como instalar o \php~ e o \apache~ no Windows.


\section{Testando o ambiente}
\label{testando-ambiente}

Após a instalação, devemos testar o nosso ambiente de desenvolvimento (composto
inicialmente por \php~ e \apache). Abra novamente o \terminal~, navegue até o
diretório \dirpadrao.