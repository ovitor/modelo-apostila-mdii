% ---
% PHP 
% ---
\chapter{\php}
% ---

Ao final deste capítulo, o aluno terá as seguintes competências:
\begin{enumerate}
	\item Entender a arquitetura cliente-servidor;
	\item Instalar o servidor web (\apache) e a linguagem \php; e
	\item Testar o ambiente de desenvolvimento. 
\end{enumerate}

\section{\phpcompleto}

O \phpcompleto, foi criado por \phpcriador~em 1995 e originalmente chamado de 
“\textit{Personal Home Page Tools}” (Ferramentas para Página Pessoal). Com a 
aceitação do projeto, muitos programadores passaram a utilizar e propor mudanças,
surgindo assim, o \php~que iremos conhecer hoje. O \php~está atualmente na
versão 7.0, chamado de \php7 ou, simplesmente de \php. A nível de estudo, 
utilizaremos o \php~\phpversao, pois é uma versão mais estável e muito 
utilizada no mercado.

O \php~é uma linguagem de programação que funciona no lado do servidor, 
ele permite criarmos \sites~dinâmicos, ou seja, o \site~se comporta de acordo 
com a entrada de dados do usuário. Outros exemplos de linguagem semelhantes são 
ASP, JSP (Java) e Python.

A linguagem \php~trabalha lado a lado com o \htmlcompleto, por conta disso vamos
precisar saber o básico de \html, principalmente as \tags~de formulário. Devemos
lembrar que o \php~tem pouca relação com o \layout~ou eventos que compõem a 
aparência de uma página \web. Portanto, podemos dizer que a maior parte do que o
\php~realiza é invisível para o usuário final. O internauta, ao visualizar a 
página desenvolvida em \php~não será capaz de identificar que a página foi 
escrita utilizando a tecnologia disponibilizada pelo \php. 

Você arriscaria dizer que o Facebook foi desenvolvido com a linguagem \php?

\section{Arquitetura cliente-servidor}
\label{arquitetura-cliente-servidor}

Como visto na seção anterior, o \php~funciona do lado do servidor. Para entendermos
melhor isso, é necessário entender a estrutura cliente/servidor. Muito utilizada
na \internet. A figura abaixo exemplifica de maneira simples a comunicação entre
cliente e servidor.

\figurasimples{arquitetura-cliente-servidor}{Ilustração de uma requisição web.}

Dessa figura, podemos tirar algumas palavras chaves. Que são:
\begin{description}[noitemsep]
  \item [Recurso] Item disponível na Internet (uma figura, uma página, um arquivo .css);
  \item [Cliente] Aquele que \textbf{requisita} algum recurso (navegador Firefox); e
  \item [Servidor] Aquele que \textbf{provê} algum recurso (servidor \apache).
\end{description}

Portanto, quando o \textbf{cliente}, ou seja, o internauta, faz uma \textbf{requisição} 
- digitando na barra de endereços do navegador o \site~
\url{http://projetoejovem.seduc.ce.gov.br} e pressionando \keys{ENTER} - o navegador 
se encarrega de fazer um pedido ao \textbf{servidor} que guarda o \site~do projeto e-Jovem.

\begin{description}
  \item [Passo 1] Usuário requisita página de internet acessando o navegador e digitando
  o \site;
  \item [Passo 2] O servidor \web~se comunica com a linguagem \php;
  \item [Passos 3 e 4] O \php~e o \html~se juntam em um só arquivo;
  \item [Passo 5] O servidor tem agora a página pronta para ser enviada ao usuário; e
  \item [Passo 5] O usuário recebe a página completa em seu navegador.
\end{description}

\section{Instalação do \php}
\label{instalacao-do-php}

Para que possamos utilizar o \php, devemos instalar a linguagem no nosso computador
de trabalho. Vamos instalar esses pacotes através do \terminal. Podemos abrir o
\terminal de várias maneiras. Veja duas delas listadas abaixo:

\begin{enumerate}
	\item clique com o botão direito na área de trabalho, escolha a opção \\
	\opcao{Abra o Emulador de Terminal aqui}; e
	\item acione a combinação de teclas \altfdois e digite \xfceterminal.
\end{enumerate}

Em seguida escreva o comando abaixo no \terminal~que acabamos de abrir. Por segurança
a senha de usuário será requisitada, e \textbf{ela não aparece ao ser digitada}.
Não se preocupe, digite a senha e ao final aperte enter. 

\begin{lstlisting}[language=bash,style=codigos]
  $ sudo apt-get install php5 libapache2-mod-php5 php5-gd curl 
  	php5-curl php5-xmlrpc php5-cli
\end{lstlisting}

Se você estiver usando o Linux do Projeto e-Jovem, então esses pacotes já devem
ter sido instalados e você visualizou a seguinte tela.

\figurasimples{php-instalacao-ok}{Instalação do \php~bem sucedida.}

\section{Instalação do \apache}
\label{instalacao-do-apache}

O servidor \apache~é um dos principais aplicativos que fazem a \web~funcionar.
Ele é responsável por interpretar os arquivos \phpextensao~e retornar para o
cliente, apenas o que ele requisitou. A versão que vamos trabalhar é a \apacheversao.
O processo de instalação é parecido com o que foi utilizado no \php. Abra o
\terminal~utilizando um dos passos da seção \ref{instalacao-do-php} e digite a 
seguinte instrução.

\lstinputlisting[language=bash,style=codigos]{codigos/instalacao-apache.sh}

Se o sistema utilizado for o Linux do Projeto e-Jovem, então já temos o \apache
\apacheversao~instalado (figura \ref{fig:apache-instalacao-ok}). Digite no navegador 
Firefox o endereço de internet \url{http://localhost} (sem os sinais de maior e menor que). 
A tela será parecida com a mostrada na figura \ref{fig:apache-verificacao-ok}.

\figuradupla{apache-instalacao-ok}{Instalação do \apache \apacheversao~bem sucedida}
			{apache-verificacao-ok}{Verificação do \apache~em execução. Digite \url{http://localhost} no navegador Firefox}

A figura \ref{fig:apache-verificacao-ok} indica que o \apache~está funcionando corretamente.
O arquivo apresentado acima pode ser encontrado no diretório \dirpadrao. Será 
essa a localização dos arquivos que vamos desenvolver. Ou seja, sempre que criarmos
um arquivo \phpextensao~ele deverá ser salvo no \dirpadrao. 

Para que seja possível o usuário do sistema (no caso você) salve no \dirpadrao,
precisamos mudar a permissão de escrita do diretório. Vamos abrir o \terminal~
de acordo com o que foi mostrado na seção \ref{instalacao-do-php}. Com o \terminal~
aberto, digite o seguinte comando.

\begin{lstlisting}[language=bash,style=codigos]
  $ sudo chmod -R 777 /var/www 
\end{lstlisting}

O comando acima permite que o usuário comum do sistema grave arquivos no \dirpadrao.

O aplicativo \apache~pode ser configurado para funcionar de diversas maneiras. 
Essa disciplina necessita apenas da configuração básica. Caso queira modificá-la, 
o aluno poderá ler mais sobre o \apache~através do site: \url{http://httpd.apache.org/docs/2.2}.

Caso você use o sistema operacional Windows na sua casa, veja no apêndice 
\ref{ap:instalacao-env-windows}, lá é explicado como instalar o \php~e o \apache~no Windows.


\section{Testando o ambiente}
\label{testando-ambiente}

\subsection{Função phpinfo}
\label{subsection:funcao-phpinfo}

Após a instalação, devemos testar o nosso ambiente de desenvolvimento (composto
inicialmente por \php~e \apache). Abra novamente o \terminal~, navegue até o
diretório \dirpadrao.

Neste diretório, iremos criar \textbf{uma pasta para cada aula} do curso, portanto, hoje
criaremos o diretório \diraula{01}~no caminho \dirpadrao. Abra novamente o \terminal~
e digite os seguintes comandos:

\begin{lstlisting}[language=bash,style=codigos]
  $ cd /var/www 
  $ mkdir aula01
\end{lstlisting}

Lembre-se! É importante que o aluno crie em cada aula um diretório 
específico para aquela aula.

Testar o ambiente significa verificar se está tudo funcionando como deve ser. Vamos criar um 
arquivo \phpextensao. Usaremos o programa editor de textos \gedit, acione as teclas \altfdois 
e digite na janela o nome do programa: \gedit. 

A figura \ref{fig:gedit-ok} representa um código simples escrito no editor de textos \gedit. 
Perceba que o arquivo \phpextensao~começa com os caracteres \phpinicio~e \phpfim. 
Todo arquivo \phpextensao~tem essa estrutura no início e no fim. O conteúdo desse arquivo é 
a função \funcaophpinfo. Ela apresenta para nós, todas as opções que estão configuradas no 
nosso \php. A figura \ref{fig:gedit-ok} mostra o resultado ao acessarmos a URL \url{http://localhost} 
no nosso navegador Firefox.

Salve o arquivo no diretório \diretorio{/var/www/aula01}. Esse arquivo deve ter o nome \arquivo{index.php}.

\figurasimples{gedit-ok}{Código PHP escrito no programa \gedit.}

\subsection{Instrução echo}
\label{subsection:funcao-echo}
Outro teste que vamos fazer é a utilização da instrução \funcaoecho. Edite o arquivo 
\arquivo{index.php} com o programa \gedit~e adicione na primeira linha o código
\texttt{<?php echo "Aula 01"; ?>}

\lstinputlisting[language=php,style=codigos]{codigos/echo.php}

O resultado pode ser visto na figura abaixo.

\subsection{Comentários no \php}
\label{subsection:comentarios-no-php}

O último tópico deste capítulo trata dos comentários que podemos escrever nos nossos
arquivos \phpextensao. Comentários são partes importantes do código desenvolvido.
Eles servem para ajudar a enteder melhor determinadas partes do código ou ainda
para descrever o que o código desenvolvido realiza. Outra funcionalidade
importante dos comentários são os de orientar outros programadores, permitindo assim
que os desenvolvedores trabalhem em conjunto de maneira mais fácil.

No \php, os comentários uma linha são indicados pelos caracteres \verb$//$ ou \verb$#$
e os comentários de múltiplas linhas são representados pelos caracteres 
\verb$/*$ (início) e \verb$*/$ (fim).

Veja o nosso arquivo \arquivo{index.php} após a adição dos comentários explicativos.

\lstinputlisting[language=php,style=codigos]{codigos/echo-com-comentarios.php}

\section{Exercícios}
\label{cap1-exercicios}

Para realizar as questões abaixo, crie um documento no diretório \directory{/var/www/aula01} 
de nome \texttt{ejovem.php}.

\begin{description}[labelindent=19pt]
  \item [Q. 1] Como o arquivo \phpextensao~deve ser inicializado? Ou seja, o que deve vir no
  começo e no final do arquivo?
  \item [Q. 2] Exiba a mensagem ``projeto e-jovem'' utilizando o comando \funcaoecho'.
  \item [Q. 3] Adicione um comentário de uma linha com os dizeres: ``Primeira aula de PHP''.
  \item [Q. 4] Adicione um comentário de múltiplas linhas com cada palavra da frase 
  ``comentário de múltiplas linhas'' em uma linha.
  \item [Q. 5] Exiba a mensagem ``Olá Aluno!'' utilizando a \tag~\taghum~e o comando \funcaoecho.
\end{description}

\section{Desafio!}
\label{desafio}
O desafio deste capítulo é você encontrar na tela do navegador a versão do \php~
que vamos trabalhar, a versão do \apache~além de nos mostrar qual o diretório
padrão em que os arquivos \phpextensao~devem ser salvos.